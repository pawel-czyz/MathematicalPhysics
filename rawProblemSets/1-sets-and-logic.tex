\documentclass[notitlepage, 11pt]{article}

\usepackage{polski}
\usepackage{amsmath}
\usepackage{amssymb}
\usepackage[utf8]{inputenc}
\usepackage[top=1in, bottom=1in, left=1in, right=1in]{geometry}
\usepackage{graphicx} 
\usepackage{fancyhdr}

\newcommand{\NN}{\mathbb{N}}
\newcommand{\ZZ}{\mathbb{Z}}
\newcommand{\RR}{\mathbb{R}}
\newcommand{\QQ}{\mathbb{Q}}
\newcommand{\dg}{^\circ}

\newcounter{itemnum}


\newenvironment{prob}
{\stepcounter{itemnum}
\paragraph*{\arabic{itemnum}.}}
{}

\makeatletter
\makeatother


\begin{document}
\pagestyle{fancy}
\lhead{Problem set 1 - Logic and sets}
\rhead{24.01.2018}
\vspace{1cm}

\begin{prob}
Prove that the following sentences are true:
	\begin{enumerate}
		\item $\neg(\neg p) \Leftrightarrow p$
		\item $p\vee \neg p$
		\item $\neg (p\wedge q) = (\neg p)\vee (\neg q)$
		\item $\neg (p\vee q) = (\neg p)\wedge (\neg q)$
		\item $(p\Rightarrow q)\Leftrightarrow (\neg p) \vee q$
		\item $0\Rightarrow 1$
	\end{enumerate}
\end{prob}

\begin{prob}
	Prove that $\{1,1,2,2,2\}=\{1,2\}$
\end{prob}

\begin{prob}
	Let $X$ be a set built from all sets such that $A\notin A.$ Prove that $X$ does not exist. Hint: what if $X\in X$? What if $X\notin X$?
\end{prob}

\begin{prob}
	Let $A,\,B\,C$ be sets. Prove that: 
	\begin{enumerate}
		\item $A\cup A=A$
		\item $A\cup B=B\cup A$
		\item $A\cup (B\cup C)=(A\cup B)\cup C$
		\item $A\cap A=A$
		\item $A\cap B=B\cap A$
		\item $A\cap (B\cap C)=(A\cap B)\cap C$
		\item $A\cap (B\cup C)=(A\cap B)\cup (A\cap C)$
		\item $A\cup (B\cap C)=(A\cup B)\cap (A\cup C)$
	\end{enumerate}
\end{prob}

\begin{prob}
	Prove that there is no set of all sets. Hint: assume there is one. Then you can select some sets to form a set that does not exist.
\end{prob}

\begin{prob}
	Prove that $A=B$ iff $A\subseteq B \wedge B\subseteq A.$
\end{prob}

\begin{prob}
	Prove the following set identites:
	\begin{enumerate}
		\item Let $A\subseteq B.$ Prove that $(A^c)^c = A$.
		\item Let $A,\, B\subset U$. Prove that $(A\cup B)^c = A^c\cap B^c$
		\item Let $A,\, B\subset U$. Prove that $(A\cap B)^c = A^c\cup B^c$
		\item $\{a\in A : a\in B\} = \{b\in B : b\in A\}$
	\end{enumerate}
\end{prob}

\begin{prob}
	\begin{enumerate}
	\item Let $A=\{1,2,3\}$. Find $2^A$. What is the number of elements in $2^A$? How is it related to the
	number of elements of $A$?
	\item Let $A$ be a finite set with $n$ elements. Using the approach in which you choose which elements belong
		to a subset, prove that $2^A$ has $2^n$ elements.
	\end{enumerate}
\end{prob}

\begin{prob}
	Let $A_i\subseteq U$ for $i\in I$. Prove that the definitions below agree with the definitions for finite $I$.
	\begin{align*}
		\bigcup_{i\in I}A_i &= \{a\in U : a\in A_i \text{ for at least one }i\in I\}\\
		\bigcap_{i\in I}A_i &= \{a\in U : a\in A_i \text{ for every }i\in I\}
	\end{align*}
\end{prob}

\begin{prob} Let $A_i\subseteq U,\, i\in I$ and 
	$$\sigma = \bigcup_{i\in I}A_i,\, \pi = \bigcap_{i\in I}A_i$$
	Prove that:
	\begin{enumerate}
		\item if $k\in I$, then $A_k\cup \sigma=\sigma$
		\item $\sigma\cap \pi = \pi$
	\end{enumerate}
\end{prob}

\begin{prob} Find infinite sum and intersection for the families of subsets of $\mathbb{R}$:
	\begin{enumerate}
		\item $A_i=(0,1/i)$ for $i=1,2,\dots$
		\item $B_i=[0,1/i)$ for $i=1,2,\dots$
	\end{enumerate}
\end{prob}

\begin{prob}
	Let $A=\{\{a\}, \{a,b\}\},\, B=\{\{c\},\{c,d\}\}$. Prove that $A=B$ iff $a=c\wedge b=d$. Such a set $A$ we call
	\textbf{the ordered pair} $(a,b)$ as it has the property $(a,b)=(c,d)$ iff $a=c$ and $b=d$. 
	Now you can forget how it has been constructed, and just remember this property.
\end{prob}

\begin{prob}
	Prove that $(a,(b,c))=(d,(e,f))$ iff $a=d\wedge b=e\wedge c=f$.
\end{prob}

\begin{prob}
	Check that defining $(a,b,c)$ as $((a,b),c)$ also works (so two ordered tuples are the same if they have the
	same first element, the same second element, ...)
\end{prob}

\begin{prob}
	Check that, in terms of sets, $(a,(b,c))\neq ((a,b),c)$, so formally we do need to stick to one convention. 
	However as we are interested in the property of ordered tuple, we will not distinguish them and denote both
	of them just as $(a,b,c)$.
\end{prob}
\\\\
\noindent We define \textbf{Cartesian product} as $A\times B = \{(a,b) : a\in A\wedge b\in B\}.$

\begin{prob}
	Do you remember the identification of $(a,(b,c))$ and $((a,b),c)$? Prove that
	$A\times (B\times C) = (A\times B)\times C$. Therefore we'll write it just as $A\times B\times C$ 
	without parentheness.
\end{prob}

\begin{prob}
	You can prove that $2^n>n$ for every natural number $n$. 
	\begin{enumerate}
		\item Prove that the formula works for $n=0$ (punch the first domino).
		\item Assume that for some $n$ you proved on some way that $2^n>n$. Using this, prove that $2^{n+1}>(n+1)$ (if $n$-th domino falls, then
		$n+1$-th domino also falls)
	\end{enumerate}
\end{prob}

\noindent You can also modify slightly the induction principle - sometimes you should start with number different than 0 or use different induction step
(start 0 and step 2 can lead to theorems valid for even numbers, step 0 and steps 1 and -1 can lead to theorems valid for all integers...) 
\begin{prob}
	\begin{enumerate}
	\item Prove that 6 divides 
		$n^3-n$ for all natural $n$.
	\item Prove that 6 divides $n^3-n$ for all integers $n$. 
		You can use a slight modification mathematical induction principle proving the implication 
		,,if the theorem works for $n$, it works also for $n-1$".
	\end{enumerate}  
\end{prob}

\begin{prob}
	(Bernoulli's inequality) Prove that for real $x > -1$ and natural $n\ge 1$, the following inequality holds:
	$$(1+x)^n\ge 1+nx.$$
\end{prob}

\begin{prob}
	In Mathsland there are $n\ge 2$ cities. Between each pair of them there is a one-way road. 
	\begin{enumerate}
		\item Prove that there is a city from which you can drive to all the other cities.
		\item Prove that there is a city to which you can drive from all the others.
	\end{enumerate}
\end{prob}

\begin{prob}
	Let $S\subseteq R$. We say that $S$ is \textbf{well-ordered} iff any non-empty subset $X\subset S$ has the smallest element.
	\begin{enumerate}
		\item Prove that reals and integers are not well-ordered.
		\item Assume that $X\subseteq \mathbb N$ doesn't have the smallest element. Define $A=\{n\in \mathbb N : \{0,1,\dots,n\}\cap X=\emptyset\}$
			and use mathematical induction to prove that $X$ is empty.
		\item Why are natural numbers well-ordered? 
	\end{enumerate}
\end{prob}

\begin{prob}
	(Thanks to Antoni Hanke) How many are there functions from the empty set to $\{1,2,3,4\}?$
\end{prob}

\begin{prob}
	Consider two functions: $f:\{0, 1\}\to \{0,1\}$ given by $f(x)=0$ and $g:\{0,1\}\to\{0\}$.
	Prove that $f=g$.
\end{prob}

\begin{prob}
	Let $f:A\to B$ and $g: C\to B,$ where $A\neq C$. Is it possible that $f=g$?
\end{prob}

\begin{prob}
	As we remember, $\mathbb{R}$ stands for well-known real numbers. Are the following functions surjective?
	\begin{enumerate}
		\item $f: \mathbb{R} \to \mathbb{R}, ~f(x)=x^3$
		\item $g: \mathbb{R} \to \mathbb{R}, ~g(x)=x^2$
		\item $h: \mathbb{R} \to \{5\}$
	\end{enumerate}
\end{prob}

\begin{prob}
	As we remember, $\mathbb{R}$ stands for well-known real numbers. Are the following functions injective?
	\begin{enumerate}
		\item $f: \mathbb{R} \to \mathbb R, ~f(x)=x^2$
		\item $h: \{0,1,2,3\} \to \mathbb R, ~h(x)=x$
	\end{enumerate}
\end{prob}

\begin{prob}
	Construct function that is:
	\begin{enumerate}
		\item surjective, but not injective
		\item injective, but not surjective
		\item neither injective nor surjective
		\item bijective
	\end{enumerate}
\end{prob}

\noindent Notice that if a function $f: A\to B$ is bijective, then we can construct a function $g:B\to A$ 
such that $f(g(b))=b$ and $g(f(a))=a$.
\begin{prob}
	Prove that, if exists, $g$ is unique.
\end{prob}

\noindent We call this function \textbf{the inverse function}: $g=f^{-1}.$

\begin{prob}
	Assume that $f^{-1}$ exists. Prove that $(f^{-1})^{-1}$ exists and is equal to $f$.
\end{prob}

If we have two functions: $f:A\to B$ and $g: B\to C$, we can construct the \textbf{composition} using formula:
$g\circ f: A\to C,~(g\circ f)(a) = g(f(a)).$

\begin{prob}
	Find functions $f,~g$ such that:
	\begin{enumerate}
		\item $g\circ f$ exists, but $f\circ g$ is not defined
		\item both $f\circ g$ and $g\circ f$ exist, but $f\circ g\neq g\circ f$
	\end{enumerate}
\end{prob}

\noindent Although function composition is not commutative, it is associative:
\begin{prob}
	Left $f:A\to B, g: B\to C, h: C\to D$. Prove that
	$$h\circ (g\circ f) = (h\circ g)\circ f.$$
\end{prob}
Therefore we can ommit the brackets and write just $h\circ g\circ f.$ We will use function composition very
often.

\begin{prob}
	\begin{enumerate}
		\item Prove that composition of two surjections is surjective.
		\item Prove that composition of two injections is injective.
		\item Prove that composition of two bijections is bijective.
	\end{enumerate}
\end{prob}

\begin{prob} We will rephrase the definition of the inverse function as follows:
	\begin{enumerate}
		\item If $X$ if a set, we define \textbf{the identity function} 
			$$\text{Id}_X=\{(x,x)\in X^2 : x\in X\}.$$
			Prove that it is indeed a function. What is it's domain?
		\item Let $f:A\to B,~g:B\to A$. Prove that $f=g^{-1}$ iff
			$$g\circ f = \text{Id}_A \text{ and } f\circ g = \text{Id}_B$$
	\end{enumerate}
\end{prob}

\begin{prob}
	What is the cardinality of $\{a, a+1, a+2, \dots, a+n\}$?
\end{prob}

\begin{prob}
	Let $A,\,B$ and $C$ be finite sets. Prove that:
	\begin{enumerate}
		\item $|2^A|=2^{|A|}$
		\item $|A\cup B|=|A|+|B|$ iff $A$ and $B$ are disjoint.
		\item $|A\setminus B|=|A|-|B|$ if $B\subseteq A.$
		\item $|A| \ge |B|$ if $B\subseteq A$. When does the equality hold?
		\item $|A\cup B| = |A| + |B| - |A\cap B|$
		\item $|A\cup B\cup C| = |A|+|B|+|C| - |A\cap B| - |B\cap C|-|C\cap A| + |A\cap B\cap C|$
	\end{enumerate}
\end{prob}

\begin{prob}
	Assume that $A$ and $B$ are finite sets. Prove that $|A|=|B|$ iff there is a bijection between $A$ and $B$.
\end{prob}

\begin{prob}
	Above we find the way of saying that two cardinalities are equal using existence of a bijection. Let's find a way to compare which is less using
	another kind of function.  
	\begin{enumerate}
		\item Let $O_n=\{1,2,\dots,n\}.$ Prove that there is no injection from $O_{n+1}$ into $O_n$. Hint: use mathematical induction.
		\item Let $A$ and $B$ be finite. Prove that there is an injection from $A$ to $B$ iff $|A| \le |B|.$ 
	\end{enumerate}
\end{prob}

\begin{prob}
	Using the above results, prove in one line that if there is an injection from $A$ onto $B$ and an injection from $B$ into $A$, then there exists
	a bijection from $A$ onto $B$.
\end{prob}

We say that sets $A$ and $B$ have the same caridnalities (or $|A|=|B|$) iff there is a bijection between $A$ and $B$.

\begin{prob}
	Let $A,\,B$ and $C$ be sets. Prove that if $|A|=|B|$ and $|B|=|C|$, then $|A|=|C|.$ Hint: find the bijection between $A$ and $C$.
\end{prob}

\begin{prob}
	Prove that:
	\begin{enumerate}
		\item $|\mathbb N| = |\mathbb Z|$.
		\item $|\mathbb N|=|\mathbb N\times \mathbb N|$.
		\item $|\mathbb N|=|\mathbb Q|$.
	\end{enumerate}
\end{prob}

\begin{prob}
	Prove that if $A\subseteq B$, then $|A|\le|B|.$
\end{prob}

\begin{prob}
	Let $A,\,B$ and $C$ be sets. Prove that if $|A|\le |B|$ and $|B|\le |C|$, then $|A|\le |C|$.
\end{prob}

\begin{prob}
	Here you can prove that there are more real numbers than naturals or rationals. We define $X=\{x\in \mathbb R : 0\le x\le 1\}$ and choose one
	 convention of writing reals (e.g 0.999... = 1.000..., so we can choose to use nines)   
	\begin{enumerate}
		\item Assume that you have written all the elements of $X$ in a single column. Can you find a real number that does not occur in the list? 
		\item Using the above, prove that $|\mathbb N| < |X|$
		\item Prove that $|\mathbb Q| < |\mathbb R|.$
	\end{enumerate}
\end{prob}

\begin{prob}
	We know that $|\mathbb R| > |\mathbb N|.$ Using binary system prove that $\mathbb R=2^\mathbb N.$ Do you see similarity between the previous result
	and $2^n > n$ for natural $n$?
\end{prob}

\begin{prob}
	\textbf{Cantor's theorem} You will prove that $|A|<\left|2^A\right|$ for any set $A$. Let $A$ be a set and $f:A\to 2^A.$
	\begin{enumerate}
		\item Consider $X=\{a\in A : a\notin f(a)\}\in 2^A$. Is there $x\in A$ for which $f(x)=X?$
		\item Is $f$ surjective? 
		\item Find an injective function $g: A\to 2^A.$
		\item Prove that $|A| < |2^A|$ for any set $A$.
		\item Use Cantor's theorem to prove that there is no set of all sets.
	\end{enumerate}
\end{prob}

\begin{prob}
	\textbf{Cantor-Schroeder-Bernstein theorem} Let's prove that if $|A|\le|B|$ and $|B|\le |A|$, then $|A|=|B|$ for any sets.
	\begin{enumerate}
		\item (Knaster-Tarski) Now assume that $F$ has \textit{monotonicity} property: $F(X)\subseteq F(Y)$ if $X\subseteq Y$. 
			Prove that $F(S)=S$, where:
			$$S=\bigcup_{X\in U} X, \text{~where~} U= \{Y\in 2^A : Y\subseteq f(Y)\}.$$
		\item (Banach) Let $f: A\to B$ and $g:B\to A$ be injections. 
			We introduce new symbol: $f[X]=\{b\in B : b=f(x) \text{ for some } x\in X\}$. Prove that
			function $$F:2^A\to 2^A,~F(X)=A-g[B-f[X]]$$
			has the monotonicity property.
		\item Using the above statements we know that there is $S$ defined above in the our case. Prove that function 
			$$h(x) = 
				\begin{cases}
					f(x), x\in S\\
					g^{-1}(x), x \notin S
				\end{cases}
			 $$  
			 is a bijection. You will need to prove that $X\setminus S\subseteq \text{Im}\,g$.
	\end{enumerate}
\end{prob}
\end{document}