%%%%%%%%%%%%%%%%%%%%% chapter.tex %%%%%%%%%%%%%%%%%%%%%%%%%%%%%%%%%
%
% sample chapter
%
% Use this file as a template for your own input.
%
%%%%%%%%%%%%%%%%%%%%%%%% Springer-Verlag %%%%%%%%%%%%%%%%%%%%%%%%%%

\chapter{Logic and sets}
\label{logic_and_sets} % Always give a unique label
% use \chaptermark{}
% to alter or adjust the chapter heading in the running head

If you are already fimilar with operations on logical formulas and sets, you may ommit this chapter.

\section{Logical formulas}
\label{sec:logic}
	Consider declarative sentences as "Water boils at $100^\circ$C" or "2+2=5". We can construct new sentences:
	\begin{enumerate}
		\item conjuntion (and): $p \wedge q$ is true if and only if $p$ is true and $q$ is true
		\item disjuntion (or): $p \vee q$ is true if and only if at least one of sentences $p,\, q$ is true
		\item implication: $p \Rightarrow q$ is false if and only if $p$ is true and $q$ is false. Intuitively,
			if you know that $p$ implies $q$ and $p$ is true, then $q$ also must be true
		\item negation (not): $\neg p$ is true if and only if $p$ is false 
		\item equivalence (iff, if and only if): $p\Leftrightarrow q$ means exactly 
			$(p\Rightarrow q)\wedge(q\Rightarrow p)$. Intuitively: if you know that two sentences are equivalent 				and one of them is true, the other is also true
	\end{enumerate}
	
\noindent Because mathematics is the art of being smart and lazy, we will assign value 1 to true sentences and 0 to
false sentences.

\begin{prob}
Prove that the following sentences are true:
	\begin{enumerate}
		\item $\neg(\neg p) \Leftrightarrow p$
		\item $p\vee \neg p$
		\item $\neg (p\wedge q) = (\neg p)\vee (\neg q)$
		\item $\neg (p\vee q) = (\neg p)\wedge (\neg q)$
		\item $(p\Rightarrow q)\Leftrightarrow (\neg p) \vee q$
		\item $0\Rightarrow 1$
	\end{enumerate}
\end{prob}

Equipped with this powerful machinery we can dive into basic set theory.

\section{Basic set theory}
\label{sec:basic_set_theory}

\subsection{Rough ideas}
In modern mathematics we do not define a set nor set membership, so heuristically you can think that set $A$
is a 'collection of objects' and $x\in A$ means that the object $x$ is inside this collection. We will assume that any finite collections of elements $\{x_1, x_2, \dots, x_n\}$ is a set 
(the empty set is called $\varnothing$ rather than $\{\}$), moreover we will assume that real numbers
form a set $\mathbb{R}$. We say that two sets are equal ($A=B$) iff they have the same elements 
($x\in A\Leftrightarrow x\in B$). Note, that we do not check how many times $x$ appears in $A$. We can
just say whether it inside or not.

\begin{prob}
Prove that $\{1,1,2,2,2\}=\{1,2\}$
\end{prob}

Not every 'collection of objects' is a set, as you can prove:
\begin{prob}
	Let $X$ be a set built from all sets such that $A\notin A.$ Prove that $X$ does not exist.\\
	Hint: what if $X\in X$? What if $X\notin X$?
\end{prob}

\subsection{A few ways of constructing new sets}
\noindent Therefore we assume that some sets (as finite sets or real numbers) exist and we will construct 
new sets from the given ones using a few rules. Assume that $A$ and $B$ are sets:
\begin{enumerate}
	\item Let's make a formula $F$ such that for every element $a\in A$, the value 
	$F(a)$ is true or false. We can then construct a set $S$ with all the elements from $A$ for wchich the formula: $a\in S$ iff $F(a)$ and $a\in A$. This set is written explicitly as $S=\{a\in A : F(a)\}$. 
	\item We can form the sum of two sets: $a \in A\cup B$ iff $a\in A$ or $a\in B$.
	\item We can contruct the intersection of two sets: $a\in A\cap B$ iff $a\in A$ and $a \in B$.
	\item We can construct the difference of two sets: $A\setminus B = \{a \in A : a\notin B\}$
\end{enumerate}

\begin{prob}
	Prove that there is no set of all sets.\\
	Hint: assume there is one. Then you can select some sets to form a set that does not exist.
\end{prob}

\subsection{Subsets and complements}

\noindent As we have some sets, we can try to compare them. We say that $A\subseteq B$ iff 
$a\in A\Rightarrow a\in B$ (or intuitively, each element of $A$ is also in $B$. We say that $A$ is 
a \textbf{subset} of $B$ or that $B$ is a \textbf{superset} of $A$. 

\begin{prob}
	Prove that $A=B$ iff $A\subseteq B \wedge B\subseteq A.$
\end{prob}

\noindent If we fix the set $B$, to each subset $A$
we can assign it's \textbf{complement}: $A^c=B\setminus A$. \footnote{It is not the best symbol possible as we need to have $B$ in mind.} Moreover, we will assume that for a set $A$ there exists it's \textit{power set}:
$2^A = \{X : X\subseteq A\}$.

\begin{prob}
	Prove the following set identites:
	\begin{enumerate}
		\item Let $A\subseteq B.$ Prove that $(A^c)^c = A$.
		\item Let $A,\, B\subset U$. Prove that $(A\cup B)^c = A^c\cap B^c$
		\item Let $A,\, B\subset U$. Prove that $(A\cap B)^c = A^c\cup B^c$
		\item $\{a\in A : a\in B\} = \{b\in B : b\in A\}$
	\end{enumerate}
\end{prob}

\begin{prob}
	Let $A=\{1,2,3\}$. Find $2^A$. What is the number of elements in $2^A$? How is it connected with the
	number of elements of $A$?
\end{prob}

\subsection{Cartesian product}
First of all, we need a useful concept:
\begin{prob}
Let $A=\{a, \{a,b\}\}\, B=\{c,\{c,d\}\}$. Prove that $A=B$ iff $a=c\wedge b=d$. Such a set $A$ we call
\textbf{the ordered pair} $(a,b)$ as it has the property $(a,b)=(c,d)$ iff $a=c$ and $b=d$. 
Now you can forget how it has been constructed, and just remember this property.
\end{prob}

\begin{prob}
Prove that $(a,(b,c))=(d,(e,f))$ iff $a=d\wedge b=e\wedge c=f$.
\end{prob}
\noindent Therefore it makes sense to write just
$(a,b,c)$ for $(a,(b,c))$ and define similarily such \textbf{ordered tuple} for four elements, five elements
and so on.
\begin{prob}
Check that defining $(a,b,c)$ as $((a,b),c)$ also works (so two ordered tuples are the same if they have the
same first element, the same second element, ...)
\end{prob}
\begin{prob}
Check that, in terms of sets, $(a,(b,c))\neq ((a,b),c)$, so formally we do need to stick to one convention. 
However as we are interested in the property of ordered tuple, we will not distinguish them and denote both
of them just as $(a,b,c)$. Such notational problems appear in various places in mathematics, so we need to
try to get used to them.
\end{prob}

\noindent We can now introduce anot
her way of creating new sets: let $A$ and $B$ be sets. Then we define their
\textbf{Cartesian product} as 
$$A\times B = \{(a,b) : a\in A\wedge b\in B\}.$$

\begin{prob}
	Do you remember the identification of $(a,(b,c))$ and $((a,b),c)$? Prove that
	$A\times (B\times C) = (A\times B)\times C$. Therefore we'll write it just as $A\times B\times C$ 
	without parentheness.
\end{prob}

\section{Functions}
\label{sec:intro_to_functions}

\subsection{Basics}
Consider two sets $A$ and $B$. We say that a subset $f\subseteq A\times B$ is a \textbf{function}
iff the following two conditions hold:
\begin{itemize}
	\item for every element $a\in A$ there is an element $b\in B$ such that $(a,b)\in f$
	\item if $(a,b)\in f$ and $(a,c)\in f$, then $b=c$
\end{itemize}
Therefore for each $a\in A$ there is exactly one $b\in B$ such that $(a,b)\in f$. Such $b$ will be called
\textbf{value of $f$ at point $a$} and given a symbol $f(a).$
\begin{prob}
	(Thanks to Antoni Hanke) How many are there functions from the empty set to $\{1,2,3,4\}?$
\end{prob}

We need to introduce more terminology: set $A$ is called \textbf{the domain of $f$}, set $B$ is called
\textbf{the codomain of $f$} and the function $f$ is written as $f: A\to B$.
\begin{prob}
	Consider two functions: $f:\{0, 1\}\to \{0,1\}$ given by $f(x)=0$ and $g:\{0,1\}\to\{0\}$.
	Prove that $f=g$.
	\footnote{Some mathematicians, as Bourbaki use an alternative definition of function - for them
	a function is the triple $(A,B,f)$, where $f$ is defined as in the our case. We see that this definition
	is incompatible with ours. Fortunately, as in the case with different definitions of ordered tuples, 
	this problem will never occur explicitly in the further chapters.}
\end{prob}

\begin{prob}
	Let $f:A\to B$ and $g: C\to B,$ where $A\neq C$. Is it possible that $f=g$?
\end{prob}

\subsection{Injectivity, surjectivity and bijectivity}
\noindent As we have already seen, there may be some elements in codomain that are not values of 
$f$. We define \textbf{the image of $f$} as:
$$\text{Im}\, f = \{b\in B : \text{there is } a\in A \text{ such that } b=f(a)\}.$$
We say that the function $f: A\to B$ is \textbf{surjective} (or \textbf{onto}) iff $\text{Im}\,f=B$.

\begin{prob}
	As we remember, $\mathbb{R}$ stands for well-known real numbers. Are the following functions surjective?
	\begin{enumerate}
		\item $f: \mathbb{R} \to \mathbb{R}, ~f(x)=x^3$
		\item $g: \mathbb{R} \to \mathbb{R}, ~g(x)=x^2$
		\item $h: \mathbb{R} \to \{5\}$
	\end{enumerate}
\end{prob}

If $f(a)$ uniquely specifies $a$ (if $f(a)=f(b)$, then $a=b$) we say that the function is \textbf{injective}
(or \textbf{one-to-one}). Are the following functions injective?
\begin{prob}
	As we remember, $\mathbb{R}$ stands for well-known real numbers. Are the following functions surjective?
	\begin{enumerate}
		\item $f: \mathbb{R} \to \mathbb R, ~f(x)=x^2$
		\item $h: \{0,1,2,3\} \to \mathbb R, ~h(x)=x$
	\end{enumerate}
\end{prob}
If a function $f$ is both surjective and injective, we say that is \textit{bijective}\footnote{If you prefer nouns: surjective function is called surjection, injective - injection
and bijective - bijection}.

\begin{prob}
	Construct function that is:
	\begin{enumerate}
		\item surjective, but not injective
		\item injective, but not surjective
		\item neither injective nor surjective
		\item bijective
	\end{enumerate}
\end{prob}

\noindent Notice that if a function $f: A\to B$ is bijective, then we can construct a function $g:B\to A$ 
such that $f(g(b))=b$ and $g(f(a))=a$.
\begin{prob}
	Prove that, if exists, $g$ is unique.
\end{prob}
We call this function \textbf{the inverse function}
\footnote{It becomes confusing when working on real numbers: $f^{-1}(x)$ is 
\textbf{not} $(f(x))^{-1}=1/f(x)$}: $g=f^{-1}.$
\begin{prob}
	Assume that $f^{-1}$ exists. Prove that $(f^{-1})^{-1}$ exists and is equal to $f$.
\end{prob}

\subsection{Function composition}
