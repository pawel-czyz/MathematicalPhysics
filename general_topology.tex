%%%%%%%%%%%%%%%%%%%%% chapter.tex %%%%%%%%%%%%%%%%%%%%%%%%%%%%%%%%%
%
% sample chapter
%
% Use this file as a template for your own input.
%
%%%%%%%%%%%%%%%%%%%%%%%% Springer-Verlag %%%%%%%%%%%%%%%%%%%%%%%%%%

\chapter{General topology}
\label{general_topology}
Now we are prepared enough to dive into general topology (often called point-set topology).

\section{Basics}
\subsection{Topology and open sets}
Consider a set $X$. A topology is a set $\mathcal{T}_X \subseteq 2^X $ such that:
\begin{enumerate}
	\item $\varnothing, X\in \mathcal{T}_X$
	\item if $A, B\in \mathcal{T}_X$, then $A\cap B\in \mathcal{T}_X$
	\item if $A_i\in \mathcal{T}_X$ for $i\in I$, then $\bigcup_{i\in I} A_i\in \mathcal T_X$
\end{enumerate}
Members of topology we call \textbf{open sets}

\begin{prob}
	Using mathematical induction prove that the intersection of finitely many open sets is open.
\end{prob}

You may wonder whether, for a given set, topology is unique. As you can prove, there are at least two (possibly more!) topologies.

\begin{prob}
	\textbf{Trivial topology} Prove that for any $X$, set $\{\varnothing, X\}$ is a topology.
\end{prob}

\begin{prob}
	\textbf{Discrete topology} Prove that for any $X$, it's power set $2^X$ is a topology.
\end{prob}

\noindent This gives the necessity of the notion of \textbf{topological space} that is a pair $(X,\mathcal T_X)$, where $\mathcal T_X$ is a topology on 
$X$. There are spaces, for which just one topology is commonly used in applications. In such situations mathematicians write topological space just as
$X$, assuming that the ,,preferred" topology is obvious to the reader. 

\begin{prob}
	


TODO - CO Z TYM ZROBIĆ?
\subsection{Definition of intervals}
\begin{align*}
	(a,b) &= \{x\in \mathbb R : a < x < b\} \text{ (open interval)}\\
	[a,b] &= \{x\in \mathbb R : a \le x \le b\} \text{ (closed interval)}\\
	(a,b] &= \{x\in \mathbb R : a < x \ge b\} \text{ (left-open, right-closed interval)}\\
	[a,b) &= \{x\in \mathbb R : a < x \ge b\} \text{ (left-closed, right-open interval)}
\end{align*}