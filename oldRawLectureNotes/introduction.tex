\chapter{Introduction}
\label{intro} % Always give a unique label
% use \chaptermark{}

There are many excellent books on mathematical physics and differential geometry, so a question araises - how does
this book differ from any other? I had a few aims working on it:
\begin{itemize}
  \item Understandable for any person that wants to learn. It does not matter
    if you are a physicists, mathematician, english literature major or a high-school student. If you have
    enough self-determination, you can understand the mathematics in this book.
  \item Self-containing. Mathematics is both broad and deep, so it must be split into many
    different  branches. But I personally found discouraging that if you want to read one book, as prerequsities you need
    to read two other books, and so on. Here, you can understand that everything contained here with no access to libraries or
    other mathematical books. Obviously, we don't cover the whole subject, but it is a good start to own research.
  \item Problem-solving approach. I want you to prove all the theorems in this book, with adjustable amount of hints.
      This way you can understand what we are actually doing, instead of ommiting proofs that look discouraging at the beginning.
  \item Abstract concepts first. We start with very abstract concepts and then move to examples and special cases. It is not always possible if we want to provide
    enough examples, but this is the aim. Starting from abstract, more general terms usually makes the whole situation easier - you have less properties and assumptions to use,
    so the solutions are more straightforward.
  \item "So what I told you was true, from a certain point of view." - many mathematical objects look differently for different mathematicians. We will always try to cover many
    "points of view" to increase the understanding of the subject.
  \item Objects and maps. We define precisely what are our objects and transformations, that are in some sense natural, that change one object into another. While we don't use the
    language of the category theory, you can get some taste.
  \item Properties, then construction. When we talk about a mathematical object, we usually think about it's \textit{properties}. Explicit construction is useful - as it proves
    the existence of the object under consideration - but usually hides many important properties of the object. Therefore we define objects by a few properties, then we think about
    theorems that can be proved using these initial properties (so we end up with many more properties) and then think how to construct the object having the initial properties.
  \item Notation abusements explained. Mathematics has been evolving for centuries in many different countries, so the notation
    is rather diverse and sometimes is not the best possible. We will abuse it as it is a standard in mathematical world, but you will always understand
    what objects are involved in expressions you are manipulating with.
  \item No jumps. In mathematics we prove theorems and then use these theorems to prove other theorems and so on. In many textbooks I know, these auxilary theorems are referenced
    as ,,Check section 3, problem 2.". I don't associate theorems with specific numbers and I don't like going to a specified section. Therefore I reference theorems by their mathematical content or commonly used name, rather than an artificial number. I believe that you will be able to prove such
    mentioned theorems quickly and without problems.
  \item You will encounter two types of problems in this book - some of them you will encounter in the text, and they are called exercises. These are strongly related to the investigated subject and are essential for the continuity of the lecture. Others, called problems, you will find at the end of
  sections of chapters. These are problems that does not need to be connected with the discussed subject at all. If you read the book carefully, without jumps, you will be able to solve all of them. But it will give you an opportunity to come up with new insights - without subject specified you'll need to come up with ideas what tools, methods and theorems will be useful. I hope this helps you becoming a scientist.\footnote{I recommend watching an excellent talk given by Barbara Oakley "Learning how to lean" at Google. It's available on YouTube, under the link \texttt{https://www.youtube.com/watch?v=vd2dtkMINIw}. I especially recommend to think about focused and diffusive modes.}
\end{itemize}

We use the following notation: \textbf{bold} will be used for definitions of new objects, and \textit{italics} will be used for additional subtle remarks that should be taken into
account. We use footnote\footnote{Like this one.} to provide additional comments.


Remember that the subject is big and it may be very hard to finish the book in just one day. I strongly advise working on it every day starting from just two minutes a day
and increasing the time spend every week. I tried to make the learning curve flat, what leghtens the book.
Any mistakes are my own failure and I would be grateful if you pointed them to me. Also any suggestions and comments are welcome. You can
create new issues on GitHub: \texttt{https://github.com/pawel-czyz/MathematicalPhysics} or write an email to \texttt{pczyz@protonmail.com}.
Good luck on your road!
%
