% !TEX root = book.tex

\section{Propositional calculus}
\subsection{New sentences from old}
\label{sec:logic}
Consider declarative sentences as "It's raining in Oxford now." or "2+2=5" that can be either true or false. There are many ways how to construct new sentences and decide
whether they are true or not.

\begin{definition}
  % Equivalence - definition
  Consider sentences $p$ and $q$. We say that they \textbf{are equivalent} (we write then $p\Leftrightarrow q$) if they are either true or false simultaneously.
  If $p$ and $q$ are equivalent, we usually say "$p$ if and only if $q$" of even "$p$ iff $q$".
\end{definition}

\begin{example}
  % Equivalence - trivial example
  Sentences "Each square is a rectangle" and "2+2=3+1" are both true, so trivially they are equivalent.
\end{example}

\begin{example}
  % Equivalence - example, people in a room
  Let $p$ be a sentence "There is an odd number of people in this room." and $q$ be "If one person enters the room, then the number of people becomes even".
  We \textit{do not know} if \textit{any} of these sentences is true - it would require to count all the people in the room! But if $p$ is true, then also $q$ must be true and vice versa - if $q$ is true, then also $p$ must be true. Therefore we can say that $p$ and $q$ are equivalent, or write $p\Leftrightarrow q$.
\end{example}

\begin{exercise}
  Prove that $(p\Leftrightarrow q) \Leftrightarrow (q\Leftrightarrow p)$. Hint: what does the sentence in the first bracket mean? What about the second? Why are they equivalent?
\end{exercise}

\begin{exercise}
  % Equivalence - transitivity
  Prove that if we know that $p\Leftrightarrow q$ and we know that $q\Leftrightarrow r$, then also $p\Leftrightarrow r$.
\end{exercise}

\begin{definition}
  % Conjunction - definition
  Consider sentences $p$ and $q$. We say that their \textbf{conjunction} $p\wedge q$ is true iff both of them are true. Usually conjunction of $p$ and $q$ is
  referred as "$p$ and $q$".
\end{definition}

\begin{example}
  % Conjunction - simple example
  Sentence: "(2+2=5) and (2+1=3)" is false, as one of them (namely, the first one) is false.
\end{example}

\begin{exercise}
  % Conjunction - commutativity
  Let $p$ and $q$ be two sentences. Prove that $p\wedge q$ is true if and only if $q\wedge p$ is true. As we can swap two elements, we say that conjunction is \textbf{commutative}.
\end{exercise}

\begin{exercise}
  % Conjunction - associativity
  Let $p,\, q,\, r$ be three sentences. Prove that $(p\wedge q)\wedge r$ is true if and only if $p\wedge (q\wedge r)$ is true. Such a property is called \textbf{associativity}
  and implies that we do not need to specify the order of calculation. Therefore we can write just $p\wedge q\wedge r$ without writing brackets.
\end{exercise}

\begin{definition}
  % Disjunction - definition
  Consider sentences $p$ and $q$. We say that their \textbf{disjunction} $p\vee q$ is true if and only if at least one of them is true. Usually disjunction of $p$ and $q$ is
  referred as "$p$ or $q$".
\end{definition}

\begin{example}
  Sentences "(2+1=3) or (2+1=4)" and "(2+1=3) or (3-1=2)" are both true while "(2+1=4) or (1+1=1)" is false.
\end{example}

\begin{exercise}
  % Disjunction - associativity and commutativity
  Prove that disjunction is both associative and commutative.
\end{exercise}

\begin{definition}
  % Negation - definition
  \textbf{Negation} of $p$ is a sentence $\neg p$ such that $\neg p$ is true if and only if $p$ is false. Usually we refer to $\neg p$ as "not $p$".
\end{definition}

\begin{exercise}
  % Negation - alternative definition as an exercise
  Prove that if $\neg p$ is false if and only if $p$ is true.
\end{exercise}

% Proof strategy - truth table
Now we will think about proof strategies. Sometimes there is an elegant way how to prove that two statements are equivalent (like in the proof of associativity of
conjunction, one can see that both sentences are true iff all three basic sentences are true), but in case of more complicated sentences, it may be hard to find it. A common
proof strategy is a \textbf{truth table} approach: we list in a table all the values that each basis sentence can take and evaluate the value of final expression.
Then \textit{two sentences are equivalent iff they have the same truth tables}.

\begin{example}
  % Truth table - conjunction
  Truth table for conjunction:\\
  \begin{center}
    \begin{tabular}{ c  c  c }
      $p$ & $q$ & $p\wedge q$ \\
      \hline
      t  &  t &        t     \\
      t  &  f &        f     \\
      f  &  t &        f     \\
      f  &  f &        f     \\
    \end{tabular}
  \end{center}
  where $t$ stands for "true" and $f$ stands for "false".
\end{example}

This is a very powerful approach, as it requires no clever tricks but a simple calculation. The only problem is the number of calculations, that grows very quickly with
the number of basic sentences!

\begin{exercise}
  % Truth table - 2^n rows.
  Assume that you have built a sentence using $n$ sentences: $p_1, p_2, \dots, p_n$. How many rows does the truth table contain?
\end{exercise}

\begin{exercise}
  % Conjunction and disjunction - distributivity
  Prove \textbf{distributivity}:
  \begin{enumerate}
    \item $(p\wedge q)\vee r \Leftrightarrow (p\vee r) \wedge (q\vee r)$
    \item $(p\vee q)\wedge r \Leftrightarrow (p\wedge r) \vee (q\wedge r)$
  \end{enumerate}
\end{exercise}

\begin{exercise}
  % De Morgan's laws
  Prove \textbf{De Morgan's laws}:
  \begin{enumerate}
    \item $\neg (p\wedge q) = (\neg p)\vee (\neg q)$
    \item $\neg (p\vee q) = (\neg p)\wedge (\neg q)$
  \end{enumerate}
\end{exercise}

\begin{definition}
  We say that $p$ \textbf{implies} $q$ (or that $q$ \textbf{is implied by} $p$) for a sentence $p\Rightarrow q$ that is false iff $p$ is false and $q$ is true.
  We can summarise it in a truth table:
  \begin{center}
    \begin{tabular}{ c  c  c }
      $p$ & $q$ & $p\Rightarrow q$ \\
      \hline
      t  &  t &        t     \\
      t  &  f &        f     \\
      f  &  t &        t     \\
      f  &  f &        t     \\
    \end{tabular}
  \end{center}
  As you can see, it's a strange behaviour - false implies everything!
\end{definition}

\begin{exercise}
  % Implication - expression in terms of negation and disjunction
  Prove that $(p\Rightarrow q)\Leftrightarrow (\neg p) \vee q$. Hint: left sentence is false for very specific $p$ and $q$. Do you need to write down all four rows in the truth table of the right-hand-side sentence?
\end{exercise}

\begin{exercise}
  % Implication - transitivity
  Prove that implication is \textbf{transitive}, that is $$((p\Rightarrow q)\wedge (q\Rightarrow r)) \Rightarrow (p\Rightarrow r).$$
\end{exercise}

\begin{exercise}
  Assuming that every topological space is homeomorphic to itself and that homeomorphic spaces are homotopic, prove that every topological is homotopic to itself. Hint: you don't need to know what the terms here mean to solve this exercise (but eventually will reach them!).
\end{exercise}

% The following exercise will later be a foundation of an operation called orthocomplementation in more general settings:
% \begin{exercise}
%   % Negation as orthocomplementation - properties
%   Let $p$ and $q$ be sentences. Prove that:
%   \begin{enumerate}
%     \item $\neg(\neg p) \Leftrightarrow p$
%     \item $p\Rightarrow q$ implies $(\neg q)\Rightarrow (\neg p)$ (Be smart! How many values of $p, q$ do you need to check?)
%     \item $p\vee (\neg p)$
%     \item $p\wedge (\neg p)$ is \textit{false} (we could write "Prove $\neg(p\wedge (\neg p))$", but it looks much more terrible!)
%   \end{enumerate}
% \end{exercise}

You may have discovered a similarity between symbols "$\Leftrightarrow$" and "$\Rightarrow$" - it's not an accident as you can prove!
\begin{exercise}
  % Equivalence via implications
  Prove that $(p\Leftrightarrow q)\Leftrightarrow ((p\Rightarrow q) \wedge (q\Rightarrow p))$.
\end{exercise}

% \subsection{Another point of view}
% In mathematics we have usually many different views on the same thing. Some of them are suited better for some kind of problems, other to others.
% We would like to introduce you to a useful model of propositional calculus. To each true sentence $p$ we assign number $v(p)$
% that is 1 if $p$ is true and 0 if $p$ is false. We define $1+1=1$
% (it's a bit unusual thing). Then:
% \begin{enumerate}
%   \item $a\Leftrightarrow b$ means the same thing as sentence $v(a)=v(b)$.
%   \item $a\wedge b$ means exactly the same thing as $v(a)\cdot v(b)$
%   \item $a\vee b$ means exactly the same thing as $v(a)+v(b)$ (this is the reason why we want $1+1=1$)
%   \item $a\Rightarrow b$ is the same as $v(a)\le (b)$.
%   \item $\neg 1 = 0$ and $\neg 0=1$
% \end{enumerate}
%
% \begin{exercise}
%   % Transitivity
%   Prove transitivity of implication, that is $((p\Rightarrow q)\wedge (q\Rightarrow r)) \Rightarrow (p\Rightarrow r)$ using transitivity of $\le$.
%   It simplifies the proof a bit, doesn't it?
% \end{exercise}

\subsection{Quantifiers}
Consider a sentence $P(n)$ involving an object $n$ (for example $n$ can be an integer and $P(n)$ can be a sentence "$n=2n$").
\begin{definition}
  % Universal and existence quantifiers - definition
  We define the \textbf{universal quantifier}
  as a sentence $\forall_n P(n)$ meaning "for all $n$, the formula $P(n)$ holds".
  We define the \textbf{existential quantifier} as a sentence $\exists_n P(n)$ meaning "there exists $n$ such that $P(n)$ holds"
  \footnote{$\forall$ is a rotated "A" symbolising "for \textbf{A}ll" and $\exists$ is a rotated "E" symbolising "\textbf{E}xists"}.
\end{definition}

\begin{example}
  % Example showing the difference between quantifiers.
  In the case of $P(n)$ meaning "$2n=n$", the sentence $\forall_n P(n)$ is false (as for $n=1$ we have $2\cdot 1\neq 1$) but the sentence $\exists_n P(n)$ is true,
  as $2\cdot 0=0$.
\end{example}

Intuitively, it is a much simpler problem to give an example of an object with a special property, than proving that \emph{every} object has a property.
In the above example, we gave an example disproving the statement. It may be useful to convert between these quantifiers. As you can prove:

\begin{exercise}
  % Quantifiers - negation
  Prove that:
  \begin{enumerate}
    \item $\neg \forall_n P(n) \Leftrightarrow \exists_n \neg P(n)$
    \item $\neg \exists_n P(n) \Leftrightarrow \forall_n \neg P(n)$
  \end{enumerate}
  What do the above state in English?
\end{exercise}
