%%%%%%%%%%%%%%%%%%%%% chapter.tex %%%%%%%%%%%%%%%%%%%%%%%%%%%%%%%%%
%
% sample chapter
%
% Use this file as a template for your own input.
%
%%%%%%%%%%%%%%%%%%%%%%%% Springer-Verlag %%%%%%%%%%%%%%%%%%%%%%%%%%

\chapter{Introduction}
\label{intro} % Always give a unique label
% use \chaptermark{}

There are many excellent books on mathematical physics and differential geometry, so a question raises - how does
this book differ from any other? I had a few aims working on it:
\begin{itemize}
	\item target audience is just a normal person that wants to understand advanced mathematics. It does not matter
	if you are a physicists, mathematician, english literature major or a high-school student. If you have
	enough self-determination, you can understand the mathematics in this book
	\item this book should be self-containing. Mathematics is both broad and deep, so it must be split into
	different  branches. But I found it discouraging that if you want to read one book, as prerequsities you need
	to read two other books, and so on. Here, you can understand everything without any access to libraries or
	other mathematical books. Obviously, we need to ommitt some Mathematics.
	\item  we define the most fundamental concepts and then we show how they work together in a more specific
	setting. Many great lecturers show how an abstract concept works in a specific case, so they provide lots of
	examples. I would like to do an experiment - show abstract concepts, give huge amount of exercises
	\item I personally enjoy problem solving approach. Therefore I just do not prove theorems - I want you
	to prove them, with adjustable amount of hints.
	
\end{itemize}

%
