% !TEX root = book.tex
\chapter{General topology}
\label{general_topology}
We have studied sets and functions between them. Now, we will add
a structure - called topology - to a set and look for special
functions that will in some sense ,,preserve the structure". This
subject is often called general topology or point-set topology as
we will focus on points and special sets called open and closed sets.
Eventually you will see, that these new concepts will enable us
to derive very quickly and elegantly the majority of the results
taught in undergraduate real analysis courses.

\section{Basic definitions}
\subsection{Topology and open sets}
Consider a set $X$. A topology is a set $\mathcal{T}_X \subseteq 2^X $ such that:
\begin{enumerate}
	\item $\varnothing, X\in \mathcal{T}_X$
	\item if $A, B\in \mathcal{T}_X$, then $A\cap B\in \mathcal{T}_X$
	\item if $A_i\in \mathcal{T}_X$ for $i\in I$, then $\bigcup_{i\in I} A_i\in \mathcal T_X$
\end{enumerate}
Members of topology we call \textbf{open sets}.

\begin{prob}
	Using mathematical induction prove that the intersection of finitely many open sets is open.
\end{prob}

You may wonder whether, for a given set, topology is unique. As you can prove, there can be many topologies.

\begin{prob}
	\textbf{Trivial topology} Prove that for any $X$, set $\{\varnothing, X\}$ is a topology.
\end{prob}

\begin{prob}
	\textbf{Discrete topology} Prove that for any $X$, it's power set $2^X$ is a topology.
\end{prob}

\begin{prob}
	\textbf{Cofinite topology} Prove that for any $X$, the set:
  $\mathcal T_X=\{\emptyset\}\cup \{A\subseteq X : X\setminus A \text{ is finite}\}$ is a topology. Hint: think in terms of complements.
\end{prob}

\begin{prob}
	For which sets, there is exactly one topology on them? (So at least
  these listed above must be the same).
\end{prob}

\begin{prob}
	Prove that for an infinite set, there are at least three distinct
  topologies.
\end{prob}

\noindent This gives the necessity of the notation of \textbf{topological space} that is a pair $(X,\mathcal T_X)$, where $\mathcal T_X$ is a topology on
$X$. There are spaces, for which just one topology is commonly used in applications. In such situations mathematicians write topological space just as
$X$, assuming that the ,,preferred" topology is obvious to the reader.
Consider a topological space $(X,\mathcal T_X)$ and a point $x\in X$.
If $x\in U\in \mathcal T_X$, we say that $U$ is an open neighborhood
of $x$. If $x\in U\subseteq V$, where $U$ is open, we call $V$ a
neighborhood of $x$.

\begin{prob}
  Prove that each point has an open neighborhood.
\end{prob}

\begin{prob}
  Prove that $A$ is an open set if and only if each point $a$ has
  a neighborhood $U_a\in A$ contained in $A$
  (that is $U_a\subseteq A$).
\end{prob}

\noindent For a set $A$ in a topological space, we define \textbf{the
interior of $A$} as:
$$\Int A=\bigcup \mathcal X, \text{ where }.$$

\begin{prob}
  Prove that:
  \begin{enumerate}
    \item $\Int A$ is an open set.
    \item if $A'\subseteq A$ is open, then $A'\subseteq \Int A$ (
    so in some sense, $\Int A$ is the biggest open set contained in
    $A$)
    \item $\Int A = A$ iff $A$ is open
    \item $\Int \Int A = \Int A$ for any $A$
  \end{enumerate}
\end{prob}

\begin{prob}
  Let $A'\subseteq A$. Prove that:
  \begin{enumerate}
    \item $\Int A' \subseteq \Int A$
    \item $\Int A \cup \Int B\subseteq \Int (A\cup B)$
  \end{enumerate}
  You can prove also that the union can be arbitrary.
\end{prob}

\begin{prob}
	We say that $a$ is an \textbf{interior point} of $A$ if there is open $U_a\subseteq A$ such that $a\in U_a$. Prove that $\Int A$ is the set
	of all interior points of $A$.
\end{prob}

\subsection{Closed sets}
Consider a topological space $(X, \mathcal T_X)$. We say that $A\subseteq X$ is closed if and only if $X\setminus A$ is open.

\begin{prob}
  Prove these properties of closed sets in space $(X, \mathcal T_X)$:
  \begin{enumerate}
    \item $\varnothing$ and  $X$ are closed
    \item If $A_1, A_2,\dots, A_n$ are closed, then their
      union $A_1\cup A_2\cup\dots\cup A_n$ is closed.
    \item If $\mathcal A$ is any family of
      closed sets, then the intersection $\bigcap \mathcal A$ is
      closed.
  \end{enumerate}
\end{prob}

\begin{prob}
  We say that $p$ is a limit point of $A\subseteq X$ if for every every open neighborhood $U$ of $p$ there is $q_U\neq p$ such that
  $q_U\in A\cap U$. Prove that $A$ is closed iff it contains all
  of it's limit points.
\end{prob}

\noindent We define \textbf{the closure} of a set $A$ as:
$$\Cl A = \bigcap \mathcal X,$$
where $\mathcal X = \{X\subseteq A : X \text{ is closed} \}$.

\begin{prob}
  Prove that:
  \begin{enumerate}
    \item $\Cl A$ is a closed set.
    \item if $C$ is closed and $A\subseteq C$,
    then $\Cl A\subseteq C$ (so in some sense,
    $\Cl A$ is the smallest closed set containing $A$)
    \item $\Cl A = A$ iff $A$ is closed
    \item $\Cl \Cl A = \Cl A$ for any $A$
  \end{enumerate}
\end{prob}

\begin{prob}
	We say that $p$ is an \textbf{adherent point} of $A$
  (or \textbf{point of closure})
  if for any neighborhood $V$ of $p$ we have
  $A\cap V\neq \emptyset$. Alternatively, we can say that every
  neighborhood of $p$ contains
  a point from $A$. Prove that $\Cl A$ is the set
	of all adherent points of $A$.
\end{prob}

\begin{prob}
  We say that $A\subseteq X$ is \textbf{dense} if $\Cl A=X$. Prove
  that $A$ is dense iff for every $U\in \mathcal T_X$, $A\cap U\neq \emptyset$
\end{prob}

\noindent We define the \textbf{boundary of $A$} as:
$$\partial A=\Fr A = \Cl A \setminus \Int A$$

\begin{prob}
  We say that $p$ is a \textbf{frontier} point of $A$ if every open
  neighborhood of $p$ intersects both $A$ and $A^c$, so if
  for every open neighborhood $U_p$ we have
  $U_p\cap A\neq \emptyset$ and $U_p\cap A^c\neq \emptyset$.
  Prove that the boundary of $A$ is exactly the set of frontier
  points of $A$.
\end{prob}

\begin{prob}
  Prove that boundary is always closed.
\end{prob}

\begin{prob}
  Prove that $\partial \partial A \subseteq \partial A$.
\end{prob}

\begin{prob}
  Prove that $\partial A = \partial A^c.$
\end{prob}

\begin{prob}
  Prove that $\partial A = \emptyset$ iff $A$ is simultaneously
  open and closed.
\end{prob}

We define the \textbf{exterior of $A$} as
$$\Ext A = X\setminus \Cl A$$

\begin{prob}
  Prove that $\partial A=\Cl A \cap \Cl \Ext A$
\end{prob}
