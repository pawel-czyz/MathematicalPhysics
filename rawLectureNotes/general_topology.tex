% !TEX root = book.tex
\chapter{General topology}
\label{general_topology}
We have studied sets and functions between them. Now, we will add
a structure - called topology - to a set and look for special
functions that will in some sense ,,preserve the structure". This
subject is often called general topology or point-set topology as
we will focus on points and special sets called open and closed sets.
Eventually you will see, that these new concepts will enable us
to derive very quickly and elegantly the majority of the results
taught in undergraduate real analysis courses.

\section{Basic definitions}
\subsection{Topology, open sets and interior}
Consider a set $X$. A topology is a set $\mathcal{T}_X \subseteq 2^X $ such that:
\begin{enumerate}
	\item $\varnothing, X\in \mathcal{T}_X$
	\item if $A, B\in \mathcal{T}_X$, then $A\cap B\in \mathcal{T}_X$
	\item if $A_i\in \mathcal{T}_X$ for $i\in I$, then $\bigcup_{i\in I} A_i\in \mathcal T_X$
\end{enumerate}
Members of topology we call \textbf{open sets}.

\begin{prob}
	Using mathematical induction prove that the intersection of finitely many open sets is open.
\end{prob}

You may wonder whether, for a given set, topology is unique. As you can prove, there can be many topologies.

\begin{prob}
	\textbf{Trivial topology} Prove that for any $X$, set $\{\varnothing, X\}$ is a topology.
\end{prob}

\begin{prob}
	\textbf{Discrete topology} Prove that for any $X$, it's power set $2^X$ is a topology.
\end{prob}

\begin{prob}
	\textbf{Cofinite topology} Prove that for any $X$, the set:
  $\mathcal T_X=\{\emptyset\}\cup \{A\subseteq X : X\setminus A \text{ is finite}\}$ is a topology. Hint: think in terms of complements.
\end{prob}

\begin{prob}
  \textbf{Subspace topology}
  Let $X$ be a set and $\mathcal T_X$ a topology on it. For $A\subseteq X$ we define $\mathcal T_A=\{U\cap A : U\in \mathcal T_X\}$. Prove that $\mathcal T_A$ is a topology on $A$.
\end{prob}

\begin{prob}
	For which sets, there is exactly one topology on them?
\end{prob}

\begin{prob}
	Prove that for an infinite set, there are at least three distinct
  topologies.
\end{prob}

\begin{prob}
  Real numbers $\mathbb R$ are usually equipped with the following topology:
  $X\subseteq \mathbb R$ is open iff for each $x\in X$ there is an open interval $(a_x,b_x)$ such that
  $x\in (a_x, b_x)\subseteq X$.
  \begin{enumerate}
    \item Prove that it is indeed a topology.
    \item Let \textbf{a ball} be a set $B(x,r) = \{y\in \mathbb R : |x-y| < r\}$ for $r>0$. Prove that $(a,b)\neq \emptyset$ can be written as $B(x,r)$ for suitable $x$ and $r$
    \item Prove that $X$ is open iff for each $x\in X$ there is a $r>0$ such that $B(x,r)\subseteq X$. We will see that this results generalises to much broader category of spaces
      than single $\mathbb R$.
  \end{enumerate}
\end{prob}

\noindent This gives the necessity of the notation of \textbf{topological space} that is a pair $(X,\mathcal T_X)$, where $\mathcal T_X$ is a topology on
$X$. There are spaces, for which just one topology is commonly used in applications. In such situations mathematicians write topological space just as
$X$, assuming that the ,,preferred" topology is obvious to the reader (as the topology given above on $\mathbb R$).

This was ,,global" point of view - we have a structure of subsets of $X$. We can also try to express these global properties using local
properties - by considering special constructions around a single point and using a set of these points to recover a global property.
Consider a topological space $(X,\mathcal T_X)$ and a point $x\in X$.
If $x\in U\in \mathcal T_X$, we say that $U$ is an open neighborhood
of $x$. If $x\in U\subseteq V$, where $U$ is open, we call $V$ a
neighborhood of $x$.

\begin{prob}
  Prove that each point has an open neighborhood.
\end{prob}

\begin{prob}
  Prove that $A$ is an open set if and only if each point $a$ has
  a neighborhood $U_a\in A$ contained in $A$
  (that is $U_a\subseteq A$).
\end{prob}

\noindent For a set $A$ in a topological space, we define \textbf{the
interior of $A$} as:
$$\Int A=\bigcup \mathcal U, \text{ where } \mathcal U = \{U\in \mathcal T : U\subseteq A \}.$$

\begin{prob}
  Prove that:
  \begin{enumerate}
    \item $\Int A$ is an open set.
    \item if $A'\subseteq A$ is open, then $A'\subseteq \Int A$ (
    so in some sense, $\Int A$ is the biggest open set contained in
    $A$)
    \item $\Int A = A$ iff $A$ is open
    \item $\Int \Int A = \Int A$ for any $A$
  \end{enumerate}
\end{prob}

\begin{prob}
  Let $A'\subseteq A$. Prove that:
  \begin{enumerate}
    \item $\Int A' \subseteq \Int A$
    \item $\Int A \cup \Int B\subseteq \Int (A\cup B)$
  \end{enumerate}
  You can prove also that the union can be arbitrary.
\end{prob}

\begin{prob}
	We say that $a$ is an \textbf{interior point} of $A$ if there is open $U_a\subseteq A$ such that $a\in U_a$. Prove that $\Int A$ is the set
	of all interior points of $A$.
\end{prob}

\subsection{Closed sets and closure}
Consider a topological space $(X, \mathcal T_X)$. We say that $A\subseteq X$ is closed if and only if $X\setminus A$ is open.

\begin{prob}
  Prove these properties of closed sets in space $(X, \mathcal T_X)$:
  \begin{enumerate}
    \item $\varnothing$ and  $X$ are closed
    \item If $A_1, A_2,\dots, A_n$ are closed, then their
      union $A_1\cup A_2\cup\dots\cup A_n$ is closed.
    \item If $\mathcal A$ is any family of
      closed sets, then the intersection $\bigcap \mathcal A$ is
      closed.
  \end{enumerate}
\end{prob}

\begin{prob}
  We say that $p$ is a limit point of $A\subseteq X$ if for every every open neighborhood $U$ of $p$ there is $q_U\neq p$ such that
  $q_U\in A\cap U$. Prove that $A$ is closed iff it contains all
  of it's limit points.
\end{prob}

\noindent We define \textbf{the closure} of a set $A$ as:
$$\Cl A = \bigcap \mathcal X,$$
where $\mathcal X = \{X\subseteq A : X \text{ is closed} \}$.

\begin{prob}
  Prove that:
  \begin{enumerate}
    \item $\Cl A$ is a closed set.
    \item if $C$ is closed and $A\subseteq C$,
    then $\Cl A\subseteq C$ (so in some sense,
    $\Cl A$ is the smallest closed set containing $A$)
    \item $A\subseteq \Cl A$
    \item $\Cl (A\cup B) = \Cl A \cup \Cl B$
    \item $\Cl A = A$ iff $A$ is closed
    \item $\Cl \Cl A = \Cl A$ for any $A$
  \end{enumerate}
\end{prob}

\begin{prob}
	We say that $p$ is an \textbf{adherent point} of $A$
  (or \textbf{point of closure})
  if for any neighborhood $V$ of $p$ we have
  $A\cap V\neq \emptyset$. Alternatively, we can say that every
  neighborhood of $p$ contains
  a point from $A$. Prove that $\Cl A$ is the set
	of all adherent points of $A$.
\end{prob}

\begin{prob}
  We say that $A\subseteq X$ is \textbf{dense} if $\Cl A=X$. Prove
  that $A$ is dense iff for every $U\in \mathcal T_X$, $A\cap U\neq \emptyset$
\end{prob}

\begin{prob}
  \begin{enumerate}
    \item Let $r\in \mathbb R$. Prove that for every neighborhood $V$ of $r$ there is $q\in \mathbb Q$ such that $q\in V$.
      Hint: each neighborhood must have an interval. And you should have proven that in each interval there is a rational.
    \item Conclude that rationals are dense in reals.
  \end{enumerate}
\end{prob}

\subsection{Boundary and exterior}

\noindent We define the \textbf{boundary of $A$} as:
$$\partial A=\Fr A = \Cl A \setminus \Int A$$

\begin{prob}
  We say that $p$ is a \textbf{frontier} point of $A$ if every open
  neighborhood of $p$ intersects both $A$ and $A^c$, so if
  for every open neighborhood $U_p$ we have
  $U_p\cap A\neq \emptyset$ and $U_p\cap A^c\neq \emptyset$.
  Prove that the boundary of $A$ is exactly the set of frontier
  points of $A$.
\end{prob}

\begin{prob}
  Prove that boundary is always closed.
\end{prob}

\begin{prob}
  Prove that $\partial \partial A \subseteq \partial A$.
\end{prob}

\begin{prob}
  Prove that $\partial A = \partial A^c.$
\end{prob}

\begin{prob}
  Prove that $\partial A = \emptyset$ iff $A$ is simultaneously
  open and closed.
\end{prob}

We define the \textbf{exterior of $A$} as
$$\Ext A = X\setminus \Cl A$$

\begin{prob}
  Prove that $\partial A=\Cl A \cap \Cl \Ext A$
\end{prob}

\subsection{Bases and countability axioms}
As we have seen, there can be many open sets. Let's try to simplify the situation by considering a smaller
family of open sets from which we will be able to recover the whole topology.

Let $(X, \mathcal T)$ be a topological space. We say that a family of sets $\mathcal B\subseteq \mathcal T$
is a \textbf{basis of topology} iff every open set can be written as a sum of a subfamily of $\mathcal B$.
Namely for each $U\in \mathcal T$ there is $\mathcal B_U\subseteq \mathcal B$ such that:
$$U=\bigcup \mathcal B_U$$

\begin{prob}
  Prove that $\mathcal B$ is a basis for $(X, \mathcal T)$ iff
  for every $x\in X$ and every neighborhood $U_i$ of $x$, there
  is $B_i\in \mathcal B$ such that $x\in B_i\subseteq U_i$.
\end{prob}

\begin{prob}
  Let $\mathcal B$ be a basis of $(X, \mathcal T)$. Prove that:
  \begin{enumerate}
    \item $\bigcup \mathcal B = X$
    \item If $U, V\in \mathcal T$ and $x\in U\cap V$, then there is a
      set $B_x\in \mathcal B$ such that $x\in B_x\subseteq U\cap V.$
  \end{enumerate}
\end{prob}

% Proving the converse, we get another, equivalent definition of basis:
% TODO TODO

We say that a space $(X,\mathcal T)$ is \textbf{second countable} iff
it has a countable basis.

\begin{prob}
  Consider $\mathbb R$ with it's standard topology. If $x\in \mathbb R$ and $U$ is an open set containing $x$, we can find a ball $B(x, r),~r>0$ such that $B(x,r)\subseteq U$. Using the fact that
  rationals are dense in reals, prove that you can
  find $p,q\in \mathbb Q$ such that $x\in (p,q)\subseteq U$.
\end{prob}

\begin{prob}
  Prove that $\mathbb R$ is second countable.
\end{prob}

\section{Continuous maps and homeomorphisms}
Consider two topological spaces $(X,\mathcal T)$ and $(Y, \tau)$.
We say that function (or \textbf{map}) $f:X\to Y$ is \textbf{continuous} iff for every
open set $U\in \tau$, it's preimage is open: $f^{-1}(U)\in \mathcal T$.

\begin{prob}
  Prove that a function is continuous iff preimage of every \textit{closed} set is closed.
\end{prob}

The next problem shows that we can employ the concept of a basis to
shorten proofs:

\begin{prob}
  Let $f: (X,\mathcal T)\to (Y, \tau)$ and $\mathcal B$ be a basis of
  $(Y, \tau)$. Then $f$ is continuous iff $f^{-1}(B)\in \mathcal T$ for every $B\in \mathcal B$.
\end{prob}

\begin{prob}
  Assuming that $\mathbb R$ is equipped with it's standard topology,
  prove that functions from $\mathbb R$ to $\mathbb R$ are continuous:
  \begin{enumerate}
    \item $f(x)=ax+b$
    \item $f(x)=x^2$
  \end{enumerate}
\end{prob}

\begin{prob}
  Let $f: (X,\mathcal T)\to (Y, \tau)$. Prove that $f$ is continuous
  iff $f(\Cl A)\subseteq \Cl f(A)$ for every $A\subseteq X$.
\end{prob}

We say that a map $f: (X,\mathcal T)\to (Y, \tau)$ is a \textbf{homeomorphism} iff is bijective and both $f$ and $f^{-1}$ are continuous. We say that two topological spaces are \textbf{homeomorphic} iff there is a homeomorphism between them. The aim of topology is to classify all the spaces up to homeomorphisms. This is a hard problem, but we can try to classify a smaller class of spaces:

\begin{prob}
  Prove that two \textit{discrete} spaces $X$ and $Y$ are homeomorphic iff $|X|=|Y|.$
\end{prob}

\section{Connected spaces}
We say that a topological space $(X, \mathcal T)$ is \textbf{disconnected} if there exists two disjoint, non-empty sets such their union is the whole space $X$. Or using symbols:
$(X, \mathcal T)$ is disconnected if $U,V\in \mathcal T$ such that $U,V\neq \emptyset,~U\cap V =\emptyset,~U\cup V=X$.
\begin{prob}
  Let $(X, \mathcal T)$ be a topological space. Prove that these conditions are equivalent:
  \begin{enumerate}
    \item The space is disconnected.
    \item There are two \textit{open} sets $A,B\subseteq X$ such that $A,B\neq \emptyset,~A\cap B=\emptyset,~A\cup B=X$.
    \item There are no two \textit{closed} sets $A,B\subseteq X$ such that $A,B\neq \emptyset,~A\cap B=\emptyset,~A\cup B=X$.
    \item There is a set $S\subset X,~S\neq \emptyset, X$ such that and $S$ is open and closed simultaneously (sometimes sets that are both open and closed are called
    \textbf{clopen}).
    \item There is a set $S\subset X,~S\neq \emptyset, X$ such that $\partial S=\emptyset$.
    \item There are subsets $A,B\subseteq X,~ A,B\neq \emptyset$ such that $A\cap \Cl B = B\cap \Cl A=\emptyset$ and $A\cup B=X$.
  \end{enumerate}
\end{prob}

If a space is not disconnected, it is called \textbf{connected}.

\begin{prob}
  Let $(X, \mathcal T)$ be a topological space. Prove that these conditions are equivalent:
  \begin{enumerate}
    \item The space is connected.
    \item There are no two \textit{open} sets $U,V\subseteq X$ such that $U,V\neq \emptyset,~U\cap V=\emptyset,~U\cup V=X$.
    \item There are no two \textit{closed} sets $U,V\subseteq X$ such that $U,V\neq \emptyset,~U\cap V=\emptyset,~U\cup V=X$.
    \item The only sets that are open and closed simultaneously are $\emptyset$ and $X$.
    \item All continous maps from $(X, \mathcal T)$ to $(\{0,1\}, \text{discrete topology})$ are constant.
    \item If $S\subseteq X$ and $\partial S=\emptyset$, then $S=\emptyset$ or $S=X$.
  \end{enumerate}
\end{prob}
