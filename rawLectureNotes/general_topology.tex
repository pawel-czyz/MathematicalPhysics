% !TEX root = book.tex

\newcommand{\Int}{\text{Int\,}}

\chapter{General topology}
\label{general_topology}
Now we are prepared enough to dive into general topology (often called point-set topology).

\section{Basics}
\subsection{Topology and open sets}
Consider a set $X$. A topology is a set $\mathcal{T}_X \subseteq 2^X $ such that:
\begin{enumerate}
	\item $\varnothing, X\in \mathcal{T}_X$
	\item if $A, B\in \mathcal{T}_X$, then $A\cap B\in \mathcal{T}_X$
	\item if $A_i\in \mathcal{T}_X$ for $i\in I$, then $\bigcup_{i\in I} A_i\in \mathcal T_X$
\end{enumerate}
Members of topology we call \textbf{open sets}.

\begin{prob}
	Using mathematical induction prove that the intersection of finitely many open sets is open.
\end{prob}

You may wonder whether, for a given set, topology is unique. As you can prove, there can be many topologies.

\begin{prob}
	\textbf{Trivial topology} Prove that for any $X$, set $\{\varnothing, X\}$ is a topology.
\end{prob}

\begin{prob}
	\textbf{Discrete topology} Prove that for any $X$, it's power set $2^X$ is a topology.
\end{prob}

\begin{prob}
	\textbf{Cofinite topology} Prove that for any $X$, the set:
  $\mathcal T_X=\{\emptyset\}\cup \{A\subseteq X : X\setminus A \text{ is finite}\}$ is a topology. Hint: think in terms of complements.
\end{prob}

\begin{prob}
	For which sets, there is exactly one topology on them? (So at least
  these listed above must be the same).
\end{prob}

\begin{prob}
	Prove that for an infinite set, there are at least three distinct
  topologies.
\end{prob}

\noindent This gives the necessity of the notation of \textbf{topological space} that is a pair $(X,\mathcal T_X)$, where $\mathcal T_X$ is a topology on
$X$. There are spaces, for which just one topology is commonly used in applications. In such situations mathematicians write topological space just as
$X$, assuming that the ,,preferred" topology is obvious to the reader.
Consider a topological space $(X,\mathcal T_X)$ and a point $x\in X$.
If $x\in U\in \mathcal T_X$, we say that $U$ is an open neighborhood
of $x$.

\begin{prob}
  Prove that each points has an open neighborhood.
\end{prob}

\begin{prob}
  Prove that $A$ is an open set if and only if each point $a$ has
  a neighborhood $U_a\in A$ contained in $A$
  (that is $U_a\subseteq A$).
\end{prob}

\noindent For a set $A$ in a topological space, we define \textbf{the
interior of $A$} as:
$$\Int A=\bigcup_{X \text{ is open and is a subset of A}} X.$$

\begin{prob}
  Prove that:
  \begin{enumerate}
    \item $\Int A$ is an open set.
    \item if $A'\subseteq A$ is open, then $A'\subseteq \Int A$ (
    so in some sense, $\Int A$ is the biggest open set contained in
    $A$)
    \item $\Int A = A$ iff $A$ is open
    \item $\Int \Int A = \Int A$ for any $A$
    \item $\Int A\cup \Int B \subseteq \Int (A\cup B)$
  \end{enumerate}
\end{prob}

\begin{prob}
	We say that $a$ is an \textbf{interior point} of $A$ if there is open $U_a\subseteq A$ such that $a\in U_a$. Prove that $\Int A$ is the set
	of all interior points of $A$.
\end{prob}

\subsection{Closed sets}
Consider a topological space $(X, \mathcal T_X)$. We say that $A\subseteq X$ is closed if and only if $X\setminus A$ is open.




%\subsection{Definition of intervals}
% \begin{align*}
% 	(a,b) &= \{x\in \mathbb R : a < x < b\} \text{ (open interval)}\\
% 	[a,b] &= \{x\in \mathbb R : a \le x \le b\} \text{ (closed interval)}\\
% 	(a,b] &= \{x\in \mathbb R : a < x \ge b\} \text{ (left-open, right-closed interval)}\\
% 	[a,b) &= \{x\in \mathbb R : a < x \ge b\} \text{ (left-closed, right-open interval)}
% \end{align*}
