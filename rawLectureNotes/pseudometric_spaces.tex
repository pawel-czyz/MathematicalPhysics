% !TEX root = book.tex

\chapter{Pseudometric spaces}
\label{pseudometric_spaces}
We have already seen how general topology works. Now we will focus on another collection of spaces, with richer structure, called pseudometric spaces.
We will follow Cain's approach (which is one of my favourite book on this subject).

\section{Pseudometric spaces}
Consider a set $X$. We say that a function $d : X\times X\to \mathbb R$ is a \textbf{pseudometric} if:
\begin{enumerate}
	\item for all $x\in X$,  $d(x,x)=0$
	\item for all $x,y\in X$, $d(x,y)=d(y,x)$
	\item for all $x,y,z\in X$, $d(x,z)\le d(x,y)+d(y,z)$
\end{enumerate}

\begin{prob}
	Prove that for a pseudometric $d$ and every $x,y\in X$, there is $d(x,y)\ge 0$.
\end{prob}

\begin{prob}
	Prove that $d(x,y)=0$ for any $x, y \in X$ is a pseudometric on $X$. This is called \textbf{trivial pseudometric}.
\end{prob}

\begin{prob}
	Prove that $d(x,y)=1$ for $x\neq y$ is a pseudometric on $X$. This is called \textbf{discrete pseudometric}.
\end{prob}

\begin{prob}
	Prove that $d(x,y)=|x-y|$ is a pseudometric on $\mathbb R$.
\end{prob}

As we can see above, pseudometric is not determined by the underlying set (as in the case with topology!). Therefore we introduce the concept of
pseudometric space $(X,d)$. As we said, these spaces have richer structure - each pseudometric space is a topological space, as you can prove in a minute.
Essential concept is the concept of a ball of radius $r>0$ centered at $x\in X$:
$$B(x, r) = \{y \in X : d(x,y) < r\}.$$

\begin{prob}
	We say that $S\subseteq X$ is an open set if for every $s$ in $S$ there is $r_s$ such that $B(s,r_s)\subseteq S$.
	Prove that this is indeed a topology on $X$.
\end{prob}

\begin{prob}
	Prove that
	\begin{enumerate}
		\item Topology obtained from trivial pseudometric is the trivial topology
		\item Topology obtained from discrete pseudometric is the discrete topology.
	\end{enumerate}
\end{prob}

As any pseudometric space is a topological space, we have many results and concepts that may work for them! In this chapter we will try to derive
stronger results (we have more assumptions, so we can obtain more results). As we remember basis of the topology can simplify many results. Let's
find it.

\begin{prob}
	Let $(X,d)$ be a pseudometric space. Prove that:
	\begin{enumerate}
		\item Any $B(x,r)$ is open
		\item $\{B(x,r) : x\in X \text{ and }r > 0\}$ is a basis
		\item This space is first-countable. Hint: consider $r=1/n$ for $n=1,2,\dots$.
	\end{enumerate}
\end{prob}

\section{Topology of $\mathbb R$}
Now, we will focus on the ,,natural" topology of the real line. As we remember
real numbers is a complete ordered field. 




%\subsection{Definition of intervals}
% \begin{align*}
% 	(a,b) &= \{x\in \mathbb R : a < x < b\} \text{ (open interval)}\\
% 	[a,b] &= \{x\in \mathbb R : a \le x \le b\} \text{ (closed interval)}\\
% 	(a,b] &= \{x\in \mathbb R : a < x \ge b\} \text{ (left-open, right-closed interval)}\\
% 	[a,b) &= \{x\in \mathbb R : a < x \ge b\} \text{ (left-closed, right-open interval)}
% \end{align*}
