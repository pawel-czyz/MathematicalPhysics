% !TEX root = book.tex

\chapter{Pseudometric spaces}
\label{pseudometric_spaces}
We have already seen how general topology works. Now we will focus on another collection of spaces, with richer structure, called pseudometric spaces.
We will follow Cain's approach (which is one of my favourite book on this subject).

\section{Pseudometric spaces}
Consider a set $X$. We say that a function $d : X\times X\to \mathbb R$ is a \textbf{pseudometric} if:
\begin{enumerate}
	\item for all $x\in X$,  $d(x,x)=0$
	\item for all $x,y\in X$, $d(x,y)=d(y,x)$
	\item for all $x,y,z\in X$, $d(x,z)\le d(x,y)+d(y,z)$
\end{enumerate}

\begin{prob}
	Prove that for a pseudometric $d$ and every $x,y\in X$, there is $d(x,y)\ge 0$.
\end{prob}

\begin{prob}
	Prove that $d(x,y)=0$ for any $x, y \in X$ is a pseudometric on $X$. This is called \textbf{trivial pseudometric}.
\end{prob}

\begin{prob}
	Prove that $d(x,y)=1$ for $x\neq y$ is a pseudometric on $X$. This is called \textbf{discrete pseudometric}.
\end{prob}

\begin{prob}
	Prove that $d(x,y)=|x-y|$ is a pseudometric on $\mathbb R$.
\end{prob}

As we can see above, pseudometric is not determined by the underlying set (as in the case with topology!). Therefore we introduce the concept of
pseudometric space $(X,d)$. As we said, these spaces have richer structure - each pseudometric space is a topological space, as you can prove in a minute.
Essential concept is the concept of a ball of radius $r>0$ centered at $x\in X$:
$$B(x, r) = \{y \in X : d(x,y) < r\}.$$

\begin{prob}
	We say that $S\subseteq X$ is an open set if for every $s$ in $S$ there is $r_s$ such that $B(s,r_s)\subseteq S$.
	Prove that this is indeed a topology on $X$.
\end{prob}

\begin{prob}
	Prove that
	\begin{enumerate}
		\item Topology obtained from trivial pseudometric is the trivial topology
		\item Topology obtained from discrete pseudometric is the discrete topology.
	\end{enumerate}
\end{prob}

As any pseudometric space is a topological space, we have many results and concepts that may work for them! In this chapter we will try to derive
stronger results (we have more assumptions, so we can obtain more results). As we remember basis of the topology can simplify many results. Let's
find it.

\begin{prob}
	Let $(X,d)$ be a pseudometric space. Prove that:
	\begin{enumerate}
		\item Any $B(x,r)$ is open
		\item $\{B(x,r) : x\in X \text{ and }r > 0\}$ is a basis
		\item This space is first-countable. Hint: consider $r=1/n$ for $n=1,2,\dots$.
	\end{enumerate}
\end{prob}

\section{Topology of $\mathbb R$}
Now, we will focus on the ,,natural" topology of the real line.
Before we start, we need a property that is crucial in the definition
of reals. Consider $A\subseteq \mathbb R$. We say that $x$ is an
\textbf{upper bound} of $A$ iff $x\ge a$ for every $a\in A$.

\begin{prob}
  Prove that a set $A\subseteq \mathbb R$ can have no upper bounds
  or infinitely many of them.
\end{prob}

\noindent If an upper bound of $A$ exists, we say that $A$ is
\textbf{bounded from above}. Among them we will distinguish the
\textbf{supremum} (or \textbf{the least upper bound - l.u.b}):
$x=\sup A$ iff $x$ is an upper bound of $A$ and for any upper bound
$y$ of $A$ we have $x\le y$.

\begin{prob}
  Prove that supremum is unique, so if $x$ and $x'$ are supremums
  of $A$, then $x=x'$.
\end{prob}

\begin{prob}
  Prove that $x=\sup A$ if and only if
  $x\ge a$ for every $a\in A$ and for every $\eps > 0$ there is
  $a\in A$ such that $x < a + \eps$.
\end{prob}

\noindent Now we can state the \textbf{completeness axiom}:
each non-empty and bounded from above subset of real numbers has
a supremum.
This axiom allows us to prove many interesting things:

\begin{prob}
  Prove that natural numbers are \textit{not} bounded from above.
  Hint: if $n\in \mathbb N$, then $n+1\in \mathbb N$
\end{prob}

\begin{prob}
  Prove the \textbf{Archimedean axiom}\footnote{In fact we do not
  need to call it axiom, as we are able to prove it.}
  that for every $r\in R$, there is $n\in \mathbb N$ such that $n>r$.
\end{prob}

\begin{prob}
  Prove that for any $r>0$ there is $n\in \mathbb N$ such that
  $1/n < r$.
\end{prob}

\begin{prob}
  Prove that rational numbers do \textit{not} have the completeness
  property:
  \begin{enumerate}
    \item Let $p, q\in \mathbb Z\setminus \{0\}$.
      Prove that $p^2\neq 2q^2$.
    \item Prove that root of two, defined as
      $x > 0, x^2=2$ is not rational.
    \item Find a subset of $\mathbb Q$ that is bounded above, but
      has no rational supremum.
  \end{enumerate}
\end{prob}

\begin{prob}
  You should prove that in each non-emptyinterval there is at least
  one rational number:
  \begin{enumerate}
    \item Assume that $0<a<b$. Define
      $$A=\left\{\frac m N : m\in \mathbb N\right\},~
      \frac 1{b-a} < N \in \mathbb N$$
      and prove that $A\cap (a,b)$ is non-empty.
    \item Use the above result to prove that in \textit{each}
      interval there is at least one rational number.
    \item Prove that in each interval there are infinitely,
    countably many, rational numbers
    \item Prove that in each interval there is an irrational number.
    \item How many irrational numbers are in each interval?
  \end{enumerate}
\end{prob}

% At the beginning we assumed that you are familiar with real numbers.
% It's high time we defined them properly.
% A \textbf{field} is a tuple $(F, +, \cdot, 1, 0).$ We have
% many symbols there, let's explain what they mean:
% \begin{itemize}
%   \item $F$ is a set
%   \item $+$ and $\cdot$ are functions from $F^2$ to $F$. We write
%     $a+b$ for $+(a,b)$ and $a\cdot b=ab$ for $\cdot (a,b)$.
%   \item $1, 0\in F$ are just distinguished elements of $F$
% \end{itemize}
% We know what object are in the definition, so we can talk about
% properties they should have:
%
% \begin{enumerate}
%   \item
% \end{enumerate}


%\subsection{Definition of intervals}
% \begin{align*}
% 	(a,b) &= \{x\in \mathbb R : a < x < b\} \text{ (open interval)}\\
% 	[a,b] &= \{x\in \mathbb R : a \le x \le b\} \text{ (closed interval)}\\
% 	(a,b] &= \{x\in \mathbb R : a < x \ge b\} \text{ (left-open, right-closed interval)}\\
% 	[a,b) &= \{x\in \mathbb R : a < x \ge b\} \text{ (left-closed, right-open interval)}
% \end{align*}
