\chapter{Taste of category theory}

\begin{quote}
  Category theory [...] is the "mathematics of mathematics".
  Robert Geroch
\end{quote}

As you have had training in sets and functions, we are able to introduce some category theory that will quickly become useful - most of the objects in mathematics form a
category and this language will be extremely convenient to find similarities between different branches of mathematics.
As we remember, it is not possible to create a set of all sets without having a contradiction. So let's use a word \textbf{collection} of sets or \textbf{class}
\footnote{Formal treatment of classes - collections that are in some sense bigger than sets - is introduced in von Neumann–Bernays–Gödel and Morse-Kelley set theories.}
 of sets - that is not a set and we don't know how to express it formally - but what has an intuitive sense.

\begin{definition}
  A \textbf{category} is:
  \begin{enumerate}
    \item a collection of objects such that
    \item for each pair objects $A$, $B$ in the collection there is a \emph{set}\footnote{Many authors don't assume that it is a set. Our definition is their "locally small" category.} $\Hom(A,B)$ called the set of \textbf{morphisms} or \textbf{maps} or \textbf{arrows} such that
    \item for morphisms $f\in \Hom(A,B), g\in \Hom(B,C)$ there is a morphism $g\circ f\in \Hom(A,C)$. We also require:
      \begin{itemize}
        \item that composition of morphisms is associative: $h\circ(g\circ f)=(h\circ g)\circ f$, where $h\in \Hom(C, D)$
        \item we have morphisms $\Id_X: X\to X$ for every object $X$ such that $f\circ\Id_A = f=\Id_B\circ f$ for $f\in \Hom(A,B)$
      \end{itemize}
  \end{enumerate}
  If $f\in \Hom(A,B)$, we can also write $f:A\to B$ or $A\xrightarrow{f} B$. Some authors also write $\text{Mor}(A,B)$ for $\Hom(A,B)$ and $gf$ for $g\circ f$. If the collection of objects happens
  to be a set, we call it \textbf{small category}.
\end{definition}

\begin{example}
  We already know very well a category - the category of sets and functions. Let's check carefully that is actually is a category:
  \begin{enumerate}
    \item objects are just sets
    \item take $\Hom(A,B)$ as a set of all functions from $A$ to $B$ (why is it a set?)
    \item define composition of morphisms just as function composition
      \begin{itemize}
        \item composition of functions is associative (recall why)
        \item identity morhpism is just identity function of a set
      \end{itemize}
  \end{enumerate}
\end{example}

\begin{exercise}
  Consider a category with one singleton: $\{\{0\}\}$, where $\{0\}$ is the only object, and functions as arrows. How many arrows are in this category?
\end{exercise}

\begin{exercise}
  Consider a category with two singletons: $\{\{0\}, \{1\}\}$, and functions as arrows. How many arrows can be in this category? Hint: 4 numbers.
\end{exercise}

\subsection{Morphisms}
In set theory we introduced special functions - injections, surjections and bijections. We would like to generalise it into category theory. Unfortunately, we cannot just state their properties in terms of elements, as objects do not need to be sets and we cannot take a look at elements. Therefore, let's introduce the following:

\begin{definition}
  Let $f: A\to B$ be a morphism. We say that is it \textbf{left-cancellative} or that it is a \textbf{monomorphism} iff for every two morphisms $g,g' \in \Hom(X,A)$ we have:
  $$f\circ g = f\circ g'\Rightarrow g=g'.$$
  Sometimes monomorphism is written as $f: A\hookrightarrow B$. We will also refer to monomorphism as \textbf{monic} morphisms\footnote{Note that some authors distinguish between monomorphisms and monic
  morphisms.}.
\end{definition}

As you can prove, this works as injections in the category of sets:
\begin{exercise}
  Here you should prove that in the category of sets, "injective" are identical to "monic".
  \begin{enumerate}
    \item Prove that an injection in the category of sets is a monomorphism.
    \item Prove that if function $f: A\to B$ is not injective, then it is not monic. Hint: if $f(x)=f(y)$ for $x\neq y$, create functions from $\{x,y\}$ to $A$ showing that $f$ is not monic.
  \end{enumerate}
\end{exercise}

You can see that if we considered a different category, with less morphisms, the construction wouldn't work. So even if we consider sets and functions, but with some restrictions or additional structure, we need to carefully investigate the relation between monomorphisms and injections.

\begin{exercise}
  Consider a set $\{\{0,1\}, \{10,11\}\}$. Construct three morphisms such that this set becomes a category, but there is a non-injective monomorphism.
\end{exercise}

\begin{definition}
  Let $f: A\to B$ be a morphism. We say that is it \textbf{right-cancellative} or that it is a \textbf{epimorphism} iff for every two morphisms $g,g' \in \Hom(B,Y)$ we have:
  $$g\circ f = g'\circ f\Rightarrow g=g'.$$
  We will also refer to epimorphisms as \textbf{epic} morphisms\footnote{Note that some authors distinguish between epimorphisms and epic morphisms.}.
\end{definition}

\begin{exercise}
  Here you should prove that in the category of sets, "surjective" is identical to "epic".
  \begin{enumerate}
    \item Prove that a surjection in the category of sets is an epimorphism.
    \item Prove that if function $f: A\to B$ is not surjective, then it is not epic. Hint: if $f$ is not surjective, then both $\Image f$ and $B\setminus \Image f$ are non-empty. Define suitable functions.
  \end{enumerate}
\end{exercise}

Also here a problem may appear - there are categories in which surjections make sense, but are not identical to epimorphisms.

\begin{exercise}
  Create a category with a non-surjective epimorphism.
\end{exercise}

Also we have an object looking as bijection:

\begin{definition}
  Let $f: A\to B$ be a morphism. We say that is is an \textbf{isomorphism} if there is a morphism $g:B\to A$ such that $f\circ g=\Id_B$ and $g\circ f=\Id_A$.
\end{definition}

\begin{exercise}
  Prove that in every category an isomorphism is both monic and epic.
\end{exercise}

\begin{exercise}
  Create a category with a morphism that is both monic and epic, but is not an isomorphism.
\end{exercise}

But there are nice categories, as the category of sets, for which it holds.

\begin{exercise}
  Prove that in the category of sets, a morhpism is an isomorphism iff is bijective.
\end{exercise}

Isomorphisms will occur frequently in this book - in the section describing equivalence relations, we discovered a concept of identifying some objects (like splitting integers into two equivalence
classes odd and even numbers). We will usually try to identify some objects in a given category. Isomorphisms are very suitable for that.

\begin{exercise}
  "Equivalence" properties of isomorphisms:
  \begin{enumerate}
    \item Prove that if there is an isomorphism from $A$ to $B$, then there is an isomorphism from $B$ to $A$.
    \item Prove that there is an isomorphism from $A$ to $A$.
    \item Prove that if there is an isomorphism from $A$ to $B$ and $B$ to $C$, then there is also an isomorphism from $A$ to $C$.
  \end{enumerate}
\end{exercise}

This "equivalence" structure will be used to idenfify some spaces. Note that if this has properties of an equivalence relation, but being formal, a relation on a \emph{set} $A$ is a subset of $A\times A$. As we are dealing with categories now, that are not necessarily sets, we cannot say formally that isomorphisms give an equivalence relation. However, we will sometimes say that informally.

% \begin{definition}
%   If there is an isomorphism from $A$ to $B$, we will say that $A$ and $B$ are \textbf{isomorphic}. If there is a \emph{unique} isomorphism from $A$ to $B$, we will say that they are \textbf{naturally isomorphic}.
% \end{definition}

\section{Functors}
\begin{definition}
    Let $\mathcal C$ and $\mathcal D$ be two categories. A \textbf{covariant functor} $\mathcal F$ from $\mathcal C$ to $\mathcal D$ is an assignment\footnote{We cannot use word function - functions are special subsets of the Cartesian product of two sets: domain and codomain. Here you should think intuitively about it - as collections are in some sense bigger sets, functors are bigger versions of functions.} such that:
    \begin{enumerate}
      \item for each object $X$ of $\mathcal C$ there is an object $\mathcal F(X)$ in category $\mathcal D$
      \item for each morphism $f: X\to Y$ in $\mathcal C$ there is a morphism $\mathcal F(f):\mathcal F(X)\to \mathcal F(Y)$ such that:
        \begin{enumerate}
          \item $\mathcal F(\Id_X)=\Id_{\mathcal F(X)}$
          \item $\mathcal F(g\circ f) = \mathcal F(g)\circ \mathcal F(f)$ for $f:X\to Y$ and $g:Y\to Z$ being morphisms of $\mathcal C$.
        \end{enumerate}
    \end{enumerate}
\end{definition}


\section{Limits and colimits}
