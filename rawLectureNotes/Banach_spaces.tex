\chapter{Banach spaces}
\label{Banach_spaces}

\begin{quote}
  Q: What's yellow, normed, and complete?\\
  A: A Bananach space.

  Standard Functional Analysis joke, from\\
  \texttt{http://dominic-mazzoni.com/mathanswers.html}
\end{quote}

We know quite well vector spaces and metric spaces, it's high time we started investigating
the mixture of these properties - a vector spaces with sufficiently nice topology placed on them.

\begin{definition}
  Let $V$ be a vector space over real or complex numbers and $n: V\to R$ be a function such that:
  \begin{enumerate}
    \item $n(v)\ge 0$ for all $v$
    \item $n(\alpha v)=|\alpha|\cdot n(v)$
    \item $n(v+u)\le n(v)+n(u)$
  \end{enumerate}
  Then we call $n$ a \textbf{seminorm}. A seminorm such that:
  \begin{enumerate}
  \setcounter{enumi}{3}
    \item $n(v)=0$ iff $v=0$
  \end{enumerate}
  is called a \textbf{norm}. A pair $(V,n)$ is called then a (semi)normed space. We usually write $\norm{v} := n(v)$ and $n=\norm{\cdot}.$
\end{definition}

Each normed space is in a natural way a metric, and therefore, a topological space, as you can prove.

\begin{exercise}
  Let $(V, \norm{\cdot})$ be a seminormed space. Prove that $$d(v,u)=\norm{v-u}$$ is a pseudometric on $V$.
  Prove that $d$ is metric iff $\norm{\cdot}$ is a norm.
\end{exercise}

A simple corollary from that is:

\begin{exercise}
  Prove that in a normed space:
  \begin{enumerate}
    \item limits of sequences are unique (hint: metric spaces are Haussdorff)
    \item we can use just sequences to characterise compactness and continuity (without nets and filters) (hint: how does it work in metric spaces?)
  \end{enumerate}
\end{exercise}

Let's think now what changes if we change a norm.

\begin{definition}
  Let $n$ and $m$ be two norms on a vector space $V$. We say that norm $n$ is \textbf{stronger} than norm $m$ iff the topology generated than $n$ is stronger than the
  topology generated by $m$. We say that they are \textbf{equivalent} iff they generate the same topology.
\end{definition}

\begin{exercise}
  Let $n$ and $m$ be two norms on a vector space $V$. Prove that the following statements are equivalent:
  \begin{enumerate}
    \item $n$ is stronger than $m$
    \item for arbitrary $v\in V$ and arbitrary sequence $v_n\to v$ in sense of topology generated by $n$, we have $v_n\to v$ in sense of topology generated by $m$
    \item function $v\mapsto m(v)/n(v)$ is bounded on set $V\setminus\{0\}$
    \item there exists a number $a>0$ such that $m(v)\le a\cdot n(v)$ for every $v\in V$
  \end{enumerate}
\end{exercise}

\begin{exercise}
  Prove that "norm $n$ is equivalent to norm $m$" is an equivalence relation on the set of all norms defined on a given vector space. 
\end{exercise}

\begin{exercise}
  Let $V$ be a finite dimensional space. Prove that each two norms are equivalent. Hint: pick a basis and a norm assigning 1 to each element of the basis.
\end{exercise}

\begin{definition}
  A normed and \emph{complete} vector space is called a \textbf{Banach space}.
\end{definition}
