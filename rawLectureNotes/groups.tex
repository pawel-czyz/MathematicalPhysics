% !TEX root = book.tex
\chapter{Groups}
\label{groups}
% use \chaptermark{} to alter or adjust the chapter heading in the running head
\begin{definition}
  A $\textbf{group}$ is a pair $(G, \cdot)$, where $G$ is a set and $\cdot: G\times G\to G$ is a function called \textbf{group law} with properties listed below. It's a common practice to write $a\cdot b := \cdot(a,b)$ or even $ab := \cdot(a,b)$. These properties are:
  \begin{enumerate}
    \item associativity: for every $a,b,c\in S$ we have $(ab)c=a(bc)$
    \item existence of right identity: there is $e\in G$ such that $ae=a$ for every $a\in G$
    \item existence of right inverse: for each $a\in G$ there exists an element called $a^{-1}$ such that $aa^{-1}=e$
  \end{enumerate}
  If $ab=ba$ for every $a,b\in G$, we say that the group is \textbf{abelian}. If $|G|$ is finite, we say that the group is \textbf{finite}.
  Sometimes group $(G, \cdot)$ is referenced as just $G$, if group law is known from the context.
\end{definition}

Before we start investigating the properties, we will provide some examples.

\begin{example}
  Pair $(\Z, +)$ is a group, as $(a+b)+c=a+(b+c)$, $a+0=a$ and $a+(-a)=0$ for every $a,b,c\in \Z$. The role of $e$ is played by 0 and right inverse $a^{-1}$ is known just as $-a$.
  Moreover it is abelian, as $a+b=b+a$ for every $a,b\in \Z$.
\end{example}

\begin{exercise}
  Let $S$ be a set and $\mathcal S$ be a set of all bijections from $S$ to $S$. Prove that $(\mathcal S, \circ)$ is a group, where $\circ$ is usual function composition. Why this group is usually not abelian?
\end{exercise}

As we are faimilar with groups, we can start investigating their properties. First concern is the word \emph{right} - why is it right, and not left?

\begin{exercise}
  Right, left, whatever - it is two-sided.
  Let $(G,\cdot)$ be a group with right identity $e$. You will prove that $e$ is two-sided identity, that is $ae=ea$ for every $a$. Moreover you will prove that $a^{-1}$ is two-sided inverse: $a^{-1}a=aa^{-1}=e$.
  \begin{enumerate}
    \item Prove that if $a\in G$ is indempotent, that is $aa=a$, then $a=e$. Hint: what is $aaa^{-1}$?
    \item Prove that right inverse is also left inverse, that is $a^{-1}a=e$ for every $a$. Hint: what is $a^{-1}aa^{-1}a$?
    \item Prove that right identity is also left identity. Hint: what is $aa^{-1}a$?
  \end{enumerate}
\end{exercise}

As we fixed this issue, we can start thinking about uniqueness of identity and inverse

\begin{exercise}
  Consider a group $G$ with identity $e$.
  \begin{enumerate}
    \item Prove that identity is unique, that is if there is an element $e'$ such that $ae'=a$ for every $a$, then $e=e'$.
    \item Prove that inverse is unique, that is if $ab=e$, then $b=a^{-1}$
    \item What is the inverse of $a^{-1}$?
  \end{enumerate}
\end{exercise}

Therefore we can give an equivalent definition of group:

\begin{definition}
  A $\textbf{group}$ is a pair $(G, \cdot)$, where $G$ is a set and $\cdot: G\times G\to G$ is a function called \textbf{group law} (written as $\cdot(a,b)=ab$) with the following properties:
  \begin{enumerate}
    \item associativity: for every $a,b,c\in S$ we have $(ab)c=a(bc)$
    \item existence of identity: there is \emph{a unique} $e\in G$ such that $ae=ea=a$ for every $a\in G$
    \item existence of inverse: for each $a\in G$ there exists \emph{a unique} $a^{-1}\in G$ such that $aa^{-1}=a^{-1}a=e$
  \end{enumerate}
\end{definition}

This definition looks much better than the previous - we have more properties clearly listed and we do not need to reference each time e.g. the proof that inverse is two-sided, we can just use it. The disadvantage
is that proving that something is a group is \emph{longer} - we have more properties to check. So our strategy will be to check whether group law candidate is associative, provide \emph{an} identity and a rule how to find \emph{an} inverse, and since then using all the properties listed in our "powerful" definition.

\begin{exercise}
  Prove that in every group: $(ab)^{-1}=b^{-1}a^{-1}.$
\end{exercise}

\begin{exercise}
  Let $G$ be an abelian group\footnote{Look, we dropped a group law sign.}. Prove that there is exactly one $x\in G$ such that $ax=xa=b$ and define division in a group. Do you understand now what is $a-b$ in abelian group $(\Z, +)$?
\end{exercise}

\begin{exercise}
  Let $G$ be a finite group with an element such that $a^2=e$, where $e$ is the identity and $a\neq e$. Prove that $|G|$ is odd.
\end{exercise}

% \begin{exercise}
%   Consider a set $S$ and an \emph{associative} operation $\cdot: S\times S\to S$. We assume that for every $a\in S$ there exists a \emph{unique} pseudoinverse, that is element $\bar a\in S$ such that $a\bar aa=a$.
%   Prove that $(S, \cdot)$ is a group.
% \end{exercise}

\section{The category of groups}
To make all groups into a category, we need a notion of a morphism. As the only operation that is allowed in groups is group law, it makes sense to define the following map.

\begin{definition}
  Consider two groups: $(G,\cdot)$ and $(H, \star)$. A function $f: G\to H$ such that:
  $$f(a\cdot b)=f(a)\star f(b)$$
  is called a \textbf{homomorphism}.
\end{definition}

To prove that it is indeed a category, we need to check a few things:

\begin{exercise}
  Prove that:
    \begin{enumerate}
      \item All homomorphisms between groups $G$ and $H$ form a set
      \item If $f: F\to G$ and $g: G\to H$ are homomorphisms, then $g\circ f$ is a homomorphism
      \item moreover $f\circ \Id_F=f=\Id_G\circ F$, where $\Id_X$ is the usual identity function on a set $X$
    \end{enumerate}
\end{exercise}



% \section{Banach-Tarski paradox}
% With some help of the Axiom of Choice and group theory machinery, we will find out how to cut one orange into a few pieces and move them a bit in such way that we get two oranges.
