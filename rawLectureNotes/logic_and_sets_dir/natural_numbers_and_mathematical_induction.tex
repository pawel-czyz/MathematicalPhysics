\section{Mathematical induction}
\label{sec:mathematical_induction}
Have you ever seen falling dominoes? To be sure that every domino falls, we need to:
\begin{enumerate}
	\item punch the first domino
	\item for every domino we must be sure the implication: if this particular domino falls, the next one also falls
\end{enumerate}
And that's all, we can be sure that all the dominoes will eventually fall. This style of reasoning\footnote{We do not show here formally \textit{why}
this principle works. For curious, you define natural numbers in such way this principle works.} is called \textbf{mathematical induction} and
formally it is written as: if $0\in S$ and for every\footnote{I repeat: for every $n$ we need to prove the implication ,,if works for $n$, then
works for $n+1$". The correct way is to write ,,I assume that there is a given $n$ for which the formula works. I will prove that is works for $n+1$".
Common mistake is to write ,,I assume that the formula works for every $n$ and I will prove that it works for $n+1$.". As professor Wiktor Bartol says
- there is no need to prove the statement as you already assumed that it works in every case.}
 $n\in N$ you can prove the implication $n\in S$ then $n+1\in S$, you know that $N\subseteq S$.
\begin{prob}
	You can prove that $2^n>n$ for every natural number $n$.
	\begin{enumerate}
		\item Prove that the formula works for $n=0$ (punch the first domino).
		\item Assume that for some $n$ you proved on some way that $2^n>n$. Using this, prove that $2^{n+1}>(n+1)$ (if $n$-th domino falls, then
		$n+1$-th domino also falls)
	\end{enumerate}
\end{prob}

\noindent You can also modify slightly the induction principle - sometimes you should start with number different than 0 or use different induction step
(start 0 and step 2 can lead to theorems valid for even numbers, step 0 and steps 1 and -1 can lead to theorems valid for all integers...)
\begin{prob}
    \begin{enumerate}
	   \item Prove\footnote{Another method is to notice that $n^3-n=(n-1)\cdot n\cdot (n+1)$. Why 2 does divide it? Why 3?} that 6 divides
		     $n^3-n$ for all natural $n$.
	    \item Prove\footnote{How $n^3-n$ and $(-n)^3-(-n)$ are related? Does it simplify the proof?} that 6 divides $n^3-n$ for all integers $n$.
		      You can use a slight modification mathematical induction principle proving the implication
		      ,,if the theorem works for $n$, it works also for $n-1$".
    \end{enumerate}
\end{prob}

\begin{prob}
	(Bernoulli's inequality) Prove that for real $x > -1$ and natural $n\ge 1$, the following inequality holds:
	$$(1+x)^n\ge 1+nx.$$
\end{prob}

\begin{prob}
	In Mathsland there are $n\ge 2$ cities. Between each pair of them there is a \textit{one-way} road.
	\begin{enumerate}
		\item Prove that there is a city from which you can drive to all the other cities. Hint: assume that the hypothesis works for some $n$ and any
			country with $n$ cities. Now consider an arbitrary $n+1$-city country. Hide one city and use your assumption.
		\item Prove that there is a city\footnote{Nice trick: what does happen if you reverse each way? Can you use the former result?}
			to which you can drive from all the others.
	\end{enumerate}
\end{prob}

\begin{prob}
	Let $S\subseteq R$. We say that $S$ is \textbf{well-ordered} iff any non-empty subset $X\subset S$ has the smallest element.
	\begin{enumerate}
		\item Prove that reals and integers with the default ordering are not well-ordered.
		\item Assume that $X\subseteq \mathbb N$ doesn't have the smallest element. Define $A=\{n\in \mathbb N : \{0,1,\dots,n\}\cap X=\emptyset\}$
			and use mathematical induction to prove that $X$ is empty.
		\item Why are natural numbers well-ordered?
	\end{enumerate}
\end{prob}
