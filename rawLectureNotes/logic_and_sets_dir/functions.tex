\section{Functions}
\label{sec:intro_to_functions}

\subsection{Basics}
\noindent Consider two sets $A$ and $B$. We say that a subset $f\subseteq A\times B$ is a \textbf{function}
iff the following two conditions hold:
\begin{itemize}
	\item for every element $a\in A$ there is an element $b\in B$ such that $(a,b)\in f$
	\item if $(a,b)\in f$ and $(a,c)\in f$, then $b=c$
\end{itemize}
Therefore for each $a\in A$ there is exactly one $b\in B$ such that $(a,b)\in f$. Such $b$ will be called
\textbf{value of $f$ at point $a$} and given a symbol $f(a).$

\begin{example}
  $f: \N\to \R$ given by $f(n)=n$.
\end{example}

\begin{example}
  $f: \R\to \R$ given by $f(x)=x^2$.
\end{example}

\begin{example}
  $f: X\to \mathcal P(X)$ given by $f(x)=\{x\}$.
\end{example}

\begin{example}
  Let $\equiv$ be an equivalence relation on a set $X$. Recall that we
  defined $X/R$ as set of all equivalence classes of $\equiv$. We have a function
  $f: X\to X/R$ given by $f(x)=[x].$
\end{example}

\begin{prob}
	How many\footnote{Thanks to Antek Hanke} are there functions from the empty set to $\{1,2,3,4\}?$
\end{prob}

\begin{exercise}
  Here, we will prove a simple inequality using set-theoretic reasoning. Let $X$ and $Y$ be finite sets, with numbers of elements, respectively, $x$ and $y$.
  \begin{enumerate}
    \item Prove that the number of relations between $X$ and $Y$ is $2^{xy}$.
    \item Prove that the number of functions from $X$ to $Y$ is $y^x$. Hint: for first element in $X$ you have $y$ possibilities to choose.
    \item Prove that for every non-zero natural numbers $x$ and $y$ the following holds:
      $$y^x<2^{xy}.$$
  \end{enumerate}
\end{exercise}

\begin{exercise}
  Let $X$ and $Y$ be any two sets. Prove that you can create a set of all functions from $X$ to $Y$. Sometimes it is called $Y^X$. Do you see why?
\end{exercise}

\begin{exercise}
  Consider a function $f: X\to X'$ and assume that there is an equivalence relation $R'$ on $X'$. We will try to define a natural (in some sense) equivalence relation on $X$.
  \begin{enumerate}
    \item Define a relation $R$ on $X$ as $xRy\Leftrightarrow f(x) R' f(y)$. Prove that it is an equivalence relation.
    \item Consider $r: X\to X/R$ and $r': X'\to X'/R'$ given by $r(x)=[x]_R$ and $r'(x')=[x']_{R'}$ and inverse function.
  \end{enumerate}
\end{exercise}

We need to introduce more terminology: set $A$ is called \textbf{the domain of $f$}, set $B$ is called
\textbf{the codomain of $f$} and the function $f$ is written as $f: A\to B$.

\begin{prob}
	Consider two functions: $f:\{0, 1\}\to \{0,1\}$ given by $f(x)=0$ and $g:\{0,1\}\to\{0\}$.
	Prove that $f=g$.
	\footnote{Some mathematicians, as Bourbaki use an alternative definition of function - for them
	a function is the triple $(A,B,f)$, where $f$ is defined as in the our case. We see that this definition
	is incompatible with ours. Fortunately, as in the case with different definitions of ordered tuples,
	this problem will never occur explicitly in the further chapters.}
\end{prob}

\begin{prob}
	Let $f:A\to B$ and $g: C\to B,$ where $A\neq C$. Is it possible that $f=g$?
\end{prob}

\begin{prob}
	Let $f: A\to B$ and $C\subseteq D\subseteq A$. We define: $f[C] = \{b\in B : b=f(c) \text{ for some }c\in C \}$ and analogously $f[D]$. Prove that
	$f(C)\subseteq f(D).$
\end{prob}

\subsection{Injectivity, surjectivity and bijectivity}

\noindent As we have already seen, there may be some elements in codomain that are not values of
$f$. We define \textbf{the image of $f$} as:
$$\text{Im}\, f = \{b\in B : \text{there is } a\in A \text{ such that } b=f(a)\}.$$
We say that the function $f: A\to B$ is \textbf{surjective} (or \textbf{onto}) iff $\text{Im}\,f=B$.

\begin{prob}
	As we remember, $\mathbb{R}$ stands for well-known real numbers. Are the following functions surjective?
	\begin{enumerate}
		\item $f: \mathbb{R} \to \mathbb{R}, ~f(x)=x^3$
		\item $g: \mathbb{R} \to \mathbb{R}, ~g(x)=x^2$
		\item $h: \mathbb{R} \to \{5\}$
	\end{enumerate}
\end{prob}

If $f(a)$ uniquely specifies $a$ (if $f(a)=f(b)$, then $a=b$) we say that the function is \textbf{injective}
(or \textbf{one-to-one}).
\begin{prob}
	As we remember, $\mathbb{R}$ stands for well-known real numbers. Are the following functions injective?
	\begin{enumerate}
		\item $f: \mathbb{R} \to \mathbb R, ~f(x)=x^2$
		\item $h: \{0,1,2,3\} \to \mathbb R, ~h(x)=x$
	\end{enumerate}
\end{prob}

If a function $f$ is both surjective and injective, we say that is \textit{bijective}\footnote{If you prefer nouns: surjective function is called surjection, injective - injection
and bijective - bijection}.

\begin{prob}
	Construct function that is:
	\begin{enumerate}
		\item surjective, but not injective
		\item injective, but not surjective
		\item neither injective nor surjective
		\item bijective
	\end{enumerate}
\end{prob}

\noindent Notice that if a function $f: A\to B$ is bijective, then we can construct a function $g:B\to A$
such that $f(g(b))=b$ and $g(f(a))=a$.
\begin{prob}
	Prove that, if exists, $g$ is unique.
\end{prob}

\noindent We call this function \textbf{the inverse function}
\footnote{It becomes confusing when working on real numbers: $f^{-1}(x)$ is
\textbf{not} $(f(x))^{-1}=1/f(x)$}: $g=f^{-1}.$

\begin{prob}
	Assume that $f^{-1}$ exists. Prove that $(f^{-1})^{-1}$ exists and is equal to $f$.
\end{prob}

\subsection{Function composition}
If we have two functions: $f:A\to B$ and $g: B\to C$, we can construct the \textbf{composition} using formula:
$g\circ f: A\to C,~(g\circ f)(a) = g(f(a)).$

\begin{exercise}
  Recall that for two relations $R\subseteq X\times Y$ and $T\subseteq Y\times Z$ we defined their composition as $$R\circ T=\{(x,z)\in X\times Z : \exists_{y\in Y} (x,y)\in R \wedge (y,z)\in T\}$$
\end{exercise}

\begin{prob}
	Find functions $f,~g$ such that:
	\begin{enumerate}
		\item $g\circ f$ exists, but $f\circ g$ is not defined
		\item both $f\circ g$ and $g\circ f$ exist, but $f\circ g\neq g\circ f$
	\end{enumerate}
\end{prob}

\noindent Although function composition is not commutative, it is associative:
\begin{prob}
	Left $f:A\to B, g: B\to C, h: C\to D$. Prove that
	$$h\circ (g\circ f) = (h\circ g)\circ f.$$
\end{prob}
Therefore we can ommit the brackets and write just $h\circ g\circ f.$ We will use function composition very
often.

\begin{prob}
    \begin{enumerate}
	   \item Prove that composition of two surjections is surjective.
	   \item Prove that composition of two injections is injective.
	   \item Prove that composition of two bijections is bijective.
    \end{enumerate}
\end{prob}

\begin{prob} We will rephrase the definition of the inverse function as follows:
	\begin{enumerate}
		\item If $X$ if a set, we define \textbf{the identity function}
			$$\text{Id}_X=\{(x,x)\in X^2 : x\in X\}.$$
			Prove that it is indeed a function. What is it's domain?
		\item Let $f:A\to B,~g:B\to A$. Prove that $f=g^{-1}$ iff
			$$g\circ f = \text{Id}_A \text{ and } f\circ g = \text{Id}_B$$
	\end{enumerate}
\end{prob}

\begin{prob}
  Let $f: A\to B$ be an injection. Prove that there is a function
  $g: \text{Im\,} f \to A$ such that $g\circ f = \text{Id}_A.$
  Such $g$ is called \textbf{left inverse of $f$}.
\end{prob}
