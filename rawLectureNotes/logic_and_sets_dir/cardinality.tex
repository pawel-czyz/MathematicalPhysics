\section{Cardinality}
\subsection{Finite sets}
For a finite set $X$ we write the number of elements of $X$ as $|X|$. We can calculate their \textbf{cardinalities} (sizes, numbers of elements) with
ease,
\begin{exercise}
	What is the cardinality of $\{a, a+1, a+2, \dots, a+n\}$?
\end{exercise}

\begin{exercise}
	Let $A,\,B$ and $C$ be finite sets. Prove that:
	\begin{enumerate}
		\item $|2^A|=2^{|A|}$
		\item $|A\cup B|=|A|+|B|$ iff $A$ and $B$ are disjoint.
		\item $|A\setminus B|=|A|-|B|$ if $B\subseteq A.$
		\item $|A| \ge |B|$ if $B\subseteq A$. When does the equality hold?
		\item $|A\cup B| = |A| + |B| - |A\cap B|$
		\item $|A\cup B\cup C| = |A|+|B|+|C| - |A\cap B| - |B\cap C|-|C\cap A| + |A\cap B\cap C|$
	\end{enumerate}
\end{exercise}

We can also employ functions to compare cardinalities:
\begin{exercise}
	Assume that $A$ and $B$ are finite sets. Prove that $|A|=|B|$ iff there is a bijection between $A$ and $B$.
\end{exercise}

\begin{exercise}
	Above we find the way of saying that two cardinalities are equal using existence of a bijection. Let's find a way to compare which is less using
	another kind of function.
	\begin{enumerate}
		\item Let $O_n=\{1,2,\dots,n\}.$ Prove that there is no injection from $O_{n+1}$ into $O_n$. Hint: use mathematical induction.
		\item Let $A$ and $B$ be finite. Prove that there is an injection from $A$ to $B$ iff $|A| \le |B|.$
	\end{enumerate}
\end{exercise}

\begin{exercise}
	Prove in one: if there is an injection from $A$ onto $B$ and an injection from $B$ into $A$, then there exists a bijection from $A$ onto $B$. Hint: you know that $|A|\le B$ and $|B|\le |A|$.
\end{exercise}

\subsection{Infinite sets}
But how can we measure the number of elements of an infinite set, as $\mathbb N$ or $\mathbb R$? As natural numbers are ,,to small" we need to
introduce new numbers, as $|\mathbb N|$ and be able to compare them. As we have seen above, the existence of a bijection is a good way of saying
that two finite sets have equal cardinalities. It intuitively makes sense to employ this observation even in the infinite case: we say that sets
(finite or infinite) $A$ and $B$ have the same caridnalities (or $|A|=|B|$) iff there is a bijection between $A$ and $B$.

\begin{prob}
	Let $A,\,B$ and $C$ be sets. Prove that if $|A|=|B|$ and $|B|=|C|$, then $|A|=|C|.$ Hint: find the bijection between $A$ and $C$.
\end{prob}

\noindent Here you can see the difference between finite and infinite sets - for finite sets a proper subset (a subset that is not the whole set)
always has smaller number of elements. In the infinite case it is not true, as a proper subset can have \textit{the same} number of elements.

\begin{prob}
	Prove that:
	\begin{enumerate}
		\item $|\mathbb N| = |\mathbb Z|$.
		\item $|\mathbb N|=|\mathbb N\times \mathbb N|$.
		\item $|\mathbb N|=|\mathbb Q|$.
	\end{enumerate}
\end{prob}

\noindent Analogously to the finite case, we define $|A|\le |B|$ as the existence of an injection from $A$ to $B$. We say that $|A|<|B|$ iff there
is an injection from $A$ to $B$ but there is no bijection.

\begin{prob}
	Prove that if $A\subseteq B$, then $|A|\le|B|.$
\end{prob}

\begin{prob}
	Let $A,\,B$ and $C$ be sets. Prove that if $|A|\le |B|$ and $|B|\le |C|$, then $|A|\le |C|$.
\end{prob}

\begin{prob}
	Here you can prove that there are more real numbers than naturals or rationals. We define $X=\{x\in \mathbb R : 0\le x\le 1\}$ and choose one
	 convention of writing reals (e.g 0.999... = 1.000..., so we can choose to use nines)
	\begin{enumerate}
		\item Assume that you have written all the elements of $X$ in a single column. Can you find a real number that does not occur in the list?
		\item Using the above, prove that $|\mathbb N| < |X|$
		\item Prove that $|\mathbb Q| < |\mathbb R|.$
	\end{enumerate}
\end{prob}

\begin{prob}
	We know that $|\mathbb R| > |\mathbb N|.$ Using binary system prove that $\mathbb R=2^\mathbb N.$ Do you see similarity between the previous result
	and $2^n > n$ for natural $n$?
\end{prob}

\begin{prob}
	\textbf{Cantor's theorem} You will prove that $|A|<\left|2^A\right|$ for any set $A$. Let $A$ be a set and $f:A\to 2^A.$
	\begin{enumerate}
		\item Consider $X=\{a\in A : a\notin f(a)\}\in 2^A$. Is there $x\in A$ for which $f(x)=X?$
		\item Is $f$ surjective?
		\item Find an injective function $g: A\to 2^A.$
		\item Prove that $|A| < |2^A|$ for any set $A$.
		\item Use Cantor's theorem to prove that there is no set of all sets.
	\end{enumerate}
\end{prob}

\begin{prob}
	\textbf{Cantor-Schroeder-Bernstein theorem} Let's prove that if $|A|\le|B|$ and $|B|\le |A|$, then $|A|=|B|$ for any sets.
	\begin{enumerate}
		\item (Knaster-Tarski) Now assume that $F$ has \textit{monotonicity} property: $F(X)\subseteq F(Y)$ if $X\subseteq Y$.
			Prove that $F$ has a fixed point $S$ (that is $F(S)=S$), where:
			$$S=\bigcup_{X\in U} X, \text{~where~} U= \{Y\in 2^A : Y\subseteq f(Y)\}.$$
		\item (Banach) Let $f: A\to B$ and $g:B\to A$ be injections.
			We introduce new symbol: $f[X]=\{b\in B : b=f(x) \text{ for some } x\in X\}$. Prove that
			function $$F:2^A\to 2^A,~F(X)=A\setminus g[B\setminus f[X]]$$
			has the monotonicity property.
    \item Prove that $A\setminus S\subseteq \text{Im}\,g$, where
      $F$ and $S$ are taken from above.
		\item Prove that function
			$$h(x) =
				\begin{cases}
					f(x), x\in S\\
					g^{-1}(x), x \notin S
				\end{cases}
			 $$
			 is a bijection.
	\end{enumerate}
\end{prob}
