\section{Real numbers}
At the beginning we assumed that you had some intuition what real numbers
are and how to work with them - to provide examples and make set theory less
abstract. But we have not treated them rigorously, as we did not have proper
glossary - it's high time we filled this gap and defined them properly.
It's high time we defined them properly, as we .
A \textbf{field} is a tuple $(F, +, \cdot, 1, 0).$ We have
many symbols there, let's explain what they mean:
\begin{itemize}
  \item $F$ is a set
  \item $+$ and $\cdot$ are functions from $F^2$ to $F$. We write
    $a+b$ for $+(a,b)$ and $a\cdot b$ for $\cdot (a,b)$.
  \item $1, 0\in F$ are just distinguished elements of $F$
\end{itemize}
We know what objects are in the definition, so we can talk about
properties they must have to form a field:

\begin{enumerate}
  \item $1\neq 0$ (so $F$ has at least two elements)
  \item $a+(b+c)=(a+b)+c$ for all $a,b,c$ (addition is associative)
  \item $a\cdot (b\cdot c)=(a\cdot b)\cdot c$ for all $a,b,c$ (multiplication is associative)
  \item $a+b=b+a$ for all $a, b$ (addition is commutative)
  \item $a\cdot b=b\cdot a$ for all $a,b$ (multiplication is commutative)
  \item $a+0=a$ for all $a$ (so 0 is neutral element of addition)
  \item $a\cdot 1=a$ for all $a$ (so 1 is neutral element of multiplication)
  \item for every $a$ there is $a'$ such that $a+a'=0$ (existence of
    an inverse element for addition)
  \item $a\cdot (b+c) = a\cdot b + a\cdot c$ for all $a,b, c$
    (multiplication distributesover addition)
  \item for every $a\neq 0$ there is $\tilde a$ such that $a\cdot \tilde a=1$
    (multiplication has an inverse element for all non-zero numbers)
\end{enumerate}

\begin{prob}
  Check that
  \begin{enumerate}
    \item real numbers understood informally, have the properties
      listed above
    \item rational numbers form a field
  \end{enumerate}
\end{prob}

From the above field axioms, you can derive many facts that may be obvious
to you:

\begin{prob}
  Prove that there is only one 0 and only one 1. Hint: assume that
  0 and 0' have property such that $a=a+0=a+0'$ and try $a=0$ and $a=0'$.
\end{prob}

\begin{prob}
  Prove that if $a+a'=0$ and $a+a''=0$, then $a'=a''$.
  Therefore we can introduce special symbol for \textit{the} additiv
  inverse: $a + (-a)=0$ and define subtraction as $a-b := a + (-b)$.
\end{prob}

\begin{prob}
  Prove that $-a=(-1)\cdot a$.
\end{prob}

% Consider more problems on this topic

As you see, many of the algebraic properties we are used to can be recovered from the axioms, but sometimes it can be complicated. Both real numbers and
rational numbers have also an order on them - for example $2>1$. It leads
to the definition of \textit{total order}. We call a pair $(F, \le)$ a \textit{totally ordered set} if for every $a,b\in F$ we have:
\begin{enumerate}
  \item $a\le b$ or $b\le a$ (we call this property totality)
  \item $a\le b$ and $b\le a$ imply $a=b$ (it's called antisymmetry)
  \item $a\le b$ and $b\le c$ imply $a\le c$ (transitivity)
\end{enumerate}
Having relation $\le$ we can define other: $b\ge a$ means that $a\le b$ and
$a<b$ means that $a\le b$ and $a\neq b$.

We say that tuple $(F, +, \cdot, 1, 0, \le)$ is \textbf{ordered field} if:
\begin{itemize}
  \item $(F, +, \cdot, 1, 0)$ is a field
  \item $(F, \le)$ is totally ordered
  \item $a\le b$ implies $a+c\le b+c$
  \item $0\le a$ and $0\le b$ imply that $0\le a\cdot b$
\end{itemize}

You can check that reals and rationals are ordered fields. These axioms give
us much more abilities, for example one is able to prove that $1>0$.
But we still have no difference in properties that distuingish rationals
from reals. This is called the completeness axiom and we will need a few
more definitions.

Consider $A\subseteq \mathbb R$. We say that $x$ is an
\textbf{upper bound} of $A$ iff $x\ge a$ for every $a\in A$.

\begin{prob}
  Prove that a set $A\subseteq \mathbb R$ can have no upper bounds
  or infinitely many of them.
\end{prob}

\noindent If an upper bound of $A$ exists, we say that $A$ is
\textbf{bounded from above}. Among them we will distinguish the
\textbf{supremum} (or \textbf{the least upper bound - l.u.b}):
$x=\sup A$ iff $x$ is an upper bound of $A$ and for any upper bound
$y$ of $A$ we have $x\le y$.

\begin{prob}
  Prove that supremum is unique, so if $x$ and $x'$ are supremums
  of $A$, then $x=x'$.
\end{prob}

\begin{prob}
  Prove that $x=\sup A$ if and only if
  $x\ge a$ for every $a\in A$ and for every $\eps > 0$ there is
  $a\in A$ such that $x < a + \eps$.
\end{prob}

\noindent Now we can state the \textbf{completeness axiom}:
each non-empty and bounded from above subset of real numbers has
a supremum.
This axiom allows us to prove many interesting things:

\begin{prob}
  Prove that natural numbers are \textit{not} bounded from above.
  Hint: if $n\in \mathbb N$, then $n+1\in \mathbb N$
\end{prob}

\begin{prob}
  Prove the \textbf{Archimedean axiom}\footnote{In fact we do not
  need to call it axiom, as we are able to prove it.}
  that for every $r\in R$, there is $n\in \mathbb N$ such that $n>r$.
\end{prob}

\begin{prob}
  Prove that for any $r>0$ there is $n\in \mathbb N$ such that
  $1/n < r$.
\end{prob}

\begin{prob} Find infinite sum and intersection for the families of subsets of $\mathbb{R}$:
  \begin{enumerate}
    \item $A_i=(0,1/i)$ for $i=1,2,\dots$
    \item $B_i=[0,1/i)$ for $i=1,2,\dots$
  \end{enumerate}
\end{prob}

\begin{prob}
  Prove that rational numbers do \textit{not} have the completeness
  property:
  \begin{enumerate}
    \item Let $p, q\in \mathbb Z\setminus \{0\}$.
      Prove that $p^2\neq 2q^2$.
    \item Prove that root of two, defined as
      $x > 0, x^2=2$ is not rational.
    \item Find a subset of $\mathbb Q$ that is bounded above, but
      has no rational supremum.
  \end{enumerate}
\end{prob}

\begin{prob}
  You should prove that in each nonempty interval there is at least
  one rational number:
  \begin{enumerate}
    \item Assume that $0<a<b$. Define
      $$A=\left\{\frac m N : m\in \mathbb N\right\},~
      \frac 1{b-a} < N \in \mathbb N$$
      and prove that $A\cap (a,b)$ is non-empty.
    \item Use the above result to prove that in \textit{each}
      interval there is at least one rational number.
    \item Prove that in each interval there are infinitely but countably many, rational numbers.
    \item Prove that in each interval there is an irrational number.
    \item How many irrational numbers are in each interval?
  \end{enumerate}
\end{prob}

\subsection{Absolute value}

Another concept that will be further useful is the \textbf{absolute value} of a real number:
if $x\in \mathbb R$ we write $|x|\in \mathbb R$ for:
$$|x| = \begin{cases}x &\text{ for } x \ge 0\\ -x &\text{ otherwise} \end{cases}.$$

\begin{prob}
  Prove that for every $x,y\in \mathbb R$:
  \begin{enumerate}
    \item $|x|=|-x|$
    \item if $|x|=|y|$ then $x=y$ or $x=-y$.
    \item $|x+y| \le |x| + |y|$ (this is called \textbf{triangle inequality})
    \item $|x-y|\le |x| + |y|$
    \item $\left||x| - |y|\right|\le |x-y|$ (this is sometimes calles \textbf{reverse triangle inequality})
  \end{enumerate}
\end{prob}
