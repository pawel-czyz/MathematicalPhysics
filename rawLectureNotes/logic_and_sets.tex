% !TEX root = book.tex

\chapter{Logic and sets}
\label{logic_and_sets} % Always give a unique label
% use \chaptermark{}
% to alter or adjust the chapter heading in the running head

Logic is a huge and beautiful branch of mathematics. We will focus on it's basics, topic called ,,propositional calculus". It is a powerful machinery, that will be used later
to prove theorems and define new objects. Moreover, it gives a good grasp on Boolean algebras, a concept that we will later meet in topology.

\section{Propositional calculus}
\subsection{New sentences from old}
\label{sec:logic}
Consider declarative sentences as "It's raining in Oxford now." or "2+2=5" that can be either true or false. There are many ways how to construct new sentences and decide
whether they are true or not.

\begin{definition}
  % Equivalence - definition
  Consider sentences $p$ and $q$. We say that they \textbf{are equivalent} (we write then $p\Leftrightarrow q$) if they are either true or false simultaneously.
  If $p$ and $q$ are equivalent, we usually say "$p$ if and only if $q$" of even "$p$ iff $q$".
\end{definition}

\begin{example}
  % Equivalence - example, people in a room
  Let $p$ be a sentence "The number of people in the room you are sitting is odd" and $q$ be "If one person enters the room, then the number of people will become even".
  We \textit{do not know} if any of these sentences is true - it would require to count all the people. But if $p$ is true, then also $q$ must be true and vice versa - if $q$ is
  true, then also $p$ must be true. Therefore we can say that $p$ and $q$ are equivalent, or write $p\Leftrightarrow q$.
\end{example}

\begin{exercise}
  % Equivalence - transitivity
  Prove that if we know that $p\Leftrightarrow q$ and $q\Leftrightarrow r$, then also $p\Leftrightarrow r$.
\end{exercise}

\begin{definition}
  % Conjunction - definition
  Consider sentences $p$ and $q$. We say that their \textbf{conjunction} $p\wedge q$ is true iff both of them are true. Usually conjunction of $p$ and $q$ is
  referred as "$p$ and $q$".
\end{definition}

\begin{example}
  % Conjunction - simple example
  Sentence: "2+2=5" and "2+1=3" is false, as one of them (namely, the first one) is false.
\end{example}

\begin{exercise}
  % Conjunction - commutativity
  Let $p$ and $q$ be two sentences. Prove that $p\wedge q$ is true if and only if $q\wedge p$ is true. As we can swap two elements, we say that conjunction is \textbf{commutative}.
\end{exercise}

\begin{exercise}
  % Conjunction - associativity
  Let $p,\, q,\, r$ be three sentences. Prove that $(p\wedge q)\wedge r$ is true if and only if $p\wedge (q\wedge r)$ is true. Such a property is called \textbf{associativity}
  and implies that we do not need to specify the order of calculation. Therefore we can write just $p\wedge q\wedge r$.
\end{exercise}

\begin{definition}
  % Disjunction - definition
  Consider sentences $p$ and $q$. We say that their \textbf{disjunction} $p\vee q$ is true if and only if at least one of them is true. Usually disjunction of $p$ and $q$ is
  referred as "$p$ or $q$".
\end{definition}

\begin{exercise}
  % Disjunction - associativity and commutativity
  Prove that disjunction is both associative and commutative.
\end{exercise}

\begin{definition}
  % Negation - definition
  \textbf{Negation} of $p$ is a sentence $\neg p$ such that $\neg p$ is true if and only if $p$ is false. Usually we refer to $\neg p$ as "not $p$".
\end{definition}

\begin{exercise}
  % Negation - alternative definition as an exercise
  Prove that if $\neg p$ is false if and only if $p$ is true.
\end{exercise}

% Proof strategy - truth table
Now we will stop and think about proof strategies. Sometimes there is an elegant way how to prove that two statements are equivalent (like in the proof of associativity of
conjunction, one can see that both sentences are true iff all three basic sentences are true), but in case of more complicated sentences, it may be hard to find it. Common
proof strategy is \textbf{truth table} approach: we list in a table all the values that each basis sentence can take and evaluate the value of final expression.
Then two sentences are equivalent iff they have the same truth tables.

\begin{example}
  % Truth table - conjunction
  Truth table for conjunction:\\
  \begin{center}
    \begin{tabular}{ c  c  c }
      $p$ & $q$ & $p\wedge q$ \\
      \hline
      t  &  t &        t     \\
      t  &  f &        f     \\
      f  &  t &        f     \\
      f  &  f &        f     \\
    \end{tabular}
  \end{center}
  where $t$ stands for "true" and $f$ stands for "false".
\end{example}

This is a very powerful approach, as it requires no clever tricks but a simple calculation. The only problem is the number of calculations, that grows very quickly with
the number of basic sentences!

\begin{exercise}
  % Truth table - 2^n rows.
  Assume that you have built a sentence using $n$ sentences: $p_1, p_2, \dots, p_n$. How many rows does the truth table contain?
\end{exercise}

\begin{exercise}
  % Conjunction and disjunction - distributivity
  Prove \textbf{distributivity}:
  \begin{enumerate}
    \item $(p\wedge q)\vee r \Leftrightarrow (p\vee r) \wedge (q\vee r)$
    \item $(p\vee q)\wedge r \Leftrightarrow (p\wedge r) \vee (q\wedge r)$
  \end{enumerate}
\end{exercise}

\begin{exercise}
  % De Morgan's laws
  Prove \textbf{De Morgan's laws}:
  \begin{enumerate}
    \item $\neg (p\wedge q) = (\neg p)\vee (\neg q)$
    \item $\neg (p\vee q) = (\neg p)\wedge (\neg q)$
  \end{enumerate}
\end{exercise}

\begin{definition}
  We say that $p$ \textbf{implies} $q$ (or that $q$ \textbf{is implied by} $p$) for a sentence $p\Rightarrow q$ that is false iff $p$ is false and $q$ is true.
  We can summarise it in a truth table:
  \begin{center}
    \begin{tabular}{ c  c  c }
      $p$ & $q$ & $p\Rightarrow q$ \\
      \hline
      t  &  t &        t     \\
      t  &  f &        f     \\
      f  &  t &        t     \\
      f  &  f &        t     \\
    \end{tabular}
  \end{center}
  As you can see, it's a strange behaviour - false implies everything!
\end{definition}

\begin{exercise}
  % Implication - expression in terms of negation and disjunction
  Prove that $(p\Rightarrow q)\Leftrightarrow (\neg p) \vee q$. Hint: left sentence is false for very specific $p$ and $q$. Do you need to write down 4 rows in a truth table for
  right-hand-side sentence?
\end{exercise}

\begin{exercise}
  % Implication - transitivity
  Prove that implication is \textbf{transitive}, that is $((p\Rightarrow q)\wedge (q\Rightarrow r)) \Rightarrow (p\Rightarrow r)$.
\end{exercise}

The following exercise will later be a foundation of an operation called orthocomplementation in more general settings:
\begin{exercise}
  % Negation as orthocomplementation - properties
  Let $p$ and $q$ be sentences. Prove that:
  \begin{enumerate}
    \item $\neg(\neg p) \Leftrightarrow p$
    \item $p\Rightarrow q$ implies $(\neg q)\Rightarrow (\neg p)$ (Be smart! How many values of $p, q$ do you need to check?)
    \item $p\vee (\neg p)$
    \item $p\wedge (\neg p)$ is \textit{false} (we could write "Prove $\neg(p\wedge (\neg p))$", but it looks much more terrible!)
  \end{enumerate}
\end{exercise}

You may have seen similarity between symbols $\Leftrightarrow$ and $\Rightarrow$ - it's not an accident as you can prove!
\begin{exercise}
  % Equivalence via implications
  Prove that $(p\Leftrightarrow q)\Leftrightarrow ((p\Rightarrow q) \wedge (q\Rightarrow p))$.
\end{exercise}

\subsection{Another point of view}
In mathematics we have usually many different views on the same thing. Some of them are suited better for some kind of problems, other to others.
We would like to introduce you to a useful model of propositional calculus. To each true sentence $p$ we assign number $v(p)$
that is 1 if $p$ is true and 0 if $p$ is false. We define $1+1=1$
(it's a bit unusual thing). Then:
\begin{enumerate}
  \item $a\Leftrightarrow b$ means the same thing as sentence $v(a)=v(b)$.
  \item $a\wedge b$ means exactly the same thing as $v(a)\cdot v(b)$
  \item $a\vee b$ means exactly the same thing as $v(a)+v(b)$ (this is the reason why we want $1+1=1$)
  \item $a\Rightarrow b$ is the same as $v(a)\le (b)$.
  \item $\neg 1 = 0$ and $\neg 0=1$
\end{enumerate}

\begin{exercise}
  % Transitivity
  Prove transitivity of implication, that is $((p\Rightarrow q)\wedge (q\Rightarrow r)) \Rightarrow (p\Rightarrow r)$ using transitivity of $\le$.
  It simplifies the proof a bit, doesn't it?
\end{exercise}

\subsection{Quantifiers}
Consider a sentence $P(n)$ involving an object $n$ (for example $n$ can be an integer and $P(n)$ can be a sentence $n=2n$).
\begin{definition}
  % Universal and existence quantifiers - definition
  We define \textbf{universal quantifier}
  as a sentence $\forall_n P(n)$ meaning "for all $n$, the formula $P(n)$ holds".
  We define \textbf{existential quantifier} as a sentence $\exists_n P(n)$ meaning "there exists $n$ such that $P(n)$ holds"
  \footnote{$\forall$ is a rotated "A" symbolising "for \textbf{A}ll" and $\exists$ is a rotated "E" symbolising "\textbf{E}xists"}.
\end{definition}

\begin{example}
  % Example showing the difference between quantifiers.
  In case of $P(n)$ meaning "$2n=n$", we $\forall_n P(n)$ is false (as for $n=1$ we have $2\cdot 1\neq 1$) but $\exists_n P(n)$ is true,
  as $2\cdot 0=0$.
\end{example}

Intuitively, it is a much simpler problem to give an example of an object with a special property, than proving that \emph{every} object has a property.
In the above example, we gave an example disproving the statement. It may be useful to convert between these quantifiers. As you can prove:

\begin{exercise}
  % Quantifiers - negation
  Prove that:
  \begin{enumerate}
    \item $\neg \forall_n P(n) \Leftrightarrow \exists_n \neg P(n)$
    \item $\neg \exists_n P(n) \Leftrightarrow \forall_n \neg P(n)$
  \end{enumerate}
\end{exercise}

\section{Basic set theory}
\label{sec:basic_set_theory}

In modern mathematics we do not define a set nor set membership, so heuristically you can think that set $A$
is a ,,collection of objects' and $x\in A$ means that the object $x$ is inside this collection.
We write $x\notin A$ as a shorthand for $\neg (x\in A)$

\begin{example}
  % Sets - example of their API using a library
  Consider a library with closed stack and with a webpage. You can check whether there is a specific book inside it -
  so you can know for example that "Alice's Adventures in Wonderland"
  is in the stack, but you don't know how many copies there are. Moreover you can't ask about place of the books - there is no concept as being "first" or "second" element,
  as we can't check the physical stack.
\end{example}

As we can discover, there are collections of objects that do not form a set:
\begin{prob}
  % Russel's paradox - no set of all sets
  \textbf{Russel's paradox}
  Let $X$ be a set built from all sets such that $A\notin A.$ Prove that $X$ does not exist. Hint: what if $X\in X$? What if $X\notin X$?
\end{prob}

Therefore we need to assume the existence of a few sets, and then construct new out of them. We assume that there exist:
\begin{enumerate}
  \item finite sets (like real libraries with finite number of books), they are written as $\{a_1,a_2,\dots,a_n\}$. Empty set is written as $\emptyset$ rather than $\{\}$.
	\item real numbers\footnote{You may feel a bit insecure - what are real numbers, integers and so on? We haven't defined them properly yet.
    We will defer the construction of them to later sections, as what really matters are they \textit{properties} that you learned in elementary school.} $\mathbb R$
	\item natural numbers $\mathbb N=\{0,1,2,\dots\}$
	\item integers $\mathbb Z$
	\item rational numbers $\mathbb Q$
\end{enumerate}

\begin{definition}
  % Equality of sets
  \textbf{Equality of sets} We say that two sets $A$, $B$ are \textbf{equal} iff they have the same elements, that is:
  $$A=B\Leftrightarrow \forall_x (x\in A \Leftrightarrow x\in B).$$
  Sometimes this definition is called the \textbf{axiom of extensionality}.
\end{definition}

\begin{definition}
  % Subset and superset definitions
  We say that $A$ \textbf{is a subset of} $B$ iff every element of $A$ is also in $B$, that is:
  $A\subseteq B \Leftrightarrow \forall_a a\in A\Rightarrow a\in B$.
  If $A$ is a subset of $B$, we also say that $B$ \textbf{is a superset} of $A$.
\end{definition}

% Improvement in quantifier notation
This is a good opportunity to slightly modify our quantifier notation - usually we will be interested in objects belonging to some sets.
Formula $$\forall_{a\in A} P(a)$$ means "for all $a\in A$, statement $P(a)$ holds"
and $$\exists_{a\in A} P(a)$$ means "there is $a\in A$ such that $P(a)$ holds".

\begin{example}
  % Improved quantifier notation in subset definition
  We can write $A\subseteq B \Leftrightarrow \forall_{a\in A} a\in B$.
\end{example}

\begin{exercise}
  % Equality using subsets
  Prove that $A=B$ iff $A$ is a subset of $B$ and $B$ is a subset of $A$.
\end{exercise}

\begin{exercise}
  % Empty set uniqueness property
  Here we will prove that the empty set is a unique set with special property of being a subset of every set:
  \begin{enumerate}
    \item Prove that for every set $A$, $\emptyset\subseteq A$.
    \item Let $\theta$ be a set such that $\theta \subseteq A$ for every set $A$. Prove that $\theta=\emptyset$.
  \end{enumerate}
\end{exercise}

\subsection{New sets from old}
At the moment we do not have many sets. Let's try to define some methods of creating new sets from the know ones:

\begin{definition}
  % Selecting elements
  \textbf{Axiom schema of specification} Consider a set $A$ and a statement that assigns a truth value $P(a)$ to each $a\in A$. We can select elements $a$
  for which formula $P(a)$ is true and create a set\footnote{Some authors write $\{a\in A\,|\,P(a)\}$}:
  $$\{a\in A : P(a)\}.$$
\end{definition}

\begin{example}
  % Example with empty set
  We assumed that real numbers $\R$ exist. We can construct the empty set using the axiom schema of specification:
  $\emptyset=\{r\in\R : r=r+1\}.$
\end{example}

Above axiom schema is important - using this we can prove that there is no set of all sets:
\begin{exercise}
  % No set of all sets
  Prove that there is \textit{no} set of all sets. Hint: assume there is one and select some elements to create Russel's paradox.
\end{exercise}

Although is is impossible to create the set of all sets, it is possible to create \textit{some} sets of sets.

\begin{definition}
  % Axiom of power set
  \textbf{Axiom of power set} Consider a set $A$. We assume that there exists
  \footnote{We cannot create it using the axiom schema of specification, as there is no set from which we could select subsets of $A$. But since now, we can do it.}
  \textbf{the power set of $A$} defined as a set of all subsets of $A$:
  $$\mathcal P(A) := 2^A := \{A' : A'\subseteq A\}.$$
\end{definition}

\begin{exercise}
  % Selection of subsets with special property
  Using the axiom of power set and the axiom schema of specification, justify the notation:
  $$\{A'\subseteq A : P(A')\},$$
  where $P(A')$ assigns true or false to each subset $A'$ of $A$.
\end{exercise}

\begin{exercise}
  % Number of elements in a power set
  \begin{enumerate}
    \item Let $A=\{1,2,3\}$. Find $2^A$. What is the number of elements in $\mathcal P(A)$? How is it related to the
      number of elements of $A$?
    \item Let $A$ be a finite set with $n$ elements. Prove that $\mathcal P(A)$ has $2^n$ elements.
      Do you see now why $\mathcal P(A)$ is also referenced as $2^A$?
      Hint: every subset is specified by elements that are inside it.
      For every element you have two options - to select it or not.
  \end{enumerate}
\end{exercise}

\begin{definition}
  % Definition of a family of sets
  By a \textbf{collection} of sets or \textbf{family of sets} we understand a set of some sets.
\end{definition}

\begin{definition}
  % Unions of sets
  \textbf{Axiom of union} Assume that we are given a family of sets $A$. There is a set called their \textbf{union}\footnote{Again, we cannot use the axiom schema of specification as there is no
  set of all everything - we would select set of sets it it existed.}:
  $$\bigcup \mathcal A = \{x : \exists_{X\in \mathcal A} : x\in X\}.$$
  If the family of sets is indexed by some index, as $\mathcal A = \{A_i : i\in I\}$, we sometimes will write:
  $$\bigcup_{i\in I} A_i := \bigcup \mathcal A.$$
\end{definition}

\begin{exercise}
  % Finite unions
  Let $A$, $B$ and $C$ be sets. Prove that:
  \begin{enumerate}
    \item union defined as $A\cup B=\{x : x\in A \vee x\in B\}$ agrees with $\bigcup \{A, B\}$
    \item $A\cup B = B\cup A$ (so union is commutative)
    \item $(A\cup B)\cup C = \bigcup \{A,B,C\}$
    \item $(A\cup B)\cup C = A\cup (B\cup C)$ (this is called associativity)
    \item $A\cup A=A$
  \end{enumerate}
\end{exercise}

\begin{definition}
  % Set difference - definition
  \textbf{Set difference} Let $A$ and $B$ be two sets. We define their \textbf{difference}:
  $$A\setminus B := A-B := \{a \in A : a\notin B\}$$
\end{definition}

\begin{exercise}
  % Using set difference and unions
  Let $A$ and $B$ be sets. Prove that $A\subseteq B\cup (A\setminus B)$, where the equality holds iff $B\subseteq A$.
\end{exercise}

\begin{definition}
  % Intersection of sets
  Consider a family of sets $\mathcal A$. We define their \textbf{intersection} as a set:
  $$\bigcap \mathcal A = \left\{x\in \bigcup \mathcal A : \forall_{X\in\mathcal A}\, x\in X\right\}.$$
  If the family of sets is indexed by some index, as $\mathcal A = \{A_i : i\in I\}$, we sometimes will write:
  $$\bigcap_{i\in I} A_i := \bigcap \mathcal A.$$
\end{definition}

\begin{exercise}
  % Easy exercise for finding an infinite intersection. Maybe too easy.
	Find sum and intersection of family of subsets of $\mathbb R$: $A_r=\{r, -r\}$ for $r\ge 0.$
\end{exercise}

\begin{exercise}
  % Properties of finite intersections
	Let $A,\,B\,C$ be sets. Writing $A\cap B := \bigcap \{A,B\}$, prove that:
	\begin{enumerate}
		\item $A\cap B=B\cap A$ (commutativity)
		\item $A\cap (B\cap C)=(A\cap B)\cap C$ (associativity)
    \item $A\cap A=A$
	\end{enumerate}
\end{exercise}

\begin{exercise}
  % Distributivity of intersection and union
  Prove distributivity:
  \begin{enumerate}
    \item $A\cap (B\cup C)=(A\cap B)\cup (A\cap C)$
    \item $A\cup (B\cap C)=(A\cup B)\cap (A\cup C)$
  \end{enumerate}
\end{exercise}

% \noindent Moreover, we will introduce two new symbols, called positive and negative infinity:
% $\infty$ and $-\infty$.
% These are \textit{not} real numbers, just symbols that are used to name a few useful sets:
%
% \begin{align*}
% 	(-\infty,b) &= \{x\in \mathbb R : x < b\}\\
% 	(-\infty,b] &= \{x\in \mathbb R : x \ge b\}\\
% 	(a,\infty)  &= \{x\in \mathbb R : a < x\}\\
% 	[a,\infty)  &= \{x\in \mathbb R : a \le x\}
% \end{align*}

\subsection{Subsets and complements}
\begin{definition}
  % Complement of a set
  Let $A$ be subset of $U$. We say that \textbf{the complement
  \footnote{Just adding an index $c$ is not the best symbol possible as we need to have $U$ in mind.}
  of $A$} is a set $A^c=U\setminus A$.
\end{definition}


\begin{prob}
	Prove the following set identites:
	\begin{enumerate}
		\item Let $A\subseteq B.$ Prove that $(A^c)^c = A$.
		\item Let $A,\, B\subset U$. Prove that $(A\cup B)^c = A^c\cap B^c$
		\item Let $A,\, B\subset U$. Prove that $(A\cap B)^c = A^c\cup B^c$
		\item $\{a\in A : a\in B\} = \{b\in B : b\in A\}$
	\end{enumerate}
\end{prob}

\begin{prob}
  % Useful lemma for cofinite topology
	Let $\mathcal X \subset \mathcal P(U)$ be a family of sets and define:
  $\mathcal Y=\{X^c : X\in \mathcal X\}$, where $X^c=U\setminus X$.
  Prove that:
  \begin{enumerate}
    \item $(\bigcup \mathcal X)^c = \bigcap \mathcal Y$
    \item $(\bigcap \mathcal X)^c = \bigcup \mathcal Y$
  \end{enumerate}
\end{prob}

\begin{exercise}
  % Useful lemma for classification of open sets using neighborhoods.
	Let $A\subseteq X_i$ for $i\in I$. Prove that
	$$A\subseteq \bigcup_{i\in I} X_i$$
\end{exercise}

\begin{exercise}
  % Useful lemma for classification of open sets using neighborhoods.
	For every point $a\in A$ there is a set $U_a\subseteq A$ such that $a\in U_a$.
	Prove that $$A=\bigcup_{a\in A} U_a.$$
\end{exercise}

\subsection{Axiom of choice}
% TODO - it is convenient to introduce it here

% END OF REFACTORING FOR NOW

% To zadanie wymaga pewnika Dedekinda/aksjomatu Euklidesa. Lepiej je zrobić później
% \begin{prob} Find infinite sum and intersection for the families of subsets of $\mathbb{R}$:
%	\begin{enumerate}
% 		\item $A_i=(0,1/i)$ for $i=1,2,\dots$
% 		\item $B_i=[0,1/i)$ for $i=1,2,\dots$
% 	\end{enumerate}
% \end{prob}

\subsection{Cartesian product}
First of all, we need a useful concept:
\begin{prob}
	Let $A=\{\{a\}, \{a,b\}\},\, B=\{\{c\},\{c,d\}\}$. Prove that $A=B$ iff $a=c\wedge b=d$. Such a set $A$ we call
	\textbf{the ordered pair} $(a,b)$ as it has the property $(a,b)=(c,d)$ iff $a=c$ and $b=d$.
	Now you can forget how it has been constructed, and just remember this property.
\end{prob}

\begin{prob}
Prove that $(a,(b,c))=(d,(e,f))$ iff $a=d\wedge b=e\wedge c=f$.
\end{prob}
\noindent Therefore it makes sense to write just
$(a,b,c)$ for $(a,(b,c))$ and define similarily such \textbf{ordered tuple} for four elements, five elements
and so on.
\begin{prob}
Check that defining $(a,b,c)$ as $((a,b),c)$ also works (so two ordered tuples are the same if they have the
same first element, the same second element, ...)
\end{prob}
\begin{prob}
Check that, in terms of sets, $(a,(b,c))\neq ((a,b),c)$, so formally we do need to stick to one convention.
However as we are interested in the property of ordered tuple, we will not distinguish them and denote both
of them just as $(a,b,c)$. Such notational problems appear in various places in mathematics, so we need to
try to get used to them.
\end{prob}

\noindent We can now introduce another way of creating new sets: let $A$ and $B$ be sets. Then we define their
\textbf{Cartesian product} as
$$A\times B = \{(a,b) : a\in A\wedge b\in B\}.$$

\begin{prob}
	Do you remember the identification of $(a,(b,c))$ and $((a,b),c)$? Prove that
	$A\times (B\times C) = (A\times B)\times C$. Therefore we'll write it just as $A\times B\times C$
	without parentheness.
\end{prob}

\noindent Commonly used notation is $X^2 = X\times X = \{(x,y) : x, y \in X\}$ and analogously for other powers.

\section{Natural numbers and mathematical induction}
\label{sec:mathematical_induction}
Have you ever seen falling dominoes? To be sure that every domino falls, we need to:
\begin{enumerate}
	\item punch the first domino
	\item for every domino we must be sure the implication: if this particular domino falls, the next one also falls
\end{enumerate}
And that's all, we can be sure that all the dominoes will eventually fall. This style of reasoning\footnote{We do not show here formally \textit{why}
this principle works. For curious, you define natural numbers in such way this principle works.} is called \textbf{mathematical induction} and
formally it is written as: if $0\in S$ and for every\footnote{I repeat: for every $n$ we need to prove the implication ,,if works for $n$, then
works for $n+1$". The correct way is to write ,,I assume that there is a given $n$ for which the formula works. I will prove that is works for $n+1$".
Common mistake is to write ,,I assume that the formula works for every $n$ and I will prove that it works for $n+1$.". As professor Wiktor Bartol says
- there is no need to prove the statement as you already assumed that it works in every case.}
 $n\in N$ you can prove the implication $n\in S$ then $n+1\in S$, you know that $N\subseteq S$.
\begin{prob}
	You can prove that $2^n>n$ for every natural number $n$.
	\begin{enumerate}
		\item Prove that the formula works for $n=0$ (punch the first domino).
		\item Assume that for some $n$ you proved on some way that $2^n>n$. Using this, prove that $2^{n+1}>(n+1)$ (if $n$-th domino falls, then
		$n+1$-th domino also falls)
	\end{enumerate}
\end{prob}

\noindent You can also modify slightly the induction principle - sometimes you should start with number different than 0 or use different induction step
(start 0 and step 2 can lead to theorems valid for even numbers, step 0 and steps 1 and -1 can lead to theorems valid for all integers...)
\begin{prob}
    \begin{enumerate}
	   \item Prove\footnote{Another method is to notice that $n^3-n=(n-1)\cdot n\cdot (n+1)$. Why 2 does divide it? Why 3?} that 6 divides
		     $n^3-n$ for all natural $n$.
	    \item Prove\footnote{How $n^3-n$ and $(-n)^3-(-n)$ are related? Does it simplify the proof?} that 6 divides $n^3-n$ for all integers $n$.
		      You can use a slight modification mathematical induction principle proving the implication
		      ,,if the theorem works for $n$, it works also for $n-1$".
    \end{enumerate}
\end{prob}

\begin{prob}
	(Bernoulli's inequality) Prove that for real $x > -1$ and natural $n\ge 1$, the following inequality holds:
	$$(1+x)^n\ge 1+nx.$$
\end{prob}

\begin{prob}
	In Mathsland there are $n\ge 2$ cities. Between each pair of them there is a \textit{one-way} road.
	\begin{enumerate}
		\item Prove that there is a city from which you can drive to all the other cities. Hint: assume that the hypothesis works for some $n$ and any
			country with $n$ cities. Now consider an arbitrary $n+1$-city country. Hide one city and use your assumption.
		\item Prove that there is a city\footnote{Nice trick: what does happen if you reverse each way? Can you use the former result?}
			to which you can drive from all the others.
	\end{enumerate}
\end{prob}

\begin{prob}
	Let $S\subseteq R$. We say that $S$ is \textbf{well-ordered} iff any non-empty subset $X\subset S$ has the smallest element.
	\begin{enumerate}
		\item Prove that reals and integers with the default ordering are not well-ordered.
		\item Assume that $X\subseteq \mathbb N$ doesn't have the smallest element. Define $A=\{n\in \mathbb N : \{0,1,\dots,n\}\cap X=\emptyset\}$
			and use mathematical induction to prove that $X$ is empty.
		\item Why are natural numbers well-ordered?
	\end{enumerate}
\end{prob}

\section{Functions}
\label{sec:intro_to_functions}

\subsection{Basics}
\noindent Consider two sets $A$ and $B$. We say that a subset $f\subseteq A\times B$ is a \textbf{function}
iff the following two conditions hold:
\begin{itemize}
	\item for every element $a\in A$ there is an element $b\in B$ such that $(a,b)\in f$
	\item if $(a,b)\in f$ and $(a,c)\in f$, then $b=c$
\end{itemize}
Therefore for each $a\in A$ there is exactly one $b\in B$ such that $(a,b)\in f$. Such $b$ will be called
\textbf{value of $f$ at point $a$} and given a symbol $f(a).$
\begin{prob}
	(Thanks to Antoni Hanke) How many are there functions from the empty set to $\{1,2,3,4\}?$
\end{prob}

We need to introduce more terminology: set $A$ is called \textbf{the domain of $f$}, set $B$ is called
\textbf{the codomain of $f$} and the function $f$ is written as $f: A\to B$.

\begin{prob}
	Consider two functions: $f:\{0, 1\}\to \{0,1\}$ given by $f(x)=0$ and $g:\{0,1\}\to\{0\}$.
	Prove that $f=g$.
	\footnote{Some mathematicians, as Bourbaki use an alternative definition of function - for them
	a function is the triple $(A,B,f)$, where $f$ is defined as in the our case. We see that this definition
	is incompatible with ours. Fortunately, as in the case with different definitions of ordered tuples,
	this problem will never occur explicitly in the further chapters.}
\end{prob}

\begin{prob}
	Let $f:A\to B$ and $g: C\to B,$ where $A\neq C$. Is it possible that $f=g$?
\end{prob}

\begin{prob}
	Let $f: A\to B$ and $C\subseteq D\subseteq A$. We define: $f[C] = \{b\in B : b=f(c) \text{ for some }c\in C \}$ and analogously $f[D]$. Prove that
	$f(C)\subseteq f(D).$
\end{prob}

\subsection{Injectivity, surjectivity and bijectivity}

\noindent As we have already seen, there may be some elements in codomain that are not values of
$f$. We define \textbf{the image of $f$} as:
$$\text{Im}\, f = \{b\in B : \text{there is } a\in A \text{ such that } b=f(a)\}.$$
We say that the function $f: A\to B$ is \textbf{surjective} (or \textbf{onto}) iff $\text{Im}\,f=B$.

\begin{prob}
	As we remember, $\mathbb{R}$ stands for well-known real numbers. Are the following functions surjective?
	\begin{enumerate}
		\item $f: \mathbb{R} \to \mathbb{R}, ~f(x)=x^3$
		\item $g: \mathbb{R} \to \mathbb{R}, ~g(x)=x^2$
		\item $h: \mathbb{R} \to \{5\}$
	\end{enumerate}
\end{prob}

If $f(a)$ uniquely specifies $a$ (if $f(a)=f(b)$, then $a=b$) we say that the function is \textbf{injective}
(or \textbf{one-to-one}).
\begin{prob}
	As we remember, $\mathbb{R}$ stands for well-known real numbers. Are the following functions injective?
	\begin{enumerate}
		\item $f: \mathbb{R} \to \mathbb R, ~f(x)=x^2$
		\item $h: \{0,1,2,3\} \to \mathbb R, ~h(x)=x$
	\end{enumerate}
\end{prob}

If a function $f$ is both surjective and injective, we say that is \textit{bijective}\footnote{If you prefer nouns: surjective function is called surjection, injective - injection
and bijective - bijection}.

\begin{prob}
	Construct function that is:
	\begin{enumerate}
		\item surjective, but not injective
		\item injective, but not surjective
		\item neither injective nor surjective
		\item bijective
	\end{enumerate}
\end{prob}

\noindent Notice that if a function $f: A\to B$ is bijective, then we can construct a function $g:B\to A$
such that $f(g(b))=b$ and $g(f(a))=a$.
\begin{prob}
	Prove that, if exists, $g$ is unique.
\end{prob}

\noindent We call this function \textbf{the inverse function}
\footnote{It becomes confusing when working on real numbers: $f^{-1}(x)$ is
\textbf{not} $(f(x))^{-1}=1/f(x)$}: $g=f^{-1}.$

\begin{prob}
	Assume that $f^{-1}$ exists. Prove that $(f^{-1})^{-1}$ exists and is equal to $f$.
\end{prob}

\subsection{Function composition}
If we have two functions: $f:A\to B$ and $g: B\to C$, we can construct the \textbf{composition} using formula:
$g\circ f: A\to C,~(g\circ f)(a) = g(f(a)).$

\begin{prob}
	Find functions $f,~g$ such that:
	\begin{enumerate}
		\item $g\circ f$ exists, but $f\circ g$ is not defined
		\item both $f\circ g$ and $g\circ f$ exist, but $f\circ g\neq g\circ f$
	\end{enumerate}
\end{prob}

\noindent Although function composition is not commutative, it is associative:
\begin{prob}
	Left $f:A\to B, g: B\to C, h: C\to D$. Prove that
	$$h\circ (g\circ f) = (h\circ g)\circ f.$$
\end{prob}
Therefore we can ommit the brackets and write just $h\circ g\circ f.$ We will use function composition very
often.

\begin{prob}
    \begin{enumerate}
	   \item Prove that composition of two surjections is surjective.
	   \item Prove that composition of two injections is injective.
	   \item Prove that composition of two bijections is bijective.
    \end{enumerate}
\end{prob}

\begin{prob} We will rephrase the definition of the inverse function as follows:
	\begin{enumerate}
		\item If $X$ if a set, we define \textbf{the identity function}
			$$\text{Id}_X=\{(x,x)\in X^2 : x\in X\}.$$
			Prove that it is indeed a function. What is it's domain?
		\item Let $f:A\to B,~g:B\to A$. Prove that $f=g^{-1}$ iff
			$$g\circ f = \text{Id}_A \text{ and } f\circ g = \text{Id}_B$$
	\end{enumerate}
\end{prob}

\begin{prob}
  Let $f: A\to B$ be an injection. Prove that there is a function
  $g: \text{Im\,} f \to A$ such that $g\circ f = \text{Id}_A.$
  Such $g$ is called \textbf{left inverse of $f$}.
\end{prob}

\section{Countability}
\subsection{Finite sets}
For a finite set $X$ we write the number of elements of $X$ as $|X|$. We can calculate their \textbf{cardinalities} (sizes, numbers of elements) with
ease,
\begin{prob}
	What is the cardinality of $\{a, a+1, a+2, \dots, a+n\}$?
\end{prob}
\begin{prob}
	Let $A,\,B$ and $C$ be finite sets. Prove that:
	\begin{enumerate}
		\item $|2^A|=2^{|A|}$
		\item $|A\cup B|=|A|+|B|$ iff $A$ and $B$ are disjoint.
		\item $|A\setminus B|=|A|-|B|$ if $B\subseteq A.$
		\item $|A| \ge |B|$ if $B\subseteq A$. When does the equality hold?
		\item $|A\cup B| = |A| + |B| - |A\cap B|$
		\item $|A\cup B\cup C| = |A|+|B|+|C| - |A\cap B| - |B\cap C|-|C\cap A| + |A\cap B\cap C|$
	\end{enumerate}
\end{prob}
We can also employ functions to compare cardinalities:
\begin{prob}
	Assume that $A$ and $B$ are finite sets. Prove that $|A|=|B|$ iff there is a bijection between $A$ and $B$.
\end{prob}
\begin{prob}
	Above we find the way of saying that two cardinalities are equal using existence of a bijection. Let's find a way to compare which is less using
	another kind of function.
	\begin{enumerate}
		\item Let $O_n=\{1,2,\dots,n\}.$ Prove that there is no injection from $O_{n+1}$ into $O_n$. Hint: use mathematical induction.
		\item Let $A$ and $B$ be finite. Prove that there is an injection from $A$ to $B$ iff $|A| \le |B|.$
	\end{enumerate}
\end{prob}
\begin{prob}
	Using the above results, prove in one line\footnote{The main step is $|A|\le B$ and $|B|\le |A|$, so $|A|=|B|.$} that if there is an injection from $A$ onto $B$ and an injection from $B$ into $A$, then there exists
	a bijection from $A$ onto $B$.
\end{prob}

\subsection{Infinite sets}
But how can we measure the number of elements of an infinite set, as $\mathbb N$ or $\mathbb R$? As natural numbers are ,,to small" we need to
introduce new numbers, as $|\mathbb N|$ and be able to compare them. As we have seen above, the existence of a bijection is a good way of saying
that two finite sets have equal cardinalities. It intuitively makes sense to employ this observation even in the infinite case: we say that sets
(finite or infinite) $A$ and $B$ have the same caridnalities (or $|A|=|B|$) iff there is a bijection between $A$ and $B$.

\begin{prob}
	Let $A,\,B$ and $C$ be sets. Prove that if $|A|=|B|$ and $|B|=|C|$, then $|A|=|C|.$ Hint: find the bijection between $A$ and $C$.
\end{prob}

\noindent Here you can see the difference between finite and infinite sets - for finite sets a proper subset (a subset that is not the whole set)
always has smaller number of elements. In the infinite case it is not true, as a proper subset can have \textit{the same} number of elements.

\begin{prob}
	Prove that:
	\begin{enumerate}
		\item $|\mathbb N| = |\mathbb Z|$.
		\item $|\mathbb N|=|\mathbb N\times \mathbb N|$.
		\item $|\mathbb N|=|\mathbb Q|$.
	\end{enumerate}
\end{prob}

\noindent Analogously to the finite case, we define $|A|\le |B|$ as the existence of an injection from $A$ to $B$. We say that $|A|<|B|$ iff there
is an injection from $A$ to $B$ but there is no bijection.

\begin{prob}
	Prove that if $A\subseteq B$, then $|A|\le|B|.$
\end{prob}

\begin{prob}
	Let $A,\,B$ and $C$ be sets. Prove that if $|A|\le |B|$ and $|B|\le |C|$, then $|A|\le |C|$.
\end{prob}

\begin{prob}
	Here you can prove that there are more real numbers than naturals or rationals. We define $X=\{x\in \mathbb R : 0\le x\le 1\}$ and choose one
	 convention of writing reals (e.g 0.999... = 1.000..., so we can choose to use nines)
	\begin{enumerate}
		\item Assume that you have written all the elements of $X$ in a single column. Can you find a real number that does not occur in the list?
		\item Using the above, prove that $|\mathbb N| < |X|$
		\item Prove that $|\mathbb Q| < |\mathbb R|.$
	\end{enumerate}
\end{prob}

\begin{prob}
	We know that $|\mathbb R| > |\mathbb N|.$ Using binary system prove that $\mathbb R=2^\mathbb N.$ Do you see similarity between the previous result
	and $2^n > n$ for natural $n$?
\end{prob}

\begin{prob}
	\textbf{Cantor's theorem} You will prove that $|A|<\left|2^A\right|$ for any set $A$. Let $A$ be a set and $f:A\to 2^A.$
	\begin{enumerate}
		\item Consider $X=\{a\in A : a\notin f(a)\}\in 2^A$. Is there $x\in A$ for which $f(x)=X?$
		\item Is $f$ surjective?
		\item Find an injective function $g: A\to 2^A.$
		\item Prove that $|A| < |2^A|$ for any set $A$.
		\item Use Cantor's theorem to prove that there is no set of all sets.
	\end{enumerate}
\end{prob}

\begin{prob}
	\textbf{Cantor-Schroeder-Bernstein theorem} Let's prove that if $|A|\le|B|$ and $|B|\le |A|$, then $|A|=|B|$ for any sets.
	\begin{enumerate}
		\item (Knaster-Tarski) Now assume that $F$ has \textit{monotonicity} property: $F(X)\subseteq F(Y)$ if $X\subseteq Y$.
			Prove that $F$ has a fixed point $S$ (that is $F(S)=S$), where:
			$$S=\bigcup_{X\in U} X, \text{~where~} U= \{Y\in 2^A : Y\subseteq f(Y)\}.$$
		\item (Banach) Let $f: A\to B$ and $g:B\to A$ be injections.
			We introduce new symbol: $f[X]=\{b\in B : b=f(x) \text{ for some } x\in X\}$. Prove that
			function $$F:2^A\to 2^A,~F(X)=A\setminus g[B\setminus f[X]]$$
			has the monotonicity property.
    \item Prove that $A\setminus S\subseteq \text{Im}\,g$, where
      $F$ and $S$ are taken from above.
		\item Prove that function
			$$h(x) =
				\begin{cases}
					f(x), x\in S\\
					g^{-1}(x), x \notin S
				\end{cases}
			 $$
			 is a bijection.
	\end{enumerate}
\end{prob}

\subsection{Pre-image of a function}
Let $f:A\to B$ and $C\subseteq A$. We used $f[C]$ for a set:
$$f[C] = \{ f(c) \in B : c\in C\},$$
but now we will abuse a bit our notation to stick to
the common nomenclature. Apparently, many mathematicians write:
$$f(C) = \{ f(c) \in B : c\in C\}.$$
This is not correct - as $f$ should take elements
$a\in A$ and returns elements $b\in B$, but here $f$ ,,takes"
a subset $C\subseteq A$ and returns a set $f(C)\subseteq B$. We will
follow this notation, but you should always check what meaning
the object feed to function has (whether it is an element or a subset).

\begin{prob}
  Let $f:A\to B$ and $X,Y\subseteq B$. Then:
  \begin{enumerate}
    \item $f(X\cup Y)=f(X)\cup f(Y)$
    \item $f(X\cap Y)\subseteq f(X)\cap f(Y)$
  \end{enumerate}
  You can also generalise this result to an arbitrary collection of
  sets.
\end{prob}

\noindent To even more abuse the notation, we will also give an additional meaning to $f^{-1}$. As we know, many functions $f$ \textit{don't} have inverses. But we will write for $D\subseteq B$:
$$f^{-1}(D) = \{a \in A : f(a)\in D\}\subseteq A.$$

We then say that $f^{-1}(D)$ is the \textbf{pre-image} of $D$.
To get accustomed with this notation, prove that:

\begin{prob}
  Let $f:A\to B$. Then $f(A)\subseteq B$ and $A=f^{-1}(B)$.
\end{prob}

You should also prove:
\begin{prob}
  Let $f:A\to B$ and $X,Y\subseteq B$. Then:
  \begin{enumerate}
    \item $f^{-1}(X\cup Y)=f^{-1}(X)\cup f^{-1}(Y)$
    \item $f^{-1}(X\cap Y)=f^{-1}(X)\cap f^{-1}(Y)$
  \end{enumerate}
  You can also generalise this result to an arbitrary collection of
  sets.
\end{prob}
Therefore, we see that although $f$ does \textit{not} preserve the
set structure, $f^{-1}$ does. This observation underlies the
notation of continuity.

\section{Real numbers}
At the beginning we assumed that you had some intuition what real numbers
are and how to work with them - to provide examples and make set theory less
abstract. But we have not treated them rigorously, as we did not have proper
glossary - it's high time we filled this gap and defined them properly.
It's high time we defined them properly, as we .
A \textbf{field} is a tuple $(F, +, \cdot, 1, 0).$ We have
many symbols there, let's explain what they mean:
\begin{itemize}
  \item $F$ is a set
  \item $+$ and $\cdot$ are functions from $F^2$ to $F$. We write
    $a+b$ for $+(a,b)$ and $a\cdot b$ for $\cdot (a,b)$.
  \item $1, 0\in F$ are just distinguished elements of $F$
\end{itemize}
We know what objects are in the definition, so we can talk about
properties they must have to form a field:

\begin{enumerate}
  \item $1\neq 0$ (so $F$ has at least two elements)
  \item $a+(b+c)=(a+b)+c$ for all $a,b,c$ (addition is associative)
  \item $a\cdot (b\cdot c)=(a\cdot b)\cdot c$ for all $a,b,c$ (multiplication is associative)
  \item $a+b=b+a$ for all $a, b$ (addition is commutative)
  \item $a\cdot b=b\cdot a$ for all $a,b$ (multiplication is commutative)
  \item $a+0=a$ for all $a$ (so 0 is neutral element of addition)
  \item $a\cdot 1=a$ for all $a$ (so 1 is neutral element of multiplication)
  \item for every $a$ there is $a'$ such that $a+a'=0$ (existence of
    an inverse element for addition)
  \item $a\cdot (b+c) = a\cdot b + a\cdot c$ for all $a,b, c$
    (multiplication distributesover addition)
  \item for every $a\neq 0$ there is $\tilde a$ such that $a\cdot \tilde a=1$
    (multiplication has an inverse element for all non-zero numbers)
\end{enumerate}

\begin{prob}
  Check that
  \begin{enumerate}
    \item real numbers understood informally, have the properties
      listed above
    \item rational numbers form a field
  \end{enumerate}
\end{prob}

From the above field axioms, you can derive many facts that may be obvious
to you:

\begin{prob}
  Prove that there is only one 0 and only one 1. Hint: assume that
  0 and 0' have property such that $a=a+0=a+0'$ and try $a=0$ and $a=0'$.
\end{prob}

\begin{prob}
  Prove that if $a+a'=0$ and $a+a''=0$, then $a'=a''$.
  Therefore we can introduce special symbol for \textit{the} additiv
  inverse: $a + (-a)=0$ and define subtraction as $a-b := a + (-b)$.
\end{prob}

\begin{prob}
  Prove that $-a=(-1)\cdot a$.
\end{prob}

% Consider more problems on this topic

As you see, many of the algebraic properties we are used to can be recovered from the axioms, but sometimes it can be complicated. Both real numbers and
rational numbers have also an order on them - for example $2>1$. It leads
to the definition of \textit{total order}. We call a pair $(F, \le)$ a \textit{totally ordered set} if for every $a,b\in F$ we have:
\begin{enumerate}
  \item $a\le b$ or $b\le a$ (we call this property totality)
  \item $a\le b$ and $b\le a$ imply $a=b$ (it's called antisymmetry)
  \item $a\le b$ and $b\le c$ imply $a\le c$ (transitivity)
\end{enumerate}
Having relation $\le$ we can define other: $b\ge a$ means that $a\le b$ and
$a<b$ means that $a\le b$ and $a\neq b$.

We say that tuple $(F, +, \cdot, 1, 0, \le)$ is \textbf{ordered field} if:
\begin{itemize}
  \item $(F, +, \cdot, 1, 0)$ is a field
  \item $(F, \le)$ is totally ordered
  \item $a\le b$ implies $a+c\le b+c$
  \item $0\le a$ and $0\le b$ imply that $0\le a\cdot b$
\end{itemize}

You can check that reals and rationals are ordered fields. These axioms give
us much more abilities, for example one is able to prove that $1>0$.
But we still have no difference in properties that distuingish rationals
from reals. This is called the completeness axiom and we will need a few
more definitions.

Consider $A\subseteq \mathbb R$. We say that $x$ is an
\textbf{upper bound} of $A$ iff $x\ge a$ for every $a\in A$.

\begin{prob}
  Prove that a set $A\subseteq \mathbb R$ can have no upper bounds
  or infinitely many of them.
\end{prob}

\noindent If an upper bound of $A$ exists, we say that $A$ is
\textbf{bounded from above}. Among them we will distinguish the
\textbf{supremum} (or \textbf{the least upper bound - l.u.b}):
$x=\sup A$ iff $x$ is an upper bound of $A$ and for any upper bound
$y$ of $A$ we have $x\le y$.

\begin{prob}
  Prove that supremum is unique, so if $x$ and $x'$ are supremums
  of $A$, then $x=x'$.
\end{prob}

\begin{prob}
  Prove that $x=\sup A$ if and only if
  $x\ge a$ for every $a\in A$ and for every $\eps > 0$ there is
  $a\in A$ such that $x < a + \eps$.
\end{prob}

\noindent Now we can state the \textbf{completeness axiom}:
each non-empty and bounded from above subset of real numbers has
a supremum.
This axiom allows us to prove many interesting things:

\begin{prob}
  Prove that natural numbers are \textit{not} bounded from above.
  Hint: if $n\in \mathbb N$, then $n+1\in \mathbb N$
\end{prob}

\begin{prob}
  Prove the \textbf{Archimedean axiom}\footnote{In fact we do not
  need to call it axiom, as we are able to prove it.}
  that for every $r\in R$, there is $n\in \mathbb N$ such that $n>r$.
\end{prob}

\begin{prob}
  Prove that for any $r>0$ there is $n\in \mathbb N$ such that
  $1/n < r$.
\end{prob}

\begin{prob}
  Prove that rational numbers do \textit{not} have the completeness
  property:
  \begin{enumerate}
    \item Let $p, q\in \mathbb Z\setminus \{0\}$.
      Prove that $p^2\neq 2q^2$.
    \item Prove that root of two, defined as
      $x > 0, x^2=2$ is not rational.
    \item Find a subset of $\mathbb Q$ that is bounded above, but
      has no rational supremum.
  \end{enumerate}
\end{prob}

\begin{prob}
  You should prove that in each nonempty interval there is at least
  one rational number:
  \begin{enumerate}
    \item Assume that $0<a<b$. Define
      $$A=\left\{\frac m N : m\in \mathbb N\right\},~
      \frac 1{b-a} < N \in \mathbb N$$
      and prove that $A\cap (a,b)$ is non-empty.
    \item Use the above result to prove that in \textit{each}
      interval there is at least one rational number.
    \item Prove that in each interval there are infinitely but countably many, rational numbers.
    \item Prove that in each interval there is an irrational number.
    \item How many irrational numbers are in each interval?
  \end{enumerate}
\end{prob}

\subsection{Absolute value}

Another concept that will be further useful is the \textbf{absolute value} of a real number:
if $x\in \mathbb R$ we write $|x|\in \mathbb R$ for:
$$|x| = \begin{cases}x &\text{ for } x \ge 0\\ -x &\text{ otherwise} \end{cases}.$$

\begin{prob}
  Prove that for every $x,y\in \mathbb R$:
  \begin{enumerate}
    \item $|x|=|-x|$
    \item if $|x|=|y|$ then $x=y$ or $x=-y$.
    \item $|x+y| \le |x| + |y|$ (this is called \textbf{triangle inequality})
    \item $|x-y|\le |x| + |y|$
    \item $\left||x| - |y|\right|\le |x-y|$ (this is sometimes calles \textbf{reverse triangle inequality})
  \end{enumerate}

\end{prob}
