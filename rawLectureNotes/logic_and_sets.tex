%%%%%%%%%%%%%%%%%%%%% chapter.tex %%%%%%%%%%%%%%%%%%%%%%%%%%%%%%%%%
%
% sample chapter
%
% Use this file as a template for your own input.
%
%%%%%%%%%%%%%%%%%%%%%%%% Springer-Verlag %%%%%%%%%%%%%%%%%%%%%%%%%%

\chapter{Logic and sets}
\label{logic_and_sets} % Always give a unique label
% use \chaptermark{}
% to alter or adjust the chapter heading in the running head

If you are already fimilar with operations on logical formulas and sets, you may ommit this chapter.

\section{Logical formulas}
\label{sec:logic}
	Consider declarative sentences as "Water boils at $100^\circ$C" or "2+2=5". We can construct new sentences from them using the following rules:
	\begin{enumerate}
		\item conjuntion (and): $p \wedge q$ is true if and only if $p$ is true and $q$ is true
		\item disjuntion (or): $p \vee q$ is true if and only if at least one of sentences $p,\, q$ is true
		\item implication: $p \Rightarrow q$ is false if and only if $p$ is true and $q$ is false. Intuitively,
			if you know that $p$ implies $q$ and $p$ is true, then $q$ also must be true
		\item negation (not): $\neg p$ is true if and only if $p$ is false 
		\item equivalence (iff, if and only if): $p\Leftrightarrow q$ means exactly 
			$(p\Rightarrow q)\wedge(q\Rightarrow p)$. Intuitively if you know that two sentences are equivalent 
			and one of them is true, the other is also true. Or if one of them is false, the other one is automatically false.
	\end{enumerate}
	
\noindent Because mathematics is the art of being smart and lazy, we will assign value 1 to true sentences and 0 to
false sentences.

\begin{prob}
Prove that the following sentences are true:
	\begin{enumerate}
		\item $\neg(\neg p) \Leftrightarrow p$
		\item $p\vee \neg p$
		\item $\neg (p\wedge q) = (\neg p)\vee (\neg q)$
		\item $\neg (p\vee q) = (\neg p)\wedge (\neg q)$
		\item $(p\Rightarrow q)\Leftrightarrow (\neg p) \vee q$
		\item $0\Rightarrow 1$
	\end{enumerate}
\end{prob}

\noindent Equipped with this powerful machinery we can dive into basic set theory.

\section{Basic set theory}
\label{sec:basic_set_theory}

\subsection{Rough ideas}
In modern mathematics we do not define a set nor set membership, so heuristically you can think that set $A$
is a 'collection of objects' and $x\in A$ means that the object $x$ is inside this collection. We will assume that any finite collections of elements $\{x_1, x_2, \dots, x_n\}$ is a set 
(the empty set is called $\varnothing$ rather than $\{\}$), moreover we will assume the existence of the following sets:
\begin{enumerate}
	\item real numbers $\mathbb R$
	\item natural numbers $\mathbb N=\{0,1,2,\dots\}$
	\item integers $\mathbb Z$
	\item rational numbers $\mathbb Q$
\end{enumerate}

\noindent We say that two sets are equal ($A=B$) iff they have the same elements 
($x\in A\Leftrightarrow x\in B$). Note, that we do not check how many times $x$ appears in $A$. We can
just say whether it inside or not.

\begin{prob}
Prove that $\{1,1,2,2,2\}=\{1,2\}$
\end{prob}

\noindent We assumed the existence of some sets at the beginning. Why? As you can prove not every 'collection of objects' is a set:

\begin{prob}
	Let $X$ be a set built from all sets such that $A\notin A.$ Prove that $X$ does not exist. Hint: what if $X\in X$? What if $X\notin X$?
\end{prob}

Therefore at the moment we do not have many sets that we assume to exist. Let's try to define some methods of creating new sets from the know ones:

\subsection{A few ways of constructing new sets}
\noindent Assume that $A$ and $B$ are sets:
\begin{enumerate}
	\item Let's make a formula $F$ that for every element $a\in A$, the value 
		$F(a)$ is true or false. We can then construct a set $S$ with all the elements $a$ from $A$ for which 
		the formula $F(a)$ holds. This set is written explicitly as $S=\{a\in A : F(a)\}$.
	\item We can form the sum of two sets: $a \in A\cup B$ iff $a\in A$ or $a\in B$.
	\item We can contruct the intersection of two sets: $a\in A\cap B$ iff $a\in A$ and $a \in B$.
	\item We can construct the difference of two sets: $A\setminus B = \{a \in A : a\notin B\}$
\end{enumerate}

\begin{prob}
	Let $A,\,B\,C$ be sets. Prove that: 
	\begin{enumerate}
		\item $A\cup A=A$
		\item $A\cup B=B\cup A$
		\item $A\cup (B\cup C)=(A\cup B)\cup C$
		\item $A\cap A=A$
		\item $A\cap B=B\cap A$
		\item $A\cap (B\cap C)=(A\cap B)\cap C$
		\item $A\cap (B\cup C)=(A\cap B)\cup (A\cap C)$
		\item $A\cup (B\cap C)=(A\cup B)\cap (A\cup C)$
	\end{enumerate}
\end{prob}

\begin{prob}
	Prove that there is no set of all sets.\\
	Hint: assume there is one. Then you can select some sets to form a set that does not exist.
\end{prob}

\noindent Moreover, we will introduce two new symbols, called positive and negative infinity: 
$\infty$ and $-\infty$. 
These are \textit{not} real numbers, just symbols that are used to name a few useful sets:

\begin{align*}
	(-\infty,b) &= \{x\in \mathbb R : x < b\}\\
	(-\infty,b] &= \{x\in \mathbb R : x \ge b\}\\
	(a,\infty)  &= \{x\in \mathbb R : a < x\}\\
	[a,\infty)  &= \{x\in \mathbb R : a \le x\}
\end{align*}

\subsection{Subsets and complements}
\noindent As we have some sets, we can try to compare them. We say that $A\subseteq B$ iff 
$a\in A\Rightarrow a\in B$ (or intuitively, each element of $A$ is also in $B$. We say that $A$ is 
a \textbf{subset} of $B$ or that $B$ is a \textbf{superset} of $A$. 

\begin{prob}
	Prove that $A=B$ iff $A\subseteq B \wedge B\subseteq A.$
\end{prob}

\noindent If we fix the set $B$, to each subset $A$
we can assign it's \textbf{complement}: $A^c=B\setminus A$. \footnote{It is not the best symbol possible as we need to have $B$ in mind.} 

\begin{prob}
	Prove the following set identites:
	\begin{enumerate}
		\item Let $A\subseteq B.$ Prove that $(A^c)^c = A$.
		\item Let $A,\, B\subset U$. Prove that $(A\cup B)^c = A^c\cap B^c$
		\item Let $A,\, B\subset U$. Prove that $(A\cap B)^c = A^c\cup B^c$
		\item $\{a\in A : a\in B\} = \{b\in B : b\in A\}$
	\end{enumerate}
\end{prob}

Moreover, we assume that for a set $A$ there exists it's \textbf{power set}:
$2^A = \{X : X\subseteq A\}$.

\begin{prob}
	\item Let $A=\{1,2,3\}$. Find $2^A$. What is the number of elements in $2^A$? How is it related to the
	number of elements of $A$?
	\item Let $A$ be a finite set with $n$ elements. Using the approach in which you choose which elements belong
		to a subset, prove that $2^A$ has $2^n$ elements.
\end{prob}

\subsection{Infinite collections of sets}
Now we understand how to construct new sets from finite number of sets. But we can also consider ,,more general" families
of sets, that are not necessarily finite:
let $A_i\subset U$ for $i\in I$, where $I$ is some indexing set. For example if $I=\{1,2,\dots,n\}$ we have a finite family. But you can imagine infinite families as $A_i = \{i\},\,i \in \mathbb{R}$. How do we define
the sum and intersection of them? We cannot sum them iteratively $A_1\cup A_2\cup ...$ as the process will never
end, so we need alternative definitions:
\begin{align*}
	\bigcup_{i\in I}A_i &= \{a\in U : a\in A_i \text{ for at least one }i\in I\}\\
	\bigcap_{i\in I}A_i &= \{a\in U : a\in A_i \text{ for every }i\in I\}
\end{align*}

\begin{prob}
	Prove that for finite $I$ these definitions agree with the previous.
\end{prob}

\begin{prob} Let $A_i\subseteq U,\, i\in I$ and 
	$$\sigma = \bigcup_{i\in I}A_i,\, \pi = \bigcap_{i\in I}A_i$$
	Prove that:
	\begin{enumerate}
		\item if $k\in I$, then $A_k\cup \sigma=\sigma$
		\item $\sigma\cap \pi = \pi$
	\end{enumerate}
\end{prob}

\begin{prob} Find infinite sum and intersection for the families of subsets of $\mathbb{R}$:
	\begin{enumerate}
		\item $A_i=(0,1/i)$ for $i=1,2,\dots$
		\item $B_i=[0,1/i)$ for $i=1,2,\dots$
	\end{enumerate}
\end{prob}

\subsection{Cartesian product}
First of all, we need a useful concept:
\begin{prob}
Let $A=\{a, \{a,b\}\},\, B=\{c,\{c,d\}\}$. Prove that $A=B$ iff $a=c\wedge b=d$. Such a set $A$ we call
\textbf{the ordered pair} $(a,b)$ as it has the property $(a,b)=(c,d)$ iff $a=c$ and $b=d$. 
Now you can forget how it has been constructed, and just remember this property.
\end{prob}

\begin{prob}
Prove that $(a,(b,c))=(d,(e,f))$ iff $a=d\wedge b=e\wedge c=f$.
\end{prob}
\noindent Therefore it makes sense to write just
$(a,b,c)$ for $(a,(b,c))$ and define similarily such \textbf{ordered tuple} for four elements, five elements
and so on.
\begin{prob}
Check that defining $(a,b,c)$ as $((a,b),c)$ also works (so two ordered tuples are the same if they have the
same first element, the same second element, ...)
\end{prob}
\begin{prob}
Check that, in terms of sets, $(a,(b,c))\neq ((a,b),c)$, so formally we do need to stick to one convention. 
However as we are interested in the property of ordered tuple, we will not distinguish them and denote both
of them just as $(a,b,c)$. Such notational problems appear in various places in mathematics, so we need to
try to get used to them.
\end{prob}

\noindent We can now introduce anot
her way of creating new sets: let $A$ and $B$ be sets. Then we define their
\textbf{Cartesian product} as 
$$A\times B = \{(a,b) : a\in A\wedge b\in B\}.$$

\begin{prob}
	Do you remember the identification of $(a,(b,c))$ and $((a,b),c)$? Prove that
	$A\times (B\times C) = (A\times B)\times C$. Therefore we'll write it just as $A\times B\times C$ 
	without parentheness.
\end{prob}

\noindent Commonly used notation is $X^2 = X\times X = \{(x,y) : x, y \in X\}$ and analogously for other powers.

\section{Natural numbers and mathematical induction}
\label{sec:mathematical_induction}
Have you ever seen falling dominoes? To be sure that every domino falls, we need to:
\begin{enumerate}
	\item punch the first domino
	\item for every domino we must be sure the implication: if this particular domino falls, the next one also falls
\end{enumerate}
And that's all, we can be sure that all the dominoes will eventually fall. This style of reasoning\footnote{We do not show here formally \textit{why}
this principle works. For curious, you define natural numbers in such way this principle works.} is called \textbf{mathematical induction} and
formally it is written as: if $0\in S$ and for every\footnote{I repeat: for every $n$ we need to prove the implication ,,if works for $n$, then
works for $n+1$". The correct way is to write "I assume that there is a given $n$ for which the formula works. I will prove that is works for $n+1$".
Common mistake is to write "I assume that the formula works for every $n$ and I will prove that it works for $n+1$.". As professor Wiktor Bartol says 
- there is no need to prove the statement as you already assumed that it works in every case.}
 $n\in N$ you can prove the implication $n\in S$ then $n+1\in S$, you know that $N\subseteq S$.
\begin{prob}
	You can prove that $2^n>n$ for every natural number $n$. 
	\begin{enumerate}
		\item Prove that the formula works for $n=0$ (punch the first domino).
		\item Assume that for some $n$ you proved on some way that $2^n>n$. Using this, prove that $2^{n+1}>(n+1)$ (if $n$-th domino falls, then
		$n+1$-th domino also falls)
	\end{enumerate}
\end{prob}

\noindent You can also modify slightly the induction principle - sometimes you should start with number different than 0 or use different induction step
(start 0 and step 2 can lead to theorems valid for even numbers, step 0 and steps 1 and -1 can lead to theorems valid for all integers...) 
\begin{prob}
	\item Prove\footnote{Another method is to notice that $n^3-n=(n-1)\cdot n\cdot (n+1)$. Why 2 does divide it? Why 3?} that 6 divides 
		$n^3-n$ for all natural $n$.
	\item Prove\footnote{How $n^3-n$ and $(-n)^3-(-n)$ are related? Does it simplify the proof?} that 6 divides $n^3-n$ for all integers $n$. 
		You can use a slight modification mathematical induction principle proving the implication 
		,,if the theorem works for $n$, it works also for $n-1$".  
\end{prob}

\begin{prob}
	(Bernoulli's inequality) Prove that for real $x > -1$ and natural $n\ge 1$, the following inequality holds:
	$$(1+x)^n\ge 1+nx.$$
\end{prob}

\begin{prob}
	In Mathsland there are $n\ge 2$ cities. Between each pair of them there is a one-way road. 
	\begin{enumerate}
		\item Prove that there is a city from which you can drive to all the other cities.
		\item Prove that there is a city\footnote{Nice trick: what does happen if you reverse each way? Can you use the former result?} 
			to which you can drive from all the others.
	\end{enumerate}
\end{prob}

\begin{prob}
	Let $S\subseteq R$. We say that $S$ is \textbf{well-ordered} iff any non-empty subset $X\subset S$ has the smallest element.
	\begin{enumerate}
		\item Prove that reals and integers are not well-ordered.
		\item Assume that $X\subseteq \mathbb N$ doesn't have the smallest element. Define $A=\{n\in \mathbb N : \{0,1,\dots,n\}\cap X=\emptyset\}$
			and use mathematical induction to prove that $X$ is empty.
		\item Why are natural numbers well-ordered? 
	\end{enumerate}
\end{prob}

\section{Functions}
\label{sec:intro_to_functions}

\subsection{Basics}
\noindent Consider two sets $A$ and $B$. We say that a subset $f\subseteq A\times B$ is a \textbf{function}
iff the following two conditions hold:
\begin{itemize}
	\item for every element $a\in A$ there is an element $b\in B$ such that $(a,b)\in f$
	\item if $(a,b)\in f$ and $(a,c)\in f$, then $b=c$
\end{itemize}
Therefore for each $a\in A$ there is exactly one $b\in B$ such that $(a,b)\in f$. Such $b$ will be called
\textbf{value of $f$ at point $a$} and given a symbol $f(a).$
\begin{prob}
	(Thanks to Antoni Hanke) How many are there functions from the empty set to $\{1,2,3,4\}?$
\end{prob}

We need to introduce more terminology: set $A$ is called \textbf{the domain of $f$}, set $B$ is called
\textbf{the codomain of $f$} and the function $f$ is written as $f: A\to B$.

\begin{prob}
	Consider two functions: $f:\{0, 1\}\to \{0,1\}$ given by $f(x)=0$ and $g:\{0,1\}\to\{0\}$.
	Prove that $f=g$.
	\footnote{Some mathematicians, as Bourbaki use an alternative definition of function - for them
	a function is the triple $(A,B,f)$, where $f$ is defined as in the our case. We see that this definition
	is incompatible with ours. Fortunately, as in the case with different definitions of ordered tuples, 
	this problem will never occur explicitly in the further chapters.}
\end{prob}

\begin{prob}
	Let $f:A\to B$ and $g: C\to B,$ where $A\neq C$. Is it possible that $f=g$?
\end{prob}

\subsection{Injectivity, surjectivity and bijectivity}

\noindent As we have already seen, there may be some elements in codomain that are not values of 
$f$. We define \textbf{the image of $f$} as:
$$\text{Im}\, f = \{b\in B : \text{there is } a\in A \text{ such that } b=f(a)\}.$$
We say that the function $f: A\to B$ is \textbf{surjective} (or \textbf{onto}) iff $\text{Im}\,f=B$.

\begin{prob}
	As we remember, $\mathbb{R}$ stands for well-known real numbers. Are the following functions surjective?
	\begin{enumerate}
		\item $f: \mathbb{R} \to \mathbb{R}, ~f(x)=x^3$
		\item $g: \mathbb{R} \to \mathbb{R}, ~g(x)=x^2$
		\item $h: \mathbb{R} \to \{5\}$
	\end{enumerate}
\end{prob}

If $f(a)$ uniquely specifies $a$ (if $f(a)=f(b)$, then $a=b$) we say that the function is \textbf{injective}
(or \textbf{one-to-one}).
\begin{prob}
	As we remember, $\mathbb{R}$ stands for well-known real numbers. Are the following functions injective?
	\begin{enumerate}
		\item $f: \mathbb{R} \to \mathbb R, ~f(x)=x^2$
		\item $h: \{0,1,2,3\} \to \mathbb R, ~h(x)=x$
	\end{enumerate}
\end{prob}

If a function $f$ is both surjective and injective, we say that is \textit{bijective}\footnote{If you prefer nouns: surjective function is called surjection, injective - injection
and bijective - bijection}.

\begin{prob}
	Construct function that is:
	\begin{enumerate}
		\item surjective, but not injective
		\item injective, but not surjective
		\item neither injective nor surjective
		\item bijective
	\end{enumerate}
\end{prob}

\noindent Notice that if a function $f: A\to B$ is bijective, then we can construct a function $g:B\to A$ 
such that $f(g(b))=b$ and $g(f(a))=a$.
\begin{prob}
	Prove that, if exists, $g$ is unique.
\end{prob}

\noindent We call this function \textbf{the inverse function}
\footnote{It becomes confusing when working on real numbers: $f^{-1}(x)$ is 
\textbf{not} $(f(x))^{-1}=1/f(x)$}: $g=f^{-1}.$

\begin{prob}
	Assume that $f^{-1}$ exists. Prove that $(f^{-1})^{-1}$ exists and is equal to $f$.
\end{prob}

\subsection{Function composition}
If we have two functions: $f:A\to B$ and $g: B\to C$, we can construct the \textbf{composition} using formula:
$g\circ f: A\to C,~(g\circ f)(a) = g(f(a)).$

\begin{prob}
	Find functions $f,~g$ such that:
	\begin{enumerate}
		\item $g\circ f$ exists, but $f\circ g$ is not defined
		\item both $f\circ g$ and $g\circ f$ exist, but $f\circ g\neq g\circ f$
	\end{enumerate}
\end{prob}

\noindent Although function composition is not commutative, it is associative:
\begin{prob}
	Left $f:A\to B, g: B\to C, h: C\to D$. Prove that
	$$h\circ (g\circ f) = (h\circ g)\circ f.$$
\end{prob}
Therefore we can ommit the brackets and write just $h\circ g\circ f.$ We will use function composition very
often.

\begin{prob}
	\item Prove that composition of two surjections is surjective.
	\item Prove that composition of two injections is injective.
	\item Prove that composition of two bijections is bijective.
\end{prob}

\begin{prob} We will rephrase the definition of the inverse function as follows:
	\begin{enumerate}
		\item If $X$ if a set, we define \textbf{the identity function} 
			$$\text{Id}_X=\{(x,x)\in X^2 : x\in X\}.$$
			Prove that it is indeed a function. What is it's domain?
		\item Let $f:A\to B,~g:B\to A$. Prove that $f=g^{-1}$ iff
			$$g\circ f = \text{Id}_A \text{ and } f\circ g = \text{Id}_B$$
	\end{enumerate}
\end{prob}

\section{Countability}
\subsection{Finite sets}
For a finite set $X$ we write the number of elements of $X$ as $|X|$. We can calculate their \textbf{cardinalities} (sizes, numbers of elements) with
ease,
\begin{prob}
	What is the cardinality of $\{a, a+1, a+2, \dots, a+n\}$?
\end{prob}
\begin{prob}
	Let $A,\,B$ and $C$ be finite sets. Prove that:
	\begin{enumerate}
		\item $|2^A|=2^{|A|}$
		\item $|A\cup B|=|A|+|B|$ iff $A$ and $B$ are disjoint.
		\item $|A\setminus B|=|A|-|B|$ if $B\subseteq A.$
		\item $|A| \ge |B|$ if $B\subseteq A$. When does the equality hold?
		\item $|A\cup B| = |A| + |B| - |A\cap B|$
		\item $|A\cup B\cup C| = |A|+|B|+|C| - |A\cap B| - |B\cap C|-|C\cap A| + |A\cap B\cap C|$
	\end{enumerate}
\end{prob}
We can also employ functions to compare cardinalities:
\begin{prob}
	Assume that $A$ and $B$ are finite sets. Prove that $|A|=|B|$ iff there is a bijection between $A$ and $B$.
\end{prob}
\begin{prob}
	Above we find the way of saying that two cardinalities are equal using existence of a bijection. Let's find a way to compare which is less using
	another kind of function.  
	\begin{enumerate}
		\item Let $O_n=\{1,2,\dots,n\}.$ Prove that there is no injection from $O_{n+1}$ into $O_n$. Hint: use mathematical induction.
		\item Let $A$ and $B$ be finite. Prove that there is an injection from $A$ to $B$ iff $|A| \le |B|.$ 
	\end{enumerate}
\end{prob}
\begin{prob}
	Using the above results, prove in one line\footnote{The main step is $|A|\le B$ and $|B|\le |A|$, so $|A|=|B|.$} that if there is an injection from $A$ onto $B$ and an injection from $B$ into $A$, then there exists
	a bijection from $A$ onto $B$.
\end{prob}

\subsection{Infinite sets}
But how can we measure the number of elements of an infinite set, as $\mathbb N$ or $\mathbb R$? As natural numbers are ,,to small" we need to
introduce new numbers, as $|\mathbb N|$ and be able to compare them. As we have seen above, the existence of a bijection is a good way of saying
that two finite sets have equal cardinalities. It intuitively makes sense to employ this observation even in the infinite case: we say that sets
(finite or infinite) $A$ and $B$ have the same caridnalities (or $|A|=|B|$) iff there is a bijection between $A$ and $B$.

\begin{prob}
	Let $A,\,B$ and $C$ be sets. Prove that if $|A|=|B|$ and $|B|=|C|$, then $|A|=|C|.$ Hint: find the bijection between $A$ and $C$.
\end{prob}

\noindent Here you can see the difference between finite and infinite sets - for finite sets a proper subset (a subset that is not the whole set)
always has smaller number of elements. In the infinite case it is not true, as a proper subset can have \textit{the same} number of elements.

\begin{prob}
	Prove that:
	\begin{enumerate}
		\item $|\mathbb N| = |\mathbb Z|$.
		\item $|\mathbb N|=|\mathbb N\times \mathbb N|$.
		\item $|\mathbb N|=|\mathbb Q|$.
	\end{enumerate}
\end{prob}

\noindent Analogously to the finite case, we define $|A|\le |B|$ as the existence of an injection from $A$ to $B$. We say that $|A|<|B|$ iff there
is an injection from $A$ to $B$ but there is no bijection.

\begin{prob}
	Prove that if $A\subseteq B$, then $|A|\le|B|.$
\end{prob}

\begin{prob}
	Let $A,\,B$ and $C$ be sets. Prove that if $|A|\le |B|$ and $|B|\le |C|$, then $|A|\le |C|$.
\end{prob}

\begin{prob}
	Here you can prove that there are more real numbers than naturals or rationals. We define $X=\{x\in \mathbb R : 0\le x\le 1\}$ and choose one
	 convention of writing reals (e.g 0.999... = 1.000..., so we can choose to use nines)   
	\begin{enumerate}
		\item Assume that you have written all the elements of $X$ in a single column. Can you find a real number that does not occur in the list? 
		\item Using the above, prove that $|\mathbb N| < |X|$
		\item Prove that $|\mathbb Q| < |\mathbb R|.$
	\end{enumerate}
\end{prob}

\begin{prob}
	We know that $|\mathbb R| > |\mathbb N|.$ Using binary system prove that $\mathbb R=2^\mathbb N.$ Do you see similarity between the previous result
	and $2^n > n$ for natural $n$?
\end{prob}

\begin{prob}
	\textbf{Cantor's theorem} You will prove that $|A|<\left|2^A\right|$ for any set $A$. Let $A$ be a set and $f:A\to 2^A.$
	\begin{enumerate}
		\item Consider $X=\{a\in A : a\notin f(a)\}\in 2^A$. Is there $x\in A$ for which $f(x)=X?$
		\item Is $f$ surjective? 
		\item Find an injective function $g: A\to 2^A.$
		\item Prove that $|A| < |2^A|$ for any set\footnote{The theorem works for finite sets as well, so we have an alternative proof of
			$2^n>n$ for natural $n$} $A$.
		\item Use Cantor's theorem to prove that there is no set of all sets.
	\end{enumerate}
\end{prob}

SKOŃCZYĆ - TODO TODO

\begin{prob}
	\textbf{Cantor-Schroeder-Bernstein theorem} Let's prove that if $|A|\le|B|$ and $|B|\le |A|$, then $|A|=|B|$ for any sets.
	\begin{enumerate}
		\item (Tarski) Now assume that $F$ has property: $F(X)\subseteq F(Y)$ if $X\subseteq Y$. Prove that $F(S)=S$, where:
			$$S=\bigcup_{X\in U} X, \text{~where~} U= \{Y\in 2^A : Y\subseteq f(Y)\}.$$
			Hint: if you know that $s\in S$, there must be $X_s\in U$ such that $s\in X_s.$
		\item (Banach) Let $f: A\to B$ and $g:B\to A$. Consider set
	\end{enumerate}
\end{prob}

\subsection{Pre-image of a function}

