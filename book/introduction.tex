\chapter{Introduction}
\label{introduction}

\section{Why another book?}

There are many excellent books on mathematical physics and differential geometry, so a question araises - how does this book differ from any other?
We have met many people with strong interest in Mathematics and Mathematical Physics without the necessary resources - great teachers, books or community. Some of them were holding a PhD from another discipline, some of them were undergraduates or even highschool students. Some of them got discouraged by too-high level of books they could afford and some of them didn't know where to start.

This book is supposed to provide an introduction to selected mathematical topics (basic algebra, topology and differential geometry) giving background necessary for more advanced books and showing how this is related to Physics (e.g. differential forms in classical electromagnetism, classical mechanics, general relativity, gauge theories).

\section{What this book aims to be}
I aimed at the following:
\begin{itemize}
  \item Understandable for any person wanting to learn. We assume very minimal prerequsities and try to develop most the concepts from scratch.
  \item Purpose-driven. We will try to provide the reasoning (often heuristical) and intuitions behind the definitions.
  \item Concepts are not introduced until there is a good example illustrating them.
  \item Problem-solving approach. I want you to prove most of the theorems in this book, with adjustable amount of hints. This way you can understand what we are actually doing, instead of ommiting proofs that look discouraging at the beginning.
  \item As self-containing as possible. Some lengthy proofs will be ommitted if they are not crucial for understanding, but in general we will try not to point you to another place.
  \item Exploring many points of view.
  \item No-jump policy. We do not refer to theorems as "Theorem 3.2 from page 231", but rather focus on their meaning.
  \item Notation abusements explained. Mathematics has been evolving for centuries in many different countries, so the notation
    is rather diverse and sometimes is not the best possible. We will abuse it as it is a standard in mathematical world, but you will always understand
    what objects are involved in expressions you are manipulating with.
  \item Free. Gathering an own mathematical library is pricey and many people don't have the access to university-level libraries. This book is supposed to be available for free and for anyone.
\end{itemize}

\section{What this book is not}
\begin{itemize}
  \item An equivalent of a Mathematics degree. Mathematics is much bigger than this. Many topics (e.g. probability, stochastic processes, classical geometry and modern analysis and modern results in general) are not included.
  \item A good example of written English. Nevertheless, I hope that this book is understandable.
  \item A good place to start learning Mathematics from very beginning. Before starting this book you should be able to carry simple proofs and understand basic combinatorics. A good indicator whether you can start this book is the test of prerequsities.
\end{itemize}

\section{Test of prerequisities}
These are a few problems that are relatively easy compared to the level of this book. If you find them hard, don't worry and spend more time with mathematics of this kind. A good place to start are problems from Junior Mathematical Olympiads.
\begin{enumerate}
  \item Find the sum $1+2+3+\dots+200$.
  \item In the farm there are goats and chickens. There are 28 heads and 70 legs. How many goats and chickens are there?
  \item Solve the equation $x^5-x^2=0$.
  \item Prove that $22^2-1$ is divisible by $3$, $7$ and $23$.
  \item There are three kinds of bread, four kinds of butter and seven different fillings. How many different sandwiches (each sandwich is made out from one kind of bread, one kind of butter and one filling) can you compose?
\end{enumerate}

\section{Now you're ready, good luck!}
We use the following notation: \textbf{bold} will be used for definitions of new objects, and \textit{italics} will be used for additional subtle remarks that should be taken into
account. We use footnote\footnote{Like this one.} to provide additional comments.

Remember that the subject is big and it may be very hard to finish the book in just one day. I strongly advise working on it every day starting from just two minutes a day
and increasing the time spend every week. I tried to make the learning curve flat, what leghtens the book.
Any mistakes are my own failure and I would be grateful if you pointed them out to me. Also any suggestions and comments are welcome. You can
create new issues on GitHub: \texttt{https://github.com/pawel-czyz/MathematicalPhysics} or write an email to \texttt{pczyz@protonmail.com}.
Good luck on your road!
