\subsection{Categories}
As you have had training in sets and functions, we are able to introduce some category theory that will quickly become useful - most of the objects in mathematics form a
category and this language will be extremely convenient to find similarities between different branches of mathematics.
As we remember, it is not possible to create a set of all sets without having a contradiction. So let's use a word \textbf{collection} of sets or \textbf{class}
\footnote{Formal treatment of classes - collections that are in some sense bigger than sets - is introduced in von Neumann–Bernays–Gödel and Morse-Kelley set theories.}
 of sets - that is not a set and we don't know how to express it formally - but what has an intuitive sense.

\begin{definition}
  A \textbf{category} is:
  \begin{enumerate}
    \item a collection of objects such that
    \item for each pair objects $A$, $B$ in the collection there is a \emph{set} $\Hom(A,B)$ called the set of \textbf{morphisms} or \textbf{maps} of \textbf{arrows} such that
    \item for morphisms $f\in \Hom(A,B), g\in \Hom(B,C)$ there is a morphism $g\circ f\in \Hom(A,C)$. We also require:
      \begin{itemize}
        \item that composition of morphisms is associative: $h\circ(g\circ f)=(h\circ g)\circ f$, where $h\in \Hom(C, D)$
        \item we have morphisms $\Id_X: X\to X$ for every object $X$ such that $f\circ\Id_A = f=\Id_B\circ f$ for $f\in \Hom(A,B)$
      \end{itemize}
  \end{enumerate}
  If $f\in \Hom(A,B)$, we can also write $f:A\to B$ or $A\xrightarrow{f} B$. Some authors also write $\text{Mor}(A,B)$ for $\Hom(A,B)$ and $gf$ for $g\circ f$. If the collection of objects happens
  to be a set, we call it \textbf{small category}.
\end{definition}

\begin{example}
  We already know very well a category - the category of sets and functions. Let's check carefully that is actually is a category:
  \begin{enumerate}
    \item objects are just sets
    \item take $\Hom(A,B)$ as a set of all functions from $A$ to $B$ (why is it a set?)
    \item define composition of morphisms just as function composition
      \begin{itemize}
        \item composition of functions is associative (recall why)
        \item identity morhpism is just identity function of a set
      \end{itemize}
  \end{enumerate}
\end{example}

\begin{exercise}
  Consider a category with one singleton: $\{\{0\}\}$, where $\{0\}$ is the only object, and functions as arrows. How many arrows are in this category?
\end{exercise}

\begin{exercise}
  Consider a category with two singletons: $\{\{0\}, \{1\}\}$, and functions as arrows. How many arrows can be in this category? Hint: 4 numbers.
\end{exercise}

\subsection{Morphisms and commutative diagrams}
