% !TEX root = book.tex

\section{The Axiom of Choice}
We formulated comparision of cardinalities in terms of injections. We based on the following exercise:

\begin{exercise}
  Let $f$ be a function from $A$ to $B$. Prove that there exists a function $g: \Image B\to A$ such that $g\circ f=\Id_A$ iff $f$ is injective.
\end{exercise}

That is for an injective function there exists a "left inverse". We may ask a question - is a some kind of inverse possible for \emph{surjections}?

\begin{exercise}
  Consider a surjective function $f: \Z \to \{0,1\}$ given by $2k+1 \mapsto 1, 2k\mapsto 0, k\in \Z$.
  \begin{enumerate}
    \item why a \emph{left} inverse does not exist?
    \item define a \emph{right} inverse, that is a function $g:\{0,1\}\to \Z$ such that $f\circ g=\Id_{\{0,1\}}$
  \end{enumerate}
\end{exercise}

In the above exercise we had no problem - just pick an element from the set of odd numbers (these that are mapped to 1) and an element from the set of even numbers (these that are mapped to 0). While there is no problem of picking an element from each set if we have just two (or three, four - any finite number), this issue may apear for \emph{infinite} families
of sets.

\begin{definition}
  \textbf{Axiom of choice (AC)} Let $\mathcal A$ be a non-empty family of non-empty sets. Then there exists a \textbf{choice function} $f:\mathcal A\to \bigcup \mathcal A$ such that
  $f(A)\in A$ for every $A\in \mathcal A$.
\end{definition}

Basically it means that for every family of sets, we can select an element from each set - for a set $A$, such element is just $f(A)$, where $f$ is the choice function. Alternatively,
we could formulate it equivalently as:

\begin{definition}
  \textbf{Axiom of choice (AC)} Let $\mathcal S=\{S_i: i\in I\}$ be any family of non-empty sets such that $S_i\cap S_j=\emptyset$ for $i\neq j$. Then it is possible to create a set $C$ such that for every $i\in I$ there is $s_i\in C$ such that $s_i\in S_i$. Or in natural-language terms: from every set of a family of nen-empty, pairwise-disjoint sets, we can select exactly one element.
\end{definition}

This axiom allows us to construct right inverses:

\begin{exercise}
  Prove that AC (the axiom of choice) is equivalent to the statement that every surjection possesses a right inverse. Hint: for $AC\Rightarrow \text{right inverse}$ use the same idea as in the previous problem. For
  $\text{right inverse}\Rightarrow AC$ construct a surjective function from $\bigcup \mathcal S\to \mathcal S$, where $\mathcal S$ is a family of non-empty, pairwise-disjoint sets.
\end{exercise}

\begin{exercise}
  Prove, assuming AC, that if $f:A\to B$ is a surjection, then, there exists an injection $g: B\to A$.
\end{exercise}

Therefore with AC it makes sense to compare cardinalities using surjections:

\begin{exercise}
% Cardinalities with surjections
  Prove, assuming AC, that:
  \begin{enumerate}
    \item $A\le B$ iff there exists a surjection from $B$ to $A$
    \item if there is a surjection from $A$ to $B$ and a surjection from $B$ to $A$, then there exists a bijection between $A$ and $B$
  \end{enumerate}
\end{exercise}

It can also be useful in problems involving infinitely many hats:

\begin{exercise}
  % Hats and AC
  A king said $\aleph_0$ mathematicians the following:
  "Tomorrow, you will be standing in a long queue and my servants will place a red or green hat on everyone's head. You will see only the hats of the people standing before you.
  On a given signal, you need to guess your own hat. If infinitely many of you guess wrong, I will send you to the prison for the rest of your lifes!".
  By considering a set of all functions from $\N\to \{"red", "green"\}$ and a suitable partition on it, prove, assuming the axiom of choice, that mathematicians can make finitely-many
  wrong guesses.
\end{exercise}

In fact, AC implies much more - as Banach-Tarski paradox says using it one can take a solid sphere, cut it into a few pieces and compose \emph{two} spheres of the same size, just by moving the pieces around. Therefore many mathematicians try to avoid it as much as possible - it is a good habit always to explicitly mention it's usage. In many places in this book we will use AC, usually in an equivalent form known as Kuratowski-Zorn lemma\footnote{In English literature it is widely known as \textbf{Zorn's lemma}. Kazimierz Kuratowski proved this lemma (although with an unnecessary assumption) in 1922 and Max Zorn, working independently, gave the above formulation in 1935. The Bourbaki group and John Tukey used the latter name in their books published in 1939 and 1940 and since then "Zorn's lemma" is widely recognised.}.

\subsection{Kuratowski-Zorn (Zorn's) lemma}
\begin{definition}
  A \textbf{partial order} is a relation $\le$ on a set $A$ such that for all $a,\,b,\,c\in A$:
  \begin{enumerate}
    \item $a\le a$
    \item $a\le b \wedge b\le a\Rightarrow a=b$
    \item $a\le b\wedge b\le c\Rightarrow a\le c$.
  \end{enumerate}
  If for every $a,\,b\in A$ we have $a\le b$ or $b\le a$, then we say that it is a \textbf{total order} or \textbf{linear order}.
\end{definition}

\begin{example}
  Natural numbers, integers and reals are totally ordered.
\end{example}

\begin{example}
  Consider a set $\mathcal P(A)$ for some set $A$. It's partially ordered by the relation:
  $$B\le C \Leftrightarrow B\subseteq C.$$
  Note that some sets cannot be compared (neither $A\le B$ nor $B\le A$), so this order is \textit{not} total.
\end{example}

\begin{definition}
  A \textbf{partially-ordered set} or a \textbf{poset} is a pair $(A, \le)$, where $A$ is a set and $\le$ is a partial order on $A$. If $B\subseteq A$ is a subset on which
  $\le$ is total (every two elements of $B$ can be compared, or in set-theoretic terms $B\times B\subseteq \le$), we call $B$ a \textbf{a chain}.
\end{definition}

\begin{example}
  Consider $A=\{0,1\}$. Then it's power set ordered by inclusion - $(\mathcal P(A), \subseteq)$ - is a poset. If we take $B=\{\emptyset,A\}\subseteq \mathcal P(A)$,
  then every two elements of $B$ can be compared - it's a chain.
\end{example}

\begin{definition}
  Let $(A, \le)$ be a poset and $B\subseteq A$ be a chain. We say that $u\in A$ is an \textbf{upper bound} of a chain $B$ if $b\le u$ for every $b\in B$.
  We say that $m\in A$ is a \textbf{maximal element} if for every $a\in A$ we have $m\le a\Rightarrow m=a$, that is there is no greater element than $m$.
\end{definition}

\begin{example}
  Let $A=\{1,2,3,4,5\}$ with standard order. Then 5 is a maximal element in $A$ and an upper bound of $A$.
\end{example}

\begin{theorem}
  \textbf{Kuratowski-Zorn (Zorn's) lemma}
  Let $(P, \le)$ be a poset such that every chain in $P$ has an upper bound. Then there exists a maximal element in $P$.
\end{theorem}

For a proof, you can check Arjun Jain's "Zorn’s Lemma An elementary proof under the Axiom of Choice"\footnote{https://arxiv.org/pdf/1207.6698.pdf}. We will usually use AC in this form.
