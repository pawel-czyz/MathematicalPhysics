% !TEX root = book.tex

\chapter{Propositional calculus and sets}
\label{chap:propositional_calculus_and_sets}
To be able to formulate and prove theorems, we need a language. In this chapter we learn propositional calculus and naive set theory, language in which most of the mathematics is expressed. Our treatment will not be exhaustive in any ways.
% !TEX root = book.tex

\section{Propositional calculus}
\subsection{New sentences from old}
\label{sec:logic}
Consider declarative sentences as "It's raining in Oxford now." or "2+2=5" that can be either true or false. There are many ways how to construct new sentences and decide
whether they are true or not.

\begin{definition}
  % Equivalence - definition
  Consider sentences $p$ and $q$. We say that they \textbf{are equivalent} (we write then $p\Leftrightarrow q$) if they are either true or false simultaneously.
  If $p$ and $q$ are equivalent, we usually say "$p$ if and only if $q$" of even "$p$ iff $q$".
\end{definition}

\begin{example}
  % Equivalence - example, people in a room
  Let $p$ be a sentence "The number of people in the room you are sitting is odd" and $q$ be "If one person enters the room, then the number of people will become even".
  We \textit{do not know} if any of these sentences is true - it would require to count all the people. But if $p$ is true, then also $q$ must be true and vice versa - if $q$ is
  true, then also $p$ must be true. Therefore we can say that $p$ and $q$ are equivalent, or write $p\Leftrightarrow q$.
\end{example}

\begin{exercise}
  % Equivalence - transitivity
  Prove that if we know that $p\Leftrightarrow q$ and $q\Leftrightarrow r$, then also $p\Leftrightarrow r$.
\end{exercise}

\begin{definition}
  % Conjunction - definition
  Consider sentences $p$ and $q$. We say that their \textbf{conjunction} $p\wedge q$ is true iff both of them are true. Usually conjunction of $p$ and $q$ is
  referred as "$p$ and $q$".
\end{definition}

\begin{example}
  % Conjunction - simple example
  Sentence: "2+2=5" and "2+1=3" is false, as one of them (namely, the first one) is false.
\end{example}

\begin{exercise}
  % Conjunction - commutativity
  Let $p$ and $q$ be two sentences. Prove that $p\wedge q$ is true if and only if $q\wedge p$ is true. As we can swap two elements, we say that conjunction is \textbf{commutative}.
\end{exercise}

\begin{exercise}
  % Conjunction - associativity
  Let $p,\, q,\, r$ be three sentences. Prove that $(p\wedge q)\wedge r$ is true if and only if $p\wedge (q\wedge r)$ is true. Such a property is called \textbf{associativity}
  and implies that we do not need to specify the order of calculation. Therefore we can write just $p\wedge q\wedge r$.
\end{exercise}

\begin{definition}
  % Disjunction - definition
  Consider sentences $p$ and $q$. We say that their \textbf{disjunction} $p\vee q$ is true if and only if at least one of them is true. Usually disjunction of $p$ and $q$ is
  referred as "$p$ or $q$".
\end{definition}

\begin{exercise}
  % Disjunction - associativity and commutativity
  Prove that disjunction is both associative and commutative.
\end{exercise}

\begin{definition}
  % Negation - definition
  \textbf{Negation} of $p$ is a sentence $\neg p$ such that $\neg p$ is true if and only if $p$ is false. Usually we refer to $\neg p$ as "not $p$".
\end{definition}

\begin{exercise}
  % Negation - alternative definition as an exercise
  Prove that if $\neg p$ is false if and only if $p$ is true.
\end{exercise}

% Proof strategy - truth table
Now we will stop and think about proof strategies. Sometimes there is an elegant way how to prove that two statements are equivalent (like in the proof of associativity of
conjunction, one can see that both sentences are true iff all three basic sentences are true), but in case of more complicated sentences, it may be hard to find it. Common
proof strategy is \textbf{truth table} approach: we list in a table all the values that each basis sentence can take and evaluate the value of final expression.
Then two sentences are equivalent iff they have the same truth tables.

\begin{example}
  % Truth table - conjunction
  Truth table for conjunction:\\
  \begin{center}
    \begin{tabular}{ c  c  c }
      $p$ & $q$ & $p\wedge q$ \\
      \hline
      t  &  t &        t     \\
      t  &  f &        f     \\
      f  &  t &        f     \\
      f  &  f &        f     \\
    \end{tabular}
  \end{center}
  where $t$ stands for "true" and $f$ stands for "false".
\end{example}

This is a very powerful approach, as it requires no clever tricks but a simple calculation. The only problem is the number of calculations, that grows very quickly with
the number of basic sentences!

\begin{exercise}
  % Truth table - 2^n rows.
  Assume that you have built a sentence using $n$ sentences: $p_1, p_2, \dots, p_n$. How many rows does the truth table contain?
\end{exercise}

\begin{exercise}
  % Conjunction and disjunction - distributivity
  Prove \textbf{distributivity}:
  \begin{enumerate}
    \item $(p\wedge q)\vee r \Leftrightarrow (p\vee r) \wedge (q\vee r)$
    \item $(p\vee q)\wedge r \Leftrightarrow (p\wedge r) \vee (q\wedge r)$
  \end{enumerate}
\end{exercise}

\begin{exercise}
  % De Morgan's laws
  Prove \textbf{De Morgan's laws}:
  \begin{enumerate}
    \item $\neg (p\wedge q) = (\neg p)\vee (\neg q)$
    \item $\neg (p\vee q) = (\neg p)\wedge (\neg q)$
  \end{enumerate}
\end{exercise}

\begin{definition}
  We say that $p$ \textbf{implies} $q$ (or that $q$ \textbf{is implied by} $p$) for a sentence $p\Rightarrow q$ that is false iff $p$ is false and $q$ is true.
  We can summarise it in a truth table:
  \begin{center}
    \begin{tabular}{ c  c  c }
      $p$ & $q$ & $p\Rightarrow q$ \\
      \hline
      t  &  t &        t     \\
      t  &  f &        f     \\
      f  &  t &        t     \\
      f  &  f &        t     \\
    \end{tabular}
  \end{center}
  As you can see, it's a strange behaviour - false implies everything!
\end{definition}

\begin{exercise}
  % Implication - expression in terms of negation and disjunction
  Prove that $(p\Rightarrow q)\Leftrightarrow (\neg p) \vee q$. Hint: left sentence is false for very specific $p$ and $q$. Do you need to write down 4 rows in a truth table for
  right-hand-side sentence?
\end{exercise}

\begin{exercise}
  % Implication - transitivity
  Prove that implication is \textbf{transitive}, that is $((p\Rightarrow q)\wedge (q\Rightarrow r)) \Rightarrow (p\Rightarrow r)$.
\end{exercise}

The following exercise will later be a foundation of an operation called orthocomplementation in more general settings:
\begin{exercise}
  % Negation as orthocomplementation - properties
  Let $p$ and $q$ be sentences. Prove that:
  \begin{enumerate}
    \item $\neg(\neg p) \Leftrightarrow p$
    \item $p\Rightarrow q$ implies $(\neg q)\Rightarrow (\neg p)$ (Be smart! How many values of $p, q$ do you need to check?)
    \item $p\vee (\neg p)$
    \item $p\wedge (\neg p)$ is \textit{false} (we could write "Prove $\neg(p\wedge (\neg p))$", but it looks much more terrible!)
  \end{enumerate}
\end{exercise}

You may have seen similarity between symbols $\Leftrightarrow$ and $\Rightarrow$ - it's not an accident as you can prove!
\begin{exercise}
  % Equivalence via implications
  Prove that $(p\Leftrightarrow q)\Leftrightarrow ((p\Rightarrow q) \wedge (q\Rightarrow p))$.
\end{exercise}

\subsection{Another point of view}
In mathematics we have usually many different views on the same thing. Some of them are suited better for some kind of problems, other to others.
We would like to introduce you to a useful model of propositional calculus. To each true sentence $p$ we assign number $v(p)$
that is 1 if $p$ is true and 0 if $p$ is false. We define $1+1=1$
(it's a bit unusual thing). Then:
\begin{enumerate}
  \item $a\Leftrightarrow b$ means the same thing as sentence $v(a)=v(b)$.
  \item $a\wedge b$ means exactly the same thing as $v(a)\cdot v(b)$
  \item $a\vee b$ means exactly the same thing as $v(a)+v(b)$ (this is the reason why we want $1+1=1$)
  \item $a\Rightarrow b$ is the same as $v(a)\le (b)$.
  \item $\neg 1 = 0$ and $\neg 0=1$
\end{enumerate}

\begin{exercise}
  % Transitivity
  Prove transitivity of implication, that is $((p\Rightarrow q)\wedge (q\Rightarrow r)) \Rightarrow (p\Rightarrow r)$ using transitivity of $\le$.
  It simplifies the proof a bit, doesn't it?
\end{exercise}

\subsection{Quantifiers}
Consider a sentence $P(n)$ involving an object $n$ (for example $n$ can be an integer and $P(n)$ can be a sentence $n=2n$).
\begin{definition}
  % Universal and existence quantifiers - definition
  We define \textbf{universal quantifier}
  as a sentence $\forall_n P(n)$ meaning "for all $n$, the formula $P(n)$ holds".
  We define \textbf{existential quantifier} as a sentence $\exists_n P(n)$ meaning "there exists $n$ such that $P(n)$ holds"
  \footnote{$\forall$ is a rotated "A" symbolising "for \textbf{A}ll" and $\exists$ is a rotated "E" symbolising "\textbf{E}xists"}.
\end{definition}

\begin{example}
  % Example showing the difference between quantifiers.
  In case of $P(n)$ meaning "$2n=n$", we $\forall_n P(n)$ is false (as for $n=1$ we have $2\cdot 1\neq 1$) but $\exists_n P(n)$ is true,
  as $2\cdot 0=0$.
\end{example}

Intuitively, it is a much simpler problem to give an example of an object with a special property, than proving that \emph{every} object has a property.
In the above example, we gave an example disproving the statement. It may be useful to convert between these quantifiers. As you can prove:

\begin{exercise}
  % Quantifiers - negation
  Prove that:
  \begin{enumerate}
    \item $\neg \forall_n P(n) \Leftrightarrow \exists_n \neg P(n)$
    \item $\neg \exists_n P(n) \Leftrightarrow \forall_n \neg P(n)$
  \end{enumerate}
\end{exercise}

\subsection{Mathematical induction}
In this section we practice our abilities on the \emph{mathematical induction principle}. Although very simple, this method appears in many proofs in mathematics.
Before we define this, we need a simple definition to provide us with many examples:

\begin{definition}
  If $a\neq 0$ and $b$ are integers, we say that \textbf{$a$ divides $b$} iff there exists $k\in \Z$ such that $b=ak$. We can also write this as $a|b$ or say that \textbf{$b$ is divisible by $a$}.
\end{definition}

\begin{example}
  $2\,|\,84$ as $84=2\cdot 42$.
\end{example}

\begin{exercise}
  Let $0\neq k\in \Z$ divide $a$ and $b$. Prove that:
  \begin{enumerate}
    \item $k\,|\,a+b$,
    \item $k\,|\,a-b$,
    \item $k^2\,|\,ab$.
  \end{enumerate}
\end{exercise}

\begin{theorem}
  \textbf{Mathematical induction principle} Let $P(n)$ be a sentence about a natural number $n$. If:
  \begin{enumerate}
    \item $P(0)$ is true, and
    \item for every natural $k$ the implication $P(k)\Rightarrow P(k+1)$ is true,
  \end{enumerate}
  then $P(n)$ is true for all natural numbers $n$.
\end{theorem}

This can be visualised with a row of dominoes. To be sure that all of them eventually fall:
\begin{enumerate}
  \item hit the first domino
  \item the dominoes are set in such manner that if $k$th domino falls, then it hits the $(k+1)$th.
\end{enumerate}

\begin{example}
  We'll prove that $2\,|\,n(n+1)$ for every $n\in \N$.
  \begin{enumerate}
    \item The statement is true for $0$ as $2\,|\,0\cdot (0+1)$,
    \item I need to prove that $(2\,|\,n(n+1))\Rightarrow (2\,|\,(n+1)(n+2))$ for every $n$.
  \end{enumerate}
  Assume that $n$ is such a number that $2\,|\,n(n+1)$.
  Then $$(n+1)(n+2)=n(n+1) + 2\cdot (n+1)$$ is divisible by 2 as well.

  Using the principle of mathematical induction we see that for every natural $n$, the number $n(n+1)$ is divisible\footnote{There is also an alternative proof: it's
  a product of two consecutive numbers - one of them is divisible by 2 and so is the product.} by 2.
\end{example}

\begin{exercise}
	Prove that $2^n>n$ for every natural number $n$.
\end{exercise}

\noindent You can also modify slightly the induction principle - sometimes you should start with number different than 0 or use different induction step
(start 0 and step 2 can lead to theorems valid for even numbers, step 0 and steps 1 and -1 can lead to theorems valid for all integers...)
\begin{exercise}
    \begin{enumerate}
	   \item Prove\footnote{Another method is to notice that $n^3-n=(n-1)\cdot n\cdot (n+1)$. Why 2 does divide it? Why 3?} that 6 divides
		     $n^3-n$ for all natural $n$.
	    \item Prove\footnote{How $n^3-n$ and $(-n)^3-(-n)$ are related? Does this simplify the proof?} that 6 divides $n^3-n$ for all integers $n$.
		      You can use a slight modification mathematical induction principle proving the implication
		      ,,if the theorem works for $n$, it works also for $n-1$".
    \end{enumerate}
\end{exercise}

\begin{exercise}
	(Bernoulli's inequality) Prove that for every real $x > -1$ and every natural $n\ge 1$, the following inequality holds:
	$$(1+x)^n\ge 1+nx.$$
\end{exercise}

\begin{exercise}
	In a country there are $n\ge 2$ cities. Between each pair of them there is a \textit{one-way} road.
	\begin{enumerate}
		\item Prove that there is a city from which you can drive to all the other cities. [Hint\footnote{Assume that the hypothesis works for some $n$ and any
			country with $n$ cities. Now consider an arbitrary $n+1$-city country. Hide one city and use your assumption.}]
		\item Prove that there is a city to which you can drive from all the others. [Hint\footnote{Nice trick: what does happen if you reverse each way? Can you use the former result?}].
	\end{enumerate}
\end{exercise}

\section{Basic set theory}
\label{sec:basic_set_theory}

In modern mathematics we do not define a set nor set membership, but rather believe that there exists objects with properties that
are listed in this chapter.

Heuristically you can think that a set $A$ is a "collection of objects" and a sentence "$x\in A$" means that the object $x$ is inside this collection. We read this as
"$x$ belongs to set $A$" or "$x$ is an element of $A$". We write $x\notin A$ as a shorthand for $\neg (x\in A)$ (and it means that $x$ is \textit{not} an element of $A$).

\begin{example}
  % Sets - example of set API using a library
  Consider a library with closed stack and with a webpage. You can check whether there is a specific book inside it -
  so you can know for example that "Alice's Adventures in Wonderland"
  is in the stack, but you don't know how many copies there are. Moreover you can't ask about place of the books - there is no concept as being "first" or "second" element,
  as we can't check the physical stack.
\end{example}

As we can discover, there are collections of objects that do not form a set:
\begin{prob}
  % Russel's paradox - no set of all sets
  \textbf{Russel's paradox}
  Let $X$ be a set built from all sets such that $A\notin A.$ Prove that $X$ does not exist. Hint: what if $X\in X$? What if $X\notin X$?
\end{prob}

Therefore we need to assume the existence of a few sets, and then construct new out of them using some rules in which we believe. We assume that there exist:
\begin{enumerate}
  \item finite sets (like real libraries with finite number of books). These are written as $\{a_1,a_2,\dots,a_n\}$. Empty set is written as $\emptyset$ rather than $\{\}$.
	\item real numbers\footnote{You may feel a bit insecure - what are real numbers, integers and so on? We haven't defined them properly yet.
    We will defer the construction of them to later sections, as what really matters are they \textit{properties} that you learned in elementary school.} $\mathbb R$
	\item natural numbers $\mathbb N=\{0,1,2,\dots\}$
	\item integers $\mathbb Z$
	\item rational numbers $\mathbb Q$
\end{enumerate}

\begin{definition}
  % Equality of sets
  \textbf{Axiom of extensionality (Equality of sets)} We say that two sets $A$, $B$ are \textbf{equal} iff they have the same elements, that is:
  $$A=B\Leftrightarrow \forall_x (x\in A \Leftrightarrow x\in B).$$
\end{definition}

\begin{definition}
  % Subset and superset definitions
  We say that $A$ \textbf{is a subset of} $B$ iff every element of $A$ is also in $B$, that is:
  $$A\subseteq B \Leftrightarrow \forall_a (a\in A\Rightarrow a\in B).$$
  If $A$ is a subset of $B$, we also say that $B$ \textbf{is a superset} of $A$.
\end{definition}

% Improvement in quantifier notation
This is a good opportunity to slightly modify our quantifier notation - usually we will be interested in objects belonging to some sets.
Formula $$\forall_{a\in A} P(a)$$ means "for all $a\in A$, statement $P(a)$ is true"
and $$\exists_{a\in A} P(a)$$ means "there is an $a\in A$ such that $P(a)$ holds".

\begin{example}
  % Improved quantifier notation in subset definition
  We can write $A\subseteq B \Leftrightarrow \forall_{a\in A} a\in B$.
\end{example}

\begin{exercise}
  % Equality using subsets
  Let $A$ and $B$ be two sets. Prove that $A=B$ iff $A$ is a subset of $B$ and $B$ is a subset of $A$.
\end{exercise}

\begin{exercise}
  % Empty set uniqueness property
  Here we will prove that the empty set is a unique set with special property of being a subset of every set:
  \begin{enumerate}
    \item Prove that for every set $A$, $\emptyset\subseteq A$.
    \item Let $\theta$ be a set such that $\theta \subseteq A$ for every set $A$. Prove that $\theta=\emptyset$.
  \end{enumerate}
\end{exercise}

\subsection{New sets from old}
At the moment we do not have many sets. Let's try to define some methods of creating new sets from the know ones:

\begin{definition}
  % Selecting elements
  \textbf{Axiom schema of specification} Consider a set $A$ and a statement that assigns a truth value $P(a)$ to each $a\in A$. We can select elements $a$
  for which formula $P(a)$ is true and create a set\footnote{Some authors write $\{a\in A\,|\,P(a)\}$}:
  $$\{a\in A : P(a)\}.$$
\end{definition}

\begin{example}
  % Example with empty set
  We assumed that the set $\R$ (of real numbers) exist. We can construct the empty set using the axiom schema of specification:
  $\emptyset=\{r\in\R : r=r+1\}.$
\end{example}

The above axiom schema of specification is important - using this we can prove that there is no set of all sets:
\begin{exercise}
  % No set of all sets
  Prove that there is \textit{no} set of all sets. Hint: assume there is one and select some elements to create Russel's paradox.
\end{exercise}

Although is is impossible to create the set of all sets, it is possible to create \textit{some} sets of sets.

\begin{definition}
  % Axiom of power set
  \textbf{Axiom of power set} Consider a set $A$. We assume that there exists
  \footnote{We cannot create it using the axiom schema of specification, as there is no set from which we could select subsets of $A$. But since now, we can do it.}
  \textbf{the power set of $A$} defined as a set of all subsets of $A$:
  $$\mathcal P(A) := 2^A := \{A' : A'\subseteq A\}.$$
  That is $A'\in \mathcal P(A)$ iff $A'\subseteq A$.
\end{definition}

\begin{exercise}
  % Selection of subsets with special property
  Using the axiom of power set and the axiom schema of specification, justify the notation:
  $$\{A'\subseteq A : P(A')\},$$
  where $P(A')$ assigns true or false to each subset $A'$ of $A$.
\end{exercise}

\begin{exercise}
  % Number of elements in a power set
  \begin{enumerate}
    \item Let $A=\{1,2,3\}$. Find it's power set $\mathcal P(A)$. What is the number of elements in $\mathcal P(A)$? How is it related to the
      number of elements of $A$?
    \item Let $A$ be a finite set with $n$ elements. Prove that $\mathcal P(A)$ has $2^n$ elements.
      Do you see now why $\mathcal P(A)$ is sometimes referenced as $2^A$?
      Hint: every subset is specified by elements that are inside it.
      For every element you have two options - to select it or not.
  \end{enumerate}
\end{exercise}

\begin{definition}
  % Definition of a family of sets
  By a \textbf{collection of sets} or \textbf{family of sets} we understand a set of some sets.
\end{definition}

\begin{definition}
  % Unions of sets
  \textbf{Axiom of union} Assume that we are given a family of sets $A$. There is a set called their \textbf{union}\footnote{Again, we cannot use the axiom schema of specification as there is no everything - we would select set of sets it it existed.}:
  $$\bigcup \mathcal A = \{x : \exists_{X\in \mathcal A} x\in X\}.$$
  If the family of sets is indexed by some index, that is: $\mathcal A = \{A_i : i\in I\}$, we can also write:
  $$\bigcup_{i\in I} A_i := \bigcup \mathcal A.$$
\end{definition}

\begin{exercise}
  % Finite unions
  Let $A$, $B$ and $C$ be sets. Prove that:
  \begin{enumerate}
    \item union defined as $A\cup B=\{x : x\in A \vee x\in B\}$ agrees with $\bigcup \{A, B\}$
    \item $A\cup B = B\cup A$ (so union is commutative)
    \item $(A\cup B)\cup C = \bigcup \{A,B,C\}$
    \item $(A\cup B)\cup C = A\cup (B\cup C)$ (this is called associativity)
    \item $A\cup A=A$
  \end{enumerate}
\end{exercise}

\begin{definition}
  % Set difference - definition
  \textbf{Set difference} Let $A$ and $B$ be two sets. We define their \textbf{difference}:
  $$A\setminus B := A-B := \{a \in A : a\notin B\}$$
\end{definition}

\begin{example}
  Let $A=\{1,2,3\}$ and $B=\{2,3,4\}$. Then $A\setminus B = \{1\}$.
\end{example}

\begin{exercise}
  Is $(A\setminus B) \cup B$ always equal to $A$?
\end{exercise}

\begin{exercise}
  % Using set difference and unions
  Let $A$ and $B$ be sets. Prove that $A\subseteq (A\setminus B)\cup B$, where the equality holds iff $B\subseteq A$.
\end{exercise}

\begin{definition}
  % Intersection of sets
  Consider a family of sets $\mathcal A$. We define their \textbf{intersection} as a set:
  $$\bigcap \mathcal A = \left\{x\in \bigcup \mathcal A : \forall_{X\in\mathcal A}\, x\in X\right\}.$$
  If the family of sets is indexed by some index, that is: $\mathcal A = \{A_i : i\in I\}$, we can write:
  $$\bigcap_{i\in I} A_i := \bigcap \mathcal A.$$
\end{definition}

\begin{exercise}
  % Easy exercise for finding an infinite intersection. Maybe too easy.
	Find sum and intersection of family of subsets of $\mathbb R$:
  $$A_r=\{r, -r\}$$ for $r\ge 0.$
\end{exercise}

\begin{exercise}
  % Properties of finite intersections
	Let $A,\,B\,C$ be sets. Writing $A\cap B := \bigcap \{A,B\}$, prove that:
	\begin{enumerate}
		\item $A\cap B=B\cap A$ (commutativity)
		\item $A\cap (B\cap C)=(A\cap B)\cap C$ (associativity)
    \item $A\cap A=A$
	\end{enumerate}
\end{exercise}

\begin{exercise}
  % Distributivity of intersection and union
  Prove distributivity:
  \begin{enumerate}
    \item $A\cap (B\cup C)=(A\cap B)\cup (A\cap C)$
    \item $A\cup (B\cap C)=(A\cup B)\cap (A\cup C)$
  \end{enumerate}
\end{exercise}

% \noindent Moreover, we will introduce two new symbols, called positive and negative infinity:
% $\infty$ and $-\infty$.
% These are \textit{not} real numbers, just symbols that are used to name a few useful sets:
%
% \begin{align*}
% 	(-\infty,b) &= \{x\in \mathbb R : x < b\}\\
% 	(-\infty,b] &= \{x\in \mathbb R : x \ge b\}\\
% 	(a,\infty)  &= \{x\in \mathbb R : a < x\}\\
% 	[a,\infty)  &= \{x\in \mathbb R : a \le x\}
% \end{align*}

\subsection{Subsets and complements}
\begin{definition}
  % Complement of a set
  Let $A$ be subset of a set $U$. We say that \textbf{the complement\footnote{Just adding an index $c$ is not the best symbol possible as we need to have $U$ in mind.}
  of $A$} is a set $A^c=U\setminus A$.
\end{definition}


\begin{prob}
	Prove the following set identites:
	\begin{enumerate}
		\item Let $A\subseteq U.$ Prove that $(A^c)^c = A$.
		\item Let $A,\, B\subset U$. Prove that $(A\cup B)^c = A^c\cap B^c$
		\item Let $A,\, B\subset U$. Prove that $(A\cap B)^c = A^c\cup B^c$
		% \item $\{a\in A : a\in B\} = \{b\in B : b\in A\}$
	\end{enumerate}
\end{prob}

\begin{prob}
  % Useful lemma for cofinite topology
	Let $\mathcal X \subseteq \mathcal P(U)$ be a family of sets and define:
  $\mathcal Y=\{X^c\subseteq U : X\in \mathcal X\}$, where $X^c=U\setminus X$.
  Prove that:
  \begin{enumerate}
    \item $(\bigcup \mathcal X)^c = \bigcap \mathcal Y$
    \item $(\bigcap \mathcal X)^c = \bigcup \mathcal Y$
  \end{enumerate}
\end{prob}

\begin{exercise}
  % Useful lemma for classification of open sets using neighborhoods.
	Let $A\subseteq X_i$ for $i\in I$. Prove that
	$$A\subseteq \bigcup_{i\in I} X_i$$
\end{exercise}

\begin{exercise}
  % Useful lemma for classification of open sets using neighborhoods.
	For every point $a\in A$ there is a set $U_a\subseteq A$ such that $a\in U_a$.
	Prove that $$A=\bigcup_{a\in A} U_a.$$
\end{exercise}

\subsection{Cartesian product}
First of all, we need a useful concept:
\begin{definition}
  We define \textbf{an ordered pair} or \textbf{a 2-tuple} as
  $$(a,b) := \{\{a\}, \{a, b\}\}.$$
\end{definition}

\begin{prob}
  Prove that $(a,b)=(a',b')$ iff $a=a'$ and $b=b'$.
\end{prob}

\begin{prob}
  Prove that $(a,(b,c))=(d,(e,f))$ iff $a=d\wedge b=e\wedge c=f$.
\end{prob}

\begin{definition}
  \textbf{An ordered $n$-tuple} or simply \textbf{a tuple} is defined as:
  $$(a_1,a_2,\dots, a_n) := (a_1, (a_2, (..., a_n)) \dots ).$$

  It's single most important property is that:
  $$(a_1,a_2,\dots,a_n)=(a_1', a_2', \dots, a_n')$$ iff $a_1=a_1', a_2=a_2', \dots, a_n=a_n'.$
\end{definition}

In fact the property is much more important than the explicit construction. For example we could define a 3-tuple as $((a,b),c)$ instead of $(a, (b,c))$ and
the property would still hold! But one needs to be careful about the notation, as shows the next exercise.

\begin{exercise}
  Check that, in terms of sets, $(a,(b,c))\neq ((a,b),c)$, so formally we do need to stick to one convention for $(a,b,c)$.
\end{exercise}

\begin{definition}
  Let $A$ and $B$ be sets. Then we assume that their
  \textbf{Cartesian product} exists:
  $$A\times B = \{(a,b) : a\in A\wedge b\in B\}.$$
\end{definition}

\begin{exercise}
  Prove that Cartesian product is \textit{not} commutative (that is $A\times B\neq B\times A$ in general).
\end{exercise}

\begin{prob}
  Prove that in general $(A\times B)\times C\neq A\times (B\times C)$, so Cartesian product is \textit{not} associative and an expression $A\times B\times C$ is ambiguous.
  Later we will address this issue.
\end{prob}

\begin{definition}
  We define a \textbf{square} of a set $X$ as:
  $$X^2 = X\times X,$$
  that is a set of all pairs made from elements of $X$.
  Similarily, we define $X^n$ as the collection of all $n$-tuples made from elements of $X$.
\end{definition}

\section{Relations}
Having defined Cartesian product, we can consider subsets of it. It will lead to two new, important concepts - relations and functions.

\begin{definition}
  A \textbf{relation $R$ between sets} $X$ and $Y$ is a subset of $X\times Y$. If $(x,y)\in R$ we write $x\,R\,y$. A \textbf{relation on a set} $X$ is a subset of $X\times X$.
\end{definition}

\begin{example}
  Consider the order of natural numbers (that is $0<1, \,1<2,\,2<3$ and so on). It is in fact a relation on $\N$: $a<b$ means exactly $(a,b)\in\, <\,\subseteq \N\times \N$ and is defined as:
  $$< := \bigcup_{n\in \N}\bigcup_{i\in \Z^+} \{(n, n+i)\}, \text{ where } \Z^+=\{n\in \N : n\neq 0\}.$$
\end{example}

\begin{exercise}
  What is "the smallest" relation between $X$ and $Y$ (in such sense that is a subset of \emph{every} relation between $X$ and $Y$)? What is "the biggest" one (every relation is a subset of the biggest one)?
\end{exercise}

\begin{exercise}
 Let $X$ and $Y$ be any sets. Prove that there exists the \textbf{set} of all relations between $X$ and $Y$. [Hint\footnote{Use the power set.}]
\end{exercise}

\begin{exercise}
  Let $X$ and $Y$ be finite sets. How many relations can be defined between them?
\end{exercise}

Among all the relations on a set $X$, we have some with very nice behaviour.

\begin{definition}
  Let $\equiv$ be a relation on $X$. We say that it is an \textbf{equivalence relation} if all of the following hold:
  \begin{enumerate}
    \item if $x\equiv y$ and $y\equiv z$, then also $x\equiv z$ (transitivity)
    \item if $x\equiv y$, then $y\equiv x$ (symmetry)
    \item $x\equiv x$ for every $x$ (reflexivity)
  \end{enumerate}
\end{definition}

\begin{example}
  Consider any set $X$. Then a set $$\Id_X := \{(x,x)\in X\times X : x\in X\}$$
  is an equivalence relation on $X$.
\end{example}

\begin{exercise}
  Prove that $n\equiv m$ iff $n$ and $m$ have the same parity is an equivalence relation on $\Z$.
\end{exercise}

As you may have noticed, using the equivalence relation with partition the set into some subsets.

\begin{definition}
  Let $X\neq \emptyset$ be a set. We say that a family of subsets $\mathcal A\subseteq \mathcal P(X)$ \textbf{partitions} X iff:
  \begin{enumerate}
    \item $\emptyset \neq X$
    \item $\bigcup \mathcal A=X$ (every element is somewhere)
    \item for $A,A'\in \mathcal A$ we have either $A=A'$ or $A\cap A'=\emptyset$ (partitioning sets are pairwise disjoint)
  \end{enumerate}
  Elements of $\mathcal A$ are called \textbf{equivalence classes}. If $a\in A\in\mathcal A$, we write $[a]:=A$.
\end{definition}

Why do we call it equivalence classes? Is it somehow related to equivalence relations?

\begin{exercise}
  Here you will prove the fundamental relationship between partitions and equivalence relations.
  \begin{enumerate}
    \item Prove that if we have a parition on $X$, then the relation given by: $x\equiv y$ iff $x$ and $y$ belong to the same equivalence class, is an equivalence relation on $X$.
    \item Let $\equiv$ be an equivalence relation on $X$. Prove that $\{[x] : x\in X\}$ is a partition on $X$, where $[x]=\{y\in X : y\equiv x\}$
  \end{enumerate}
  The partition of $X$ corresponding to relation $\equiv$ is written as $X/\equiv.$
\end{exercise}

\begin{exercise}
  Consider an equivalence relation $\equiv$.
  \begin{enumerate}
    \item Prove that $[a]=[b]$ iff $a\equiv b$.
    \item Prove that $[a]\cap [b]=\emptyset$ iff $a\not\equiv b$.
  \end{enumerate}
  This means that equivalence classes can be either identical or disjoint (what is not surprising as equivalence classes form a partition).
\end{exercise}

\begin{exercise}
  Let $X$ be a set with $n$ elements and $q$ be the number of possible equivalence classes on $X$. Prove that $$n\le q \le 2^{n^2}-1.$$ [Hint\footnote{For $n\ge 2$ construct $n$ equivalence relations with two classes.}]
\end{exercise}

Usually our sets will be equipped with some additional structure - for example integers can be added together. Sometimes we can move this structure to the equivalence classes. Let's start by finding a nice equivalence class on them.

\begin{example}
  \textbf{Modulo arithmetics}
  Let $p$ and $q$ be integers. $p\,|\,q$ means that $p$ divides $q$ (there exists a $m\in \Z$ such that $q=pm$). We fix a non-zero number $p\in \Z$ and define \textbf{equivalence modulo $p$}:
  $$m\equiv_p n \Leftrightarrow p\,|\,m-n.$$

  It's easy to check that this is an equivalence relation. We would like to define a sum on the set of equivalence classes. Let's try to do this intuitively:
  $$[m] + [n] := [m+n].$$

  Although it looks right, we need to check whether this definition is independent on the chosen representatives! So let's $m\equiv_p m'$ and $n\equiv_p n'$. We would like to show
  that $m+n\equiv_p m'+n'$. In other words, we want $p$ to divide $(m+n)-(m'+n')$, what is true as $(m+n)-(m'+n')=(m-m')+(n-n')$, that is a sum of numbers divisible by $p$.
\end{example}

Analogously one can define multplication and subtraction to get the modulo arithmetics known from elementary number theory.

\begin{exercise}
  \textbf{Construction of rationals}
  \begin{enumerate}
    \item Let $\Z^*=\Z\setminus\{0\}$. Consider $X=\Z\times \Z^*$. Prove that relation $\equiv$ on $X$ given as: $(m,n)\equiv (p,q)\Leftrightarrow mq=pn$ is an equivalence relation.
    \item To simplify notation, we will write $[m,n]$ for $[(m,n)]\in X/\equiv$. Prove that the following operations do not depend on class representatives:
      \begin{enumerate}
        \item $[m,n] + [p,q] := [mq+np, nq]$
        \item $[m,n]\cdot [p,q] := [mp, nq]$
      \end{enumerate}
    \item Prove that:
      \begin{enumerate}
        \item $[m,n]=[am,an]$
        \item $[0,1]+[m,n]=[m,n]$
        \item $[1,1]\cdot [m,n]=[m,n]$
        \item $[m,n] + [-m,n]=[0,1]$
        \item if $[a,b]\neq [0,1]$, then $[a,b]\cdot [b,a]=[1,1]$
      \end{enumerate}
    \item Consider any rational numbers $m/n$ and $p/q$. What equivalence classes do they correspond to? What is their sum and product? Do you see now how we can construct rationals using integers only?
  \end{enumerate}
\end{exercise}

The last example and exercise showed us how to move algebraic structures from one set to another (usually corresponding to equivalence classes of some relation). In fact one can define integers using natural numbers only\footnote{This is even simpler - our equivalence classes are 1-element. Consider $\N\times \{0,1\}$ with $(n,0)$ corresponding to $n$
and $(n,1)$ corresponding to $-n$. Figure how to define addition, subtraction and multiplication. Later we will also discover how to construct reals from rationals.} or reals from rationals\footnote{This actually involves equivalence classes, put on sequences of rationals. We will investigate this construction later.}.

\section{Functions}
\label{sec:intro_to_functions}

\begin{definition}
  Consider two sets $A$ and $B$. We say that a relation $f$ (that is a subset $f\subseteq A\times B$) is a \textbf{function}
  iff the following two conditions hold:
  \begin{itemize}
    \item for every element $a\in A$ there is an element $b\in B$ such that $(a,b)\in f$
    \item if $(a,b)\in f$ and $(a,c)\in f$, then $b=c$
  \end{itemize}
  Therefore for each $a\in A$ there is exactly one $b\in B$ such that $(a,b)\in f$. Such $b$ will be called \textbf{value of $f$ at point $a$} and given a symbol $f(a).$
  We will frite $f: A\to B$ for $f$ and call $A$ the \textbf{domain of $f$} and $B$ the \textbf{codomain of $f$}.

  Being very concise we can also write $f$ as
  $$f: A\ni a \mapsto f(a)\in B.$$
  Note that we use two different arrows.
\end{definition}

\begin{example}
  $f: \N\to \R$ given by $f(n)=n^2$. We can also write: $$f: \N\ni n\mapsto n^2\in \R.$$
\end{example}

\begin{example}
  $f: \R\to \R$ given by $f(x)=x^{10}+x^2-1$.
\end{example}

\begin{example}
  $f: X\to \mathcal P(X)$ given by $f(x)=\{x\}$.
\end{example}

\begin{exercise}
  Let $X$ and $Y$ be two sets. Prove that there exists a set of all functions from $X$ to $Y$. [Hint\footnote{You can form a set of all relations between $X$ and $Y$. How are functions related to relations?}]
\end{exercise}

\begin{exercise}
	How many\footnote{Thanks to Antek Hanke} are there functions from the empty set to $\{1,2,3,4\}$? [Hint\footnote{What is a function in set-theoretical terms?}]
\end{exercise}

\begin{exercise}
  Here, we will prove a simple inequality using a set-theoretic reasoning. Let $X$ and $Y$ be finite sets, with numbers of elements, respectively, $x=|X|$ and $y=|Y|$.
  \begin{enumerate}
    \item Prove that the number of relations between $X$ and $Y$ is $2^{xy}$.
    \item Prove that the number of functions from $X$ to $Y$ is $y^x$. [Hint\footnote{For first element in $X$ you have $y$ possibilities to choose from.}]
    \item Prove that for every non-zero natural numbers $x$ and $y$ the following holds:
      $$y^x<2^{xy}.$$
  \end{enumerate}
\end{exercise}

\begin{exercise}
  Let $X$ and $Y$ be any two sets. Prove that you can create a set of all functions from $X$ to $Y$. Sometimes it is called $Y^X$. Do you know why?
\end{exercise}

\begin{exercise}
  Consider a function $f: X\to X'$ and assume that there is an equivalence relation $R'$ on $X'$. We will try to define a natural (in some sense) equivalence relation on $X$.
  \begin{enumerate}
    \item Define a relation $R$ on $X$ as $xRy\Leftrightarrow f(x) R' f(y)$. Prove that it is an equivalence relation.
    \item Consider $r: X\to X/R$ and $r': X'\to X'/R'$ given by $r(x)=[x]_R$ and $r'(x')=[x']_{R'}$ and inverse function.
  \end{enumerate}
\end{exercise}

\begin{exercise}
	Let $f: A\to B$ and $C\subseteq D\subseteq A$. We define: $f[C] = \{b\in B : b=f(c) \text{ for some }c\in C \}$ and analogously $f[D]$. Prove that
	$f[C]\subseteq f[D].$
\end{exercise}

\begin{definition}
  Consider a set $X$. We say that it's \textbf{identity function} is $f:X\to X$ given by $f(x)=x$ for all $x\in X$.
\end{definition}

\subsection{Injectivity, surjectivity and bijectivity}

\noindent As we have already seen, there may be some elements in codomain that are not values of $f$. Such a set is important enough to be given a name:

\begin{definition}
  Let $f:A\to B$ be a function. \textbf{The image of $f$} is a set:
  $$\Image f=\{b\in B : \text{there is } a\in A \text{ such that } b=f(a)\}.$$

  We say that the function $f: A\to B$ is \textbf{surjective} (or \textbf{onto}) iff $\Image f=B$.
\end{definition}

\begin{prob}
	As we remember, $\mathbb{R}$ stands for real numbers. Are the following functions surjective?
	\begin{enumerate}
		\item $f: \mathbb{R} \to \mathbb{R}, ~f(x)=x^3$
		\item $g: \mathbb{R} \to \mathbb{R}, ~g(x)=x^2$
		\item $h: \mathbb{R} \to \{5\}$
	\end{enumerate}
\end{prob}

\begin{definition}
  Let $f:A\to B$ be a function. If $f$ gives distinct values to distinct arguments (that is, if $f(a)=f(b)$, then $a=b$), we say that the function is \textbf{injective}
  (or \textbf{one-to-one}).
\end{definition}

\begin{exercise}
  Are the following functions injective?
	\begin{enumerate}
		\item $f: \mathbb{R} \to \mathbb R, ~f(x)=x^2$
		\item $h: \{0,1,2,3\} \to \mathbb R, ~h(x)=x$
	\end{enumerate}
\end{exercise}

\begin{exercise}
	Let $f$ be a function from $A$ to $B$. Prove that there exists a function $g: \Image B\to A$ such that $g\circ f=\Id_A$ iff $f$ is injective.
\end{exercise}

\begin{exercise}
	Let $f: A\to B$ and $g: B\to C$ be functions such that $g\circ f$ is injective but $g$ is not. Why isn't $f$ surjective?
\end{exercise}

\begin{definition}
If a function $f$ is both surjective and injective, we say that is \textbf{bijective}\footnote{If you prefer nouns: surjective function is called a surjection, injective - injection
and bijective - bijection}.
\end{definition}

\begin{exercise}
	Construct a function that is:
	\begin{enumerate}
		\item surjective, but not injective
		\item injective, but not surjective
		\item neither injective nor surjective
		\item bijective
	\end{enumerate}
\end{exercise}

\noindent Notice that if a function $f: A\to B$ is bijective, then we can construct a function $g:B\to A$
such that $f(g(b))=b$ and $g(f(a))=a$.

\begin{prob}
	Prove that, if exists, $g$ is unique.
\end{prob}

\begin{definition}
  Consider a bijective function $f:X\to Y$. We say that it's \textbf{inverse function} $f^{-1}:Y\to X$ iff:
  $$f^{-1}(f(x))=x, f(f^{-1}(y))=y,$$
  for all $x\in X,\, y\in Y$.
\end{definition}

\noindent We call this function \textbf{the inverse function}\footnote{It becomes confusing when working on real numbers: $f^{-1}(x)$ is
\textbf{not} $(f(x))^{-1}=1/f(x)$}: $g=f^{-1}.$

\begin{prob}
	Assume that $f^{-1}$ exists. Prove that $(f^{-1})^{-1}$ exists and is equal to $f$.
\end{prob}

\subsection{Function composition}
If we have two functions: $f:A\to B$ and $g: B\to C$, we can construct the \textbf{composition} using formula:
$g\circ f: A\to C,~(g\circ f)(a) = g(f(a)).$

\begin{exercise}
  Recall that for two relations $R\subseteq X\times Y$ and $T\subseteq Y\times Z$ we defined their composition as $$R\circ T=\{(x,z)\in X\times Z : \exists_{y\in Y} (x,y)\in R \wedge (y,z)\in T\}$$
\end{exercise}

\begin{exercise}
	Find functions $f,~g$ such that:
	\begin{enumerate}
		\item $g\circ f$ exists, but $f\circ g$ is not defined
		\item both $f\circ g$ and $g\circ f$ exist, but $f\circ g\neq g\circ f$
	\end{enumerate}
\end{exercise}

Although function composition is not commutative, it is associative:
\begin{exercise}
	Left $f:A\to B, g: B\to C, h: C\to D$. Prove that
	$$h\circ (g\circ f) = (h\circ g)\circ f.$$
\end{exercise}
Therefore we can ommit the brackets and write just $h\circ g\circ f.$ We will use function composition very
often.

\begin{exercise}
    \begin{enumerate}
	   \item Prove that composition of two surjections is surjective.
	   \item Prove that composition of two injections is injective.
	   \item Prove that composition of two bijections is bijective.
    \end{enumerate}
\end{exercise}

\begin{definition} We will rephrase the definition of the inverse function using the identity function\footnote{For a set $X$, it's identity function is
    $$\Id_X=\{(x,x)\in X\times X : x\in X\}.$$}:

    consider a function $f:X\to Y$. If there exists a function $f^{-1}:Y\to X$ such that:

    $$f^{-1}\circ f=\Id_X,  f\circ f^{-1}=\Id_Y,$$

    we say that $f^{-1}$ if \textbf{the inverse} to $f$.
\end{definition}

\begin{exercise}
  Let $f: A\to B$ be an injection. Prove that there is a function
  $g: \text{Im\,} f \to A$ such that $g\circ f = \text{Id}_A.$
  Such $g$ is called \textbf{left inverse of $f$}.
\end{exercise}

\subsection{Commutative diagrams}
\begin{quote}
  Use a picture. It's worth a thousand words.\\
  - Tess Flanders
\end{quote}

Consider functions $f:X\to Y$ and $g:Y\to Z$. We introduced the composition of them given us $g\circ f: X\to Z$. We can visualise it using a following \textbf{diagram} (Fig. \ref{fig:commutative_diagram_intro}):

\begin{figure}
  \centering
  \begin{tikzcd}
    X \arrow[rd, "g\circ f"] \arrow[rr, "f"] &       & Y \arrow[ld, "g"]\\
                                           &  Z   &
  \end{tikzcd}
  \caption{An example of a diagram.}
  \label{fig:commutative_diagram_intro}
\end{figure}

We say that this diagram \textbf{commutes} (or we say that this is a \textbf{commutative diagram}) as you can use follow any path and obtain the same result.

\begin{exercise}
  Prove that the diagram \ref{fig:commutative_diagram_composition} commutes iff $h=g\circ f$.

  \begin{figure}
    \centering
    \begin{tikzcd}
      X \arrow[rd, "h"] \arrow[rr, "f"] &       & Y \arrow[ld, "g"]\\
                                             &  Z   &
    \end{tikzcd}

    \caption{What can you say about $f,\,g,\,h$ if the diagram commutes?}
    \label{fig:commutative_diagram_composition}
  \end{figure}
\end{exercise}

\begin{exercise}
  What can you say if diagram \ref{fig:commutative_diagram_toy_natural_transformation} commutes?

  \begin{figure}
    \centering
    \begin{tikzcd}
      X \arrow[d, "h"] \arrow[r, "f"] &  Y \arrow[d, "g"]\\
      Z \arrow[r, "j"]                &  T
    \end{tikzcd}

    \caption{What can you say about the functions involved if the diagram commutes?}
    \label{fig:commutative_diagram_toy_natural_transformation}
  \end{figure}
\end{exercise}

\section{Cardinality}
\subsection{Finite sets}
\begin{definition}
  The \textbf{cardinality} $|X|$ of a finite set $X$ is defined as the number of elements in $X$.
\end{definition}

\begin{example}
  Let $A=\{0,1,2,3\}$. Then $|A|=4$.
\end{example}

\begin{exercise}
	What is the cardinality of $\{a, a+1, a+2, \dots, a+n\}$?
\end{exercise}

\begin{theorem}
  \textbf{Inclusion-exclusion principle}
  If $X$ and $Y$ are finite sets, then:
  $$|X\cup Y|=|X|+|Y|-|X\cap Y|.$$
\end{theorem}

Intuitively, adding two sets we count elements in each set twice and then subtract the number of elements that were counted twice. The formal proof goes as follows:

\begin{exercise}
  Prove the inclusion-exclusion principle:

  \begin{enumerate}
    \item Let $X$ and $Y$ be finite, disjoint (that is $X\cap Y=\emptyset$) sets. Prove that:
    $$|X\cup Y| = |X| + |Y|.$$
    \item Prove that for $A\subseteq X$, where $X$ is finite, we have $|X\setminus A|=|X|-|A|$. Hint: $X\setminus A$ and $A$ are disjoint and sum up to $X$...
    \item Prove that $$|X\cup Y|=|X|+|Y|-|X\cap Y|$$ for finite sets $X,\, Y$ (now we don't assume that they are disjoint). Hint: what is $(X\setminus(X\cap Y))\cup Y$?
  \end{enumerate}
\end{exercise}

\begin{exercise}
  Prove that if $B\subseteq A$, and $A$ is finite, then $|B|\le |A|$. When does the equality hold?
\end{exercise}

\begin{exercise}
  Prove that $|\mathcal P(A)|=2^{|A|}$ for a finite set $A$. Do you see why the power set $\mathcal P(A)$ is often referenced as $2^A$?
\end{exercise}

\begin{exercise}
    Let $A,B,C$ be finite sets. Prove that:
		$$|A\cup B\cup C| = |A|+|B|+|C| - |A\cap B| - |B\cap C|-|C\cap A| + |A\cap B\cap C|.$$
\end{exercise}

\begin{exercise}
  Let $X=\{1,2,\dots, 2018\}$.
\end{exercise}

\subsection{Characteristic functions}
\begin{definition}
  Fix a set $U$. For each subset $A\subseteq U$ we define it's \textbf{characteristic function} or \textbf{indicator function} as:

  $$1_A: U\to \{0,1\}$$
  $$1_A(x) = \begin{cases}1, \text{ if } x\in A\\ 0, \text{ if } x\notin A\end{cases}$$
\end{definition}

\begin{example}
  Consider a set $U$. Then $1_\emptyset(x)=0$ and $1_U(x)=1$ for every $x\in U$. It's usually abbreviated as:
  $$1_\emptyset=0, 1_U=1.$$
\end{example}

\begin{exercise}
  Let $A,B\subseteq U$. Prove that:
  \begin{enumerate}
    \item $1_{A_\cap B}=1_A\cdot 1_B$\footnote{It means that for every $x\in U$ we have $1_{A_\cap B}(x)=1_A(x)\cdot 1_B(x)$}
    \item $1_{A^c}=1-1_A$, where $A^c=U\setminus A$
    \item $1_{A\cup B}=1_A+1_B-1_A\cdot 1_B$
  \end{enumerate}
\end{exercise}

\begin{exercise}
  Prove inclusion-exclusion principle for finite sets using characteristic functions. Hint: write $1_{A\cup B}$ in terms of $1_A, 1_B, 1_{A\cap B}$ and sum it's values over
  all elements in \textit{finite} set $A\cup B$.
\end{exercise}

\subsection{Comparing cardinalities}
Although we feel comfortable in counting elements of \textit{finite} sets, we don't know how to say how to compare infinite sets - there is no natural number we could use to denote
their cardinalities!

Therefore, we'll try another approach. Assume that we have a set of children and a set of toys. If we want to compare them, we can either try to calculate how many children and toys there are (it may be very hard if there are lots of children and lots of toys) or to ask each child to get one toy. If every child has \textit{one} toy and no toys are left, we know
that there are exactly as many children as toys! We'll use this approach to compare infinite sets.

\begin{definition}
  Let $A$ and $B$ be two sets. If there exists a bijection $f:A\to B$, we say that $|A|=|B|$ (are of the same cardinality).
\end{definition}

\begin{example}
  $|\N|=|2\N|$, where $2\N$ is a set of all even natural numbers, as we can find a bijection $n\mapsto 2n$. It's a surprising result, as $2\N\subseteq \N$ is a \textit{proper} subset. If $\N$ was finite, all it's proper subsets would have smaller cardinalities!
\end{example}

\begin{exercise}
  Being of the same cardinality has similar properties to these of equivalence relation\footnote{... but as there is no sets of all sets, it is not formally an equivalence relation.}. Prove that:
  \begin{enumerate}
    \item $|A|=|A|$
    \item $|A|=|B|$ implies that $|B|=|A|$ (hint: bijections have inverses)
    \item if $|A|=|B|$ and $|B|=|C|$, then $|A|=|C|$ (hint: what is a composition of bijections?)
  \end{enumerate}
\end{exercise}

\begin{definition}
  We say that a set $X$ is \textbf{countably infinite} if $|X|=|\N|$. Usually we'll write that $\aleph_0:=|\N|$ (read "aleph 0").
  We say that a set $X$ is \textbf{countable} if $X$ is finite or countably infinite.
\end{definition}

\begin{example}
  Sets $\N, 2\N, \{0,1,6, 41\}$ are countable.
\end{example}

\begin{example}
  $\Z$ is countable: $0\mapsto 0, 1\mapsto 1, 2\mapsto -1, 3\mapsto 2, 4\mapsto -2,\dots$
\end{example}

\begin{exercise}
  Prove that a subset of a countable set is countable.
\end{exercise}

\begin{exercise}
  Let $A$ and $B$ be countable sets. Prove that $A\cup B$ and $A\cap B$ are countable.
\end{exercise}

\begin{exercise}
  Let $A$ and $B$ be countable. Prove that $A\times B$ is countable. Hint: you can write all elements of $A$ as $a_1,a_2,\dots$ and the elements of $B$ as $b_1, b_2,\dots$.
  Think about an ordering $(a_1,b_1); (a_1, b_2), (a_2, b_1); (a_1, b_3), (a_2, b_2), (a_3, b_1); \dots$ (some terms may be repeated if $A$ and $B$ are not disjoint, think how to fix it).
\end{exercise}

\begin{exercise}
  Prove that $\Q$ is countable.
\end{exercise}

\begin{exercise}
  Let $\mathcal A$ be a countable family of countable sets. Prove that $\bigcup \mathcal A$ is countable.
\end{exercise}

\begin{exercise}
  Prove that is $X$ is an infinite subset, then it contains a countably-infinite subset $S\subseteq X, |S|=\aleph_0$.
\end{exercise}

The last exercise shows that we can compare cardinalities. That is, if we can find a bijection between $A$ and \textit{some subset} of $B$, we can be sure that $B$ contains at least
as many elements as $A$. This is exactly requiring the existence of an \textit{injection} from $A\to B$.

\begin{definition}
  If there exists an injection $f:A\to B$ we say that $B$ has greater or equal cardinality than $A$ and write $|A|\le |B|$. If $|A|\le |B|$ and $|A|\neq |B|$, we write
  $|A| < |B|$ (that is: we can find an injection from $A$ to $B$, but there is no bijection between them).
\end{definition}

% \begin{exercise}
% 	Above we find the way of saying that two cardinalities are equal using existence of a bijection. Let's find a way to compare which is less using
% 	another kind of function.
% 	\begin{enumerate}
% 		\item Let $O_n=\{1,2,\dots,n\}.$ Prove that there is no injection from $O_{n+1}$ into $O_n$. Hint: use mathematical induction.
% 		\item Let $A$ and $B$ be finite. Prove that there is an injection from $A$ to $B$ iff $|A| \le |B|.$
% 	\end{enumerate}
% \end{exercise}
%
% \begin{exercise}
% 	Prove in one: if there is an injection from $A$ onto $B$ and an injection from $B$ into $A$, then there exists a bijection from $A$ onto $B$. Hint: you know that $|A|\le B$ and $|B|\le |A|$.
% \end{exercise}


\begin{exercise}
	Let $A,\,B$ and $C$ be sets. Prove that if $|A|\le |B|$ and $|B|\le |C|$, then $|A|\le |C|$.
\end{exercise}

\begin{prob}
	Here you can prove that there are more real numbers than naturals or rationals. We define $X=\{x\in \mathbb R : 0\le x\le 1\}$ and choose one
	 convention of writing reals (e.g 0.999... = 1.000..., so we can choose to use nines)
	\begin{enumerate}
		\item Assume that you have written all the elements of $X$ in a single column. Can you find a real number that does not occur in the list?
		\item Using the above, prove that $|\mathbb N| < |X|$
		\item Prove that $|\mathbb Q| < |\mathbb R|.$
	\end{enumerate}
\end{prob}

\begin{prob}
	We know that $|\mathbb R| > |\mathbb N|.$ Using binary system prove that $\mathbb R=\mathcal P(\mathbb N)$. Do you see similarity between the previous result
	and $2^n > n$ for natural $n$?
\end{prob}

\begin{prob}
	\textbf{Cantor's theorem} You will prove that $|A|<\left|\mathcal P(A)\right|$ for any set $A$. Let $A$ be a set and $f:A\to \mathcal P(A).$
	\begin{enumerate}
		\item Consider $X=\{a\in A : a\notin f(a)\}\in \mathcal P(A)$. Is there $x\in A$ for which $f(x)=X?$
		\item Is $f$ surjective?
		\item Find an injective function $g: A\to \mathcal P(A)$.
		\item Prove that $|A| < |\mathcal P(A)|$ for any set $A$.
		\item Use Cantor's theorem to prove that there is no set of all sets.
	\end{enumerate}
\end{prob}

\begin{prob}
	\textbf{Cantor-Schroeder-Bernstein theorem} Let's prove that if $|A|\le|B|$ and $|B|\le |A|$, then $|A|=|B|$.
	\begin{enumerate}
		\item (Knaster-Tarski) Now assume that $F$ has \textit{monotonicity} property: $F(X)\subseteq F(Y)$ if $X\subseteq Y$.
			Prove that $F$ has a fixed point $S$ (that is $F(S)=S$), where:
			$$S=\bigcup_{X\in U} X, \text{~where~} U= \{Y\in 2^A : Y\subseteq f(Y)\}.$$
		\item (Banach) Let $f: A\to B$ and $g:B\to A$ be injections.
			We introduce new symbol: $f[X]=\{b\in B : b=f(x) \text{ for some } x\in X\}$. Prove that
			function $$F:\mathcal P(A)\to \mathcal P(A),~F(X)=A\setminus g[B\setminus f[X]]$$
			has the monotonicity property.
    \item Prove that $A\setminus S\subseteq \text{Im}\,g$, where
      $F$ and $S$ are taken from above.
		\item Prove that function
			$$h(x) =
				\begin{cases}
					f(x), x\in S\\
					g^{-1}(x), x \notin S
				\end{cases}
			 $$
			 is a bijection.
	\end{enumerate}
\end{prob}

% !TEX root = book.tex

\section{The Axiom of Choice}
We formulated comparision of cardinalities in terms of injections. We based on the following exercise:

\begin{exercise}
  Let $f$ be a function from $A$ to $B$. Prove that there exists a function $g: \Image B\to A$ such that $g\circ f=\Id_A$ iff $f$ is injective.
\end{exercise}

That is for an injective function there exists a "left inverse". We may ask a question - is a some kind of inverse possible for \emph{surjections}?

\begin{exercise}
  Consider a surjective function $f: \Z \to \{0,1\}$ given by $2k+1 \mapsto 1, 2k\mapsto 0, k\in \Z$.
  \begin{enumerate}
    \item why a \emph{left} inverse does not exist?
    \item define a \emph{right} inverse, that is a function $g:\{0,1\}\to \Z$ such that $f\circ g=\Id_{\{0,1\}}$
  \end{enumerate}
\end{exercise}

In the above exercise we had no problem - just pick an element from the set of odd numbers (these that are mapped to 1) and an element from the set of even numbers (these that are mapped to 0). While there is no problem of picking an element from each set if we have just two (or three, four - any finite number), this issue may apear for \emph{infinite} families
of sets.

\begin{definition}
  \textbf{Axiom of choice (AC)} Let $\mathcal A$ be a non-empty family of non-empty sets. Then there exists a \textbf{choice function} $f:\mathcal A\to \bigcup \mathcal A$ such that
  $f(A)\in A$ for every $A\in \mathcal A$.
\end{definition}

Basically it means that for every family of sets, we can select an element from each set - for a set $A$, such element is just $f(A)$, where $f$ is the choice function. Alternatively,
we could formulate it equivalently as:

\begin{definition}
  \textbf{Axiom of choice (AC)} Let $\mathcal S=\{S_i: i\in I\}$ be any family of non-empty sets such that $S_i\cap S_j=\emptyset$ for $i\neq j$. Then it is possible to create a set $C$ such that for every $i\in I$ there is $s_i\in C$ such that $s_i\in S_i$. Or in natural-language terms: from every set of a family of nen-empty, pairwise-disjoint sets, we can select exactly one element.
\end{definition}

This axiom allows us to construct right inverses:

\begin{exercise}
  Prove that AC (the axiom of choice) is equivalent to the statement that every surjection possesses a right inverse. Hint: for $AC\Rightarrow \text{right inverse}$ use the same idea as in the previous problem. For
  $\text{right inverse}\Rightarrow AC$ construct a surjective function from $\bigcup \mathcal S\to \mathcal S$, where $\mathcal S$ is a family of non-empty, pairwise-disjoint sets.
\end{exercise}

\begin{exercise}
  Prove, assuming AC, that if $f:A\to B$ is a surjection, then, there exists an injection $g: B\to A$.
\end{exercise}

Therefore with AC it makes sense to compare cardinalities using surjections:

\begin{exercise}
% Cardinalities with surjections
  Prove, assuming AC, that:
  \begin{enumerate}
    \item $A\le B$ iff there exists a surjection from $B$ to $A$
    \item if there is a surjection from $A$ to $B$ and a surjection from $B$ to $A$, then there exists a bijection between $A$ and $B$
  \end{enumerate}
\end{exercise}

It can also be useful in problems involving infinitely many hats:

\begin{exercise}
  % Hats and AC
  A king said $\aleph_0$ mathematicians the following:
  "Tomorrow, you will be standing in a long queue and my servants will place a red or green hat on everyone's head. You will see only the hats of the people standing before you.
  On a given signal, you need to guess your own hat. If infinitely many of you guess wrong, I will send you to the prison for the rest of your lifes!".
  By considering a set of all functions from $\N\to \{"red", "green"\}$ and a suitable partition on it, prove, assuming the axiom of choice, that mathematicians can make finitely-many
  wrong guesses.
\end{exercise}

In fact, AC implies much more - as Banach-Tarski paradox says using it one can take a solid sphere, cut it into a few pieces and compose \emph{two} spheres of the same size, just by moving the pieces around. Therefore many mathematicians try to avoid it as much as possible - it is a good habit always to explicitly mention it's usage. In many places in this book we will use AC, usually in an equivalent form known as Kuratowski-Zorn lemma\footnote{In English literature it is widely known as \textbf{Zorn's lemma}. Kazimierz Kuratowski proved this lemma (although with an unnecessary assumption) in 1922 and Max Zorn, working independently, gave the above formulation in 1935. The Bourbaki group and John Tukey used the latter name in their books published in 1939 and 1940 and since then "Zorn's lemma" is widely recognised.}.

\subsection{Kuratowski-Zorn (Zorn's) lemma}
\begin{definition}
  A \textbf{partial order} is a relation $\le$ on a set $A$ such that for all $a,\,b,\,c\in A$:
  \begin{enumerate}
    \item $a\le a$
    \item $a\le b \wedge b\le a\Rightarrow a=b$
    \item $a\le b\wedge b\le c\Rightarrow a\le c$.
  \end{enumerate}
  If for every $a,\,b\in A$ we have $a\le b$ or $b\le a$, then we say that it is a \textbf{total order} or \textbf{linear order}.
\end{definition}

\begin{example}
  Natural numbers, integers and reals are totally ordered.
\end{example}

\begin{example}
  Consider a set $\mathcal P(A)$ for some set $A$. It's partially ordered by the relation:
  $$B\le C \Leftrightarrow B\subseteq C.$$
  Note that some sets cannot be compared (neither $A\le B$ nor $B\le A$), so this order is \textit{not} total.
\end{example}

\begin{definition}
  A \textbf{partially-ordered set} or a \textbf{poset} is a pair $(A, \le)$, where $A$ is a set and $\le$ is a partial order on $A$. If $B\subseteq A$ is a subset on which
  $\le$ is total (every two elements of $B$ can be compared, or in set-theoretic terms $B\times B\subseteq \le$), we call $B$ a \textbf{a chain}.
\end{definition}

\begin{example}
  Consider $A=\{0,1\}$. Then it's power set ordered by inclusion - $(\mathcal P(A), \subseteq)$ - is a poset. If we take $B=\{\emptyset,A\}\subseteq \mathcal P(A)$,
  then every two elements of $B$ can be compared - it's a chain.
\end{example}

\begin{definition}
  Let $(A, \le)$ be a poset and $B\subseteq A$ be a chain. We say that $u\in A$ is an \textbf{upper bound} of a chain $B$ if $b\le u$ for every $b\in B$.
  We say that $m\in A$ is a \textbf{maximal element} if for every $a\in A$ we have $m\le a\Rightarrow m=a$, that is there is no greater element than $m$.
\end{definition}

\begin{example}
  Let $A=\{1,2,3,4,5\}$ with standard order. Then 5 is a maximal element in $A$ and an upper bound of $A$.
\end{example}

\begin{theorem}
  \textbf{Kuratowski-Zorn (Zorn's) lemma}
  Let $(P, \le)$ be a poset such that every chain in $P$ has an upper bound. Then there exists a maximal element in $P$.
\end{theorem}

For a proof, you can check Arjun Jain's "Zorn’s Lemma An elementary proof under the Axiom of Choice"\footnote{https://arxiv.org/pdf/1207.6698.pdf}. We will usually use AC in this form.

\section{Problems}
\begin{exercise}
	Let $S\subseteq R$. We say that $S$ is \textbf{well-ordered} iff any non-empty subset $X\subset S$ has the smallest element.
	\begin{enumerate}
		\item Prove that reals and integers with the default ordering are not well-ordered.
		\item Assume that $X\subseteq \mathbb N$ doesn't have the smallest element. Define $$A=\{n\in \mathbb N : \{0,1,\dots,n\}\cap X=\emptyset\}$$
			and use mathematical induction to prove that $X$ is empty.
		\item Why are natural numbers well-ordered?
	\end{enumerate}
\end{exercise}

% \section{Numbers}

At the beginning we assumed that you know how to work with different kinds of numbers. This aim of this section is to give a few properties of them.

\subsection{Real numbers}
\begin{definition}
  A \textbf{field} is a tuple $(\F, +, \cdot)$ such that:
  \begin{itemize}
    \item $\F$ is a set
    \item $+$ and $\cdot$ are functions from $\F\times \F$ to $\F$. We write
      $a+b$ for $+(a,b)$ and $a\cdot b$ for $\cdot (a,b)$.
  \end{itemize}
  They have these properties:
  \begin{enumerate}
    \item $a+(b+c)=(a+b)+c$ for all $a,b,c$ (addition is associative)
    \item $a\cdot (b\cdot c)=(a\cdot b)\cdot c$ for all $a,b,c$ (multiplication is associative)
    \item $a+b=b+a$ for all $a, b$ (addition is commutative)
    \item $a\cdot b=b\cdot a$ for all $a,b$ (multiplication is commutative)
    \item there is an element $0\in\F$ such that $a+0=a$ for all $a$ (addition has a neutral element). We call it \textbf{zero}
    \item there is an element $1\in\F$ such that $a\cdot 1=a$ for all $a$ (so 1 is neutral element of multiplication). We call if \textbf{one}
    \item for every $a$ there is $a'$ such that $a+a'=0$ (existence of
      an inverse element for addition)
    \item $a\cdot (b+c) = a\cdot b + a\cdot c$ for all $a,b, c$
      (multiplication distributesover addition)
    \item for every $a\neq 0$ there is $\tilde a$ such that $a\cdot \tilde a=1$
      (multiplication has an inverse element for all non-zero numbers)
    \item $1\neq 0$ (so $\F$ has at least two elements)
  \end{enumerate}

  Usually we will reference a field just as $\F$.
\end{definition}

\begin{exercise}
  Check that
  \begin{enumerate}
    \item real numbers understood informally, have the field properties
    \item rational numbers form a field
  \end{enumerate}
\end{exercise}

From the above field axioms, you can derive many facts that may be obvious
to you:
\begin{exercise}
  Prove that there is 0 and 1 are unique. Hint: assume that
  0 and 0' have property such that $a=a+0=a+0'$ and try $a=0$ and $a=0'$.
\end{exercise}

\begin{exercise}
  Prove that if $a+a'=0$ and $a+a''=0$, then $a'=a''$.
  Therefore we can introduce special symbol for \textit{the} additiv
  inverse: $a + (-a)=0$ and define subtraction as $a-b := a + (-b)$.
\end{exercise}

\begin{exercise}
  Prove that $-a=(-1)\cdot a$.
\end{exercise}

% Consider more problems on this topic

As you see, many of the algebraic properties we are used to can be recovered from the axioms, but sometimes it can be complicated. Both real numbers and
rational numbers have also an order on them - for example $2>1$. It leads
to the definition of \textit{total order}.

\begin{definition}
  We call a pair $(\F, \le)$ a \textit{totally ordered set} if for every $a,b\in \F$ we have:
  \begin{enumerate}
    \item $a\le b$ or $b\le a$ (we call this property totality)
    \item $a\le b$ and $b\le a$ imply $a=b$ (it's called antisymmetry)
    \item $a\le b$ and $b\le c$ imply $a\le c$ (transitivity)
  \end{enumerate}
  Having relation $\le$ we can define others: $b\ge a$ means that $a\le b$ and
  $a<b$ means that $a\le b$ and $a\neq b$.
\end{definition}

\begin{definition}
  \textbf{Ordered field} is a field $\F$ with a total order such that:
  \begin{itemize}
    \item $a\le b$ implies $a+c\le b+c$
    \item $0\le a$ and $0\le b$ imply that $0\le a\cdot b$
  \end{itemize}
  We say that $\F_+ := \{x\in \F : x \le 0\}$ is a set of \textbf{non-negative} numbers.
\end{definition}

\begin{exercise}
  Check that non-negative numbers are closed under summation and multiplication (the sum and product of two non-negative numbers is non-negative).
\end{exercise}

\begin{definition}
  Consider a subset of a totally-ordered set $A\subseteq \F$. We say that $x$ is an
  \textbf{upper bound} of $A$ iff $x\ge a$ for every $a\in A$.
  In such case we say that $A$ \textbf{is bounded from above}.
\end{definition}

\begin{exercise}
  Let $\F$ be either $\R$ or $\Q$. Prove that a set $A\subseteq \mathbb \F$ can have no upper bounds
  or infinitely many of them.
\end{exercise}

\begin{definition}
  Let $A\subseteq \F$ be a set bounded from above. It's \textbf{supremum} or \textbf{lower upper bound} is a number $\sup A\in F$ such that:
  \begin{itemize}
    \item $\sup A$ is an upper bound of $\F$
    \item if $y$ is an upper bound of $A$, then $y\ge \sup A$
  \end{itemize}
  Supremum is often referenced as $\text{lub}$ or $\text{l.u.b}$.
\end{definition}

\begin{exercise}
  Prove that supremum is unique, so if $x$ and $x'$ are supremums
  of $A$, then $x=x'$.
\end{exercise}

\begin{exercise}
  Prove that $x=\sup A$ if and only if
  $x\ge a$ for every $a\in A$ and for every $\eps > 0$ there is
  $a\in A$ such that $x < a + \eps$.
\end{exercise}

\begin{definition}
  \textbf{Completeness axiom} - each non-empty, bounded from above subset of $\R$ has a supremum.
\end{definition}

This axiom allows us to prove many interesting things:

\begin{exercise}
  Prove that natural numbers are \textit{not} bounded from above.
  Hint: if $n\in \mathbb N$, then $n+1\in \mathbb N$
\end{exercise}

\begin{exercise}
  Prove the \textbf{Archimedean axiom}\footnote{In fact we do not
  need to call it axiom, as we are able to prove it.}
  that for every $r\in R$, there is $n\in \mathbb N$ such that $n>r$.
\end{exercise}

\begin{exercise}
  Prove that for any $r>0$ there is $n\in \mathbb N$ such that
  $1/n < r$.
\end{exercise}

\begin{exercise} Find infinite sum and intersection for the families of subsets of $\mathbb{R}$:
  \begin{enumerate}
    \item $A_i=(0,1/i)$ for $i=1,2,\dots$
    \item $B_i=[0,1/i)$ for $i=1,2,\dots$
  \end{enumerate}
\end{exercise}

\begin{exercise}
  Prove that rational numbers do \textit{not} have the completeness
  property:
  \begin{enumerate}
    \item Let $p, q\in \mathbb Z\setminus \{0\}$.
      Prove that $p^2\neq 2q^2$.
    \item Prove that root of two, defined as
      $x > 0, x^2=2$ is not rational.
    \item Find a subset of $\mathbb Q$ that is bounded above, but
      has no rational supremum.
  \end{enumerate}
\end{exercise}

\begin{exercise}
  You should prove that in each nonempty interval there is at least
  one rational number:
  \begin{enumerate}
    \item Assume that $0<a<b$. Define
      $$A=\left\{\frac m N : m\in \mathbb N\right\},~
      \frac 1{b-a} < N \in \mathbb N$$
      and prove that $A\cap (a,b)$ is non-empty.
    \item Use the above result to prove that in \textit{each}
      interval there is at least one rational number.
    \item Prove that in each interval there are infinitely but countably many, rational numbers.
    \item Prove that in each interval there is an irrational number.
    \item How many irrational numbers are in each interval?
  \end{enumerate}
\end{exercise}

\subsection{Absolute value}
Another concept that will be further useful is the \textbf{absolute value} of a real number:
if $x\in \mathbb R$ we write $|x|\in \mathbb R$ for:
$$|x| = \begin{cases}x &\text{ for } x \ge 0\\ -x &\text{ otherwise} \end{cases}.$$

\begin{exercise}
  Prove that for every $x,y\in \mathbb R$:
  \begin{enumerate}
    \item $|x|=|-x|$
    \item if $|x|=|y|$ then $x=y$ or $x=-y$.
    \item $|x+y| \le |x| + |y|$ (this is called \textbf{triangle inequality})
    \item $|x-y|\le |x| + |y|$
    \item $\left||x| - |y|\right|\le |x-y|$ (this is sometimes calles \textbf{reverse triangle inequality})
  \end{enumerate}
\end{exercise}

\subsection{Mathematical induction}
We characterise natural numbers as a subset of $\R$ with some properties.

\begin{definition}
  Let $A\subseteq \R$ be a set such that:
  \begin{itemize}
    \item $0\in A$
    \item if $n\in A$, then $n+1\in A$
  \end{itemize}
  The intersection of all sets with this property is called \textbf{the set of natural numbers}. Alternatively, natural numbers is the smallest set (in sense of subset order)
  with the properties listed above.
\end{definition}

These properties allow us to define a powerful proof technique:

\begin{definition}
  \textbf{Principle of mathematical induction} -
  let $f$ be a logical statement defined for all natural numbers (that is $f(n)$ is either true or false). Then if $f(0)$ is true and an implication:
  $$f(n)\Rightarrow f(n+1)$$
  is true for every $n$, then $f(n)$ is true for every $n$.
\end{definition}

Have you ever seen falling dominoes? To be sure that all domino falls, we need to:
\begin{enumerate}
	\item punch the first domino
	\item every domino must punch the next domino (if this particular domino falls, the next one also falls)
\end{enumerate}
And that's all, we can be sure that all the dominoes will eventually fall.

\begin{example}
  We'll prove that $2| n(n+1)$ for every $n\in \N$ ($a|b$ means: $a$ divides $b$).
  \begin{enumerate}
    \item the statement is true for $0$: $2 | 0\cdot 1$ as $0=2\cdot 0$
    \item I need to prove that $(2|n(n+1))\Rightarrow (2|(n+1)(n+2))$ for every $n$. Assume that $n$ is such a number that $2|n(n+1)$.
    Then $(n+1)(n+2)=n(n+1) + 2\cdot (n+1)$ is divisible by 2, what we needed to prove.
  \end{enumerate}
  Therefore from the principle of mathematical induction we know that for all number of the form $n(n+1)$ are divisible by 2\footnote{There is also an alternative proof: it's
  a product of two consecutive numbers - one of them is divisible by 2 and so is the product.}.
\end{example}

\begin{exercise}
	Prove that $2^n>n$ for every natural number $n$.
\end{exercise}

\noindent You can also modify slightly the induction principle - sometimes you should start with number different than 0 or use different induction step
(start 0 and step 2 can lead to theorems valid for even numbers, step 0 and steps 1 and -1 can lead to theorems valid for all integers...)
\begin{exercise}
    \begin{enumerate}
	   \item Prove\footnote{Another method is to notice that $n^3-n=(n-1)\cdot n\cdot (n+1)$. Why 2 does divide it? Why 3?} that 6 divides
		     $n^3-n$ for all natural $n$.
	    \item Prove\footnote{How $n^3-n$ and $(-n)^3-(-n)$ are related? Does this simplify the proof?} that 6 divides $n^3-n$ for all integers $n$.
		      You can use a slight modification mathematical induction principle proving the implication
		      ,,if the theorem works for $n$, it works also for $n-1$".
    \end{enumerate}
\end{exercise}

\begin{exercise}
	(Bernoulli's inequality) Prove that for real $x > -1$ and natural $n\ge 1$, the following inequality holds:
	$$(1+x)^n\ge 1+nx.$$
\end{exercise}

\begin{exercise}
	In Mathsland there are $n\ge 2$ cities. Between each pair of them there is a \textit{one-way} road.
	\begin{enumerate}
		\item Prove that there is a city from which you can drive to all the other cities. Hint: assume that the hypothesis works for some $n$ and any
			country with $n$ cities. Now consider an arbitrary $n+1$-city country. Hide one city and use your assumption.
		\item Prove that there is a city\footnote{Nice trick: what does happen if you reverse each way? Can you use the former result?}
			to which you can drive from all the others.
	\end{enumerate}
\end{exercise}

\begin{exercise}
	Let $S\subseteq R$. We say that $S$ is \textbf{well-ordered} iff any non-empty subset $X\subset S$ has the smallest element.
	\begin{enumerate}
		\item Prove that reals and integers with the default ordering are not well-ordered.
		\item Assume that $X\subseteq \mathbb N$ doesn't have the smallest element. Define $A=\{n\in \mathbb N : \{0,1,\dots,n\}\cap X=\emptyset\}$
			and use mathematical induction to prove that $X$ is empty.
		\item Why are natural numbers well-ordered?
	\end{enumerate}
\end{exercise}



% \section{Cardinality}
\subsection{Finite sets}
\begin{definition}
  The \textbf{cardinality} $|X|$ of a finite set $X$ is defined as the number of elements in $X$.
\end{definition}

\begin{example}
  Let $A=\{0,1,2,3\}$. Then $|A|=4$.
\end{example}

\begin{exercise}
	What is the cardinality of $\{a, a+1, a+2, \dots, a+n\}$?
\end{exercise}

\begin{theorem}
  \textbf{Inclusion-exclusion principle}
  If $X$ and $Y$ are finite sets, then:
  $$|X\cup Y|=|X|+|Y|-|X\cap Y|.$$
\end{theorem}

Intuitively, adding two sets we count elements in each set twice and then subtract the number of elements that were counted twice. The formal proof goes as follows:

\begin{exercise}
  Prove the inclusion-exclusion principle:

  \begin{enumerate}
    \item Let $X$ and $Y$ be finite, disjoint (that is $X\cap Y=\emptyset$) sets. Prove that:
    $$|X\cup Y| = |X| + |Y|.$$
    \item Prove that for $A\subseteq X$, where $X$ is finite, we have $|X\setminus A|=|X|-|A|$. Hint: $X\setminus A$ and $A$ are disjoint and sum up to $X$...
    \item Prove that $$|X\cup Y|=|X|+|Y|-|X\cap Y|$$ for finite sets $X,\, Y$ (now we don't assume that they are disjoint). Hint: what is $(X\setminus(X\cap Y))\cup Y$?
  \end{enumerate}
\end{exercise}

\begin{exercise}
  Prove that if $B\subseteq A$, and $A$ is finite, then $|B|\le |A|$. When does the equality hold?
\end{exercise}

\begin{exercise}
  Prove that $|\mathcal P(A)|=2^{|A|}$ for a finite set $A$. Do you see why the power set $\mathcal P(A)$ is often referenced as $2^A$?
\end{exercise}

\begin{exercise}
    Let $A,B,C$ be finite sets. Prove that:
		$$|A\cup B\cup C| = |A|+|B|+|C| - |A\cap B| - |B\cap C|-|C\cap A| + |A\cap B\cap C|.$$
\end{exercise}

\begin{exercise}
  Let $X=\{1,2,\dots, 2018\}$.
\end{exercise}

\subsection{Characteristic functions}
\begin{definition}
  Fix a set $U$. For each subset $A\subseteq U$ we define it's \textbf{characteristic function} or \textbf{indicator function} as:

  $$1_A: U\to \{0,1\}$$
  $$1_A(x) = \begin{cases}1, \text{ if } x\in A\\ 0, \text{ if } x\notin A\end{cases}$$
\end{definition}

\begin{example}
  Consider a set $U$. Then $1_\emptyset(x)=0$ and $1_U(x)=1$ for every $x\in U$. It's usually abbreviated as:
  $$1_\emptyset=0, 1_U=1.$$
\end{example}

\begin{exercise}
  Let $A,B\subseteq U$. Prove that:
  \begin{enumerate}
    \item $1_{A_\cap B}=1_A\cdot 1_B$\footnote{It means that for every $x\in U$ we have $1_{A_\cap B}(x)=1_A(x)\cdot 1_B(x)$}
    \item $1_{A^c}=1-1_A$, where $A^c=U\setminus A$
    \item $1_{A\cup B}=1_A+1_B-1_A\cdot 1_B$
  \end{enumerate}
\end{exercise}

\begin{exercise}
  Prove inclusion-exclusion principle for finite sets using characteristic functions. Hint: write $1_{A\cup B}$ in terms of $1_A, 1_B, 1_{A\cap B}$ and sum it's values over
  all elements in \textit{finite} set $A\cup B$.
\end{exercise}

\subsection{Comparing cardinalities}
Although we feel comfortable in counting elements of \textit{finite} sets, we don't know how to say how to compare infinite sets - there is no natural number we could use to denote
their cardinalities!

Therefore, we'll try another approach. Assume that we have a set of children and a set of toys. If we want to compare them, we can either try to calculate how many children and toys there are (it may be very hard if there are lots of children and lots of toys) or to ask each child to get one toy. If every child has \textit{one} toy and no toys are left, we know
that there are exactly as many children as toys! We'll use this approach to compare infinite sets.

\begin{definition}
  Let $A$ and $B$ be two sets. If there exists a bijection $f:A\to B$, we say that $|A|=|B|$ (are of the same cardinality).
\end{definition}

\begin{example}
  $|\N|=|2\N|$, where $2\N$ is a set of all even natural numbers, as we can find a bijection $n\mapsto 2n$. It's a surprising result, as $2\N\subseteq \N$ is a \textit{proper} subset. If $\N$ was finite, all it's proper subsets would have smaller cardinalities!
\end{example}

\begin{exercise}
  Being of the same cardinality has similar properties to these of equivalence relation\footnote{... but as there is no sets of all sets, it is not formally an equivalence relation.}. Prove that:
  \begin{enumerate}
    \item $|A|=|A|$
    \item $|A|=|B|$ implies that $|B|=|A|$ (hint: bijections have inverses)
    \item if $|A|=|B|$ and $|B|=|C|$, then $|A|=|C|$ (hint: what is a composition of bijections?)
  \end{enumerate}
\end{exercise}

\begin{definition}
  We say that a set $X$ is \textbf{countably infinite} if $|X|=|\N|$. Usually we'll write that $\aleph_0:=|\N|$ (read "aleph 0").
  We say that a set $X$ is \textbf{countable} if $X$ is finite or countably infinite.
\end{definition}

\begin{example}
  Sets $\N, 2\N, \{0,1,6, 41\}$ are countable.
\end{example}

\begin{example}
  $\Z$ is countable: $0\mapsto 0, 1\mapsto 1, 2\mapsto -1, 3\mapsto 2, 4\mapsto -2,\dots$
\end{example}

\begin{exercise}
  Prove that a subset of a countable set is countable.
\end{exercise}

\begin{exercise}
  Let $A$ and $B$ be countable sets. Prove that $A\cup B$ and $A\cap B$ are countable.
\end{exercise}

\begin{exercise}
  Let $A$ and $B$ be countable. Prove that $A\times B$ is countable. Hint: you can write all elements of $A$ as $a_1,a_2,\dots$ and the elements of $B$ as $b_1, b_2,\dots$.
  Think about an ordering $(a_1,b_1); (a_1, b_2), (a_2, b_1); (a_1, b_3), (a_2, b_2), (a_3, b_1); \dots$ (some terms may be repeated if $A$ and $B$ are not disjoint, think how to fix it).
\end{exercise}

\begin{exercise}
  Prove that $\Q$ is countable.
\end{exercise}

\begin{exercise}
  Let $\mathcal A$ be a countable family of countable sets. Prove that $\bigcup \mathcal A$ is countable.
\end{exercise}

\begin{exercise}
  Prove that is $X$ is an infinite subset, then it contains a countably-infinite subset $S\subseteq X, |S|=\aleph_0$.
\end{exercise}

The last exercise shows that we can compare cardinalities. That is, if we can find a bijection between $A$ and \textit{some subset} of $B$, we can be sure that $B$ contains at least
as many elements as $A$. This is exactly requiring the existence of an \textit{injection} from $A\to B$.

\begin{definition}
  If there exists an injection $f:A\to B$ we say that $B$ has greater or equal cardinality than $A$ and write $|A|\le |B|$. If $|A|\le |B|$ and $|A|\neq |B|$, we write
  $|A| < |B|$ (that is: we can find an injection from $A$ to $B$, but there is no bijection between them).
\end{definition}

% \begin{exercise}
% 	Above we find the way of saying that two cardinalities are equal using existence of a bijection. Let's find a way to compare which is less using
% 	another kind of function.
% 	\begin{enumerate}
% 		\item Let $O_n=\{1,2,\dots,n\}.$ Prove that there is no injection from $O_{n+1}$ into $O_n$. Hint: use mathematical induction.
% 		\item Let $A$ and $B$ be finite. Prove that there is an injection from $A$ to $B$ iff $|A| \le |B|.$
% 	\end{enumerate}
% \end{exercise}
%
% \begin{exercise}
% 	Prove in one: if there is an injection from $A$ onto $B$ and an injection from $B$ into $A$, then there exists a bijection from $A$ onto $B$. Hint: you know that $|A|\le B$ and $|B|\le |A|$.
% \end{exercise}


\begin{exercise}
	Let $A,\,B$ and $C$ be sets. Prove that if $|A|\le |B|$ and $|B|\le |C|$, then $|A|\le |C|$.
\end{exercise}

\begin{prob}
	Here you can prove that there are more real numbers than naturals or rationals. We define $X=\{x\in \mathbb R : 0\le x\le 1\}$ and choose one
	 convention of writing reals (e.g 0.999... = 1.000..., so we can choose to use nines)
	\begin{enumerate}
		\item Assume that you have written all the elements of $X$ in a single column. Can you find a real number that does not occur in the list?
		\item Using the above, prove that $|\mathbb N| < |X|$
		\item Prove that $|\mathbb Q| < |\mathbb R|.$
	\end{enumerate}
\end{prob}

\begin{prob}
	We know that $|\mathbb R| > |\mathbb N|.$ Using binary system prove that $\mathbb R=\mathcal P(\mathbb N)$. Do you see similarity between the previous result
	and $2^n > n$ for natural $n$?
\end{prob}

\begin{prob}
	\textbf{Cantor's theorem} You will prove that $|A|<\left|\mathcal P(A)\right|$ for any set $A$. Let $A$ be a set and $f:A\to \mathcal P(A).$
	\begin{enumerate}
		\item Consider $X=\{a\in A : a\notin f(a)\}\in \mathcal P(A)$. Is there $x\in A$ for which $f(x)=X?$
		\item Is $f$ surjective?
		\item Find an injective function $g: A\to \mathcal P(A)$.
		\item Prove that $|A| < |\mathcal P(A)|$ for any set $A$.
		\item Use Cantor's theorem to prove that there is no set of all sets.
	\end{enumerate}
\end{prob}

\begin{prob}
	\textbf{Cantor-Schroeder-Bernstein theorem} Let's prove that if $|A|\le|B|$ and $|B|\le |A|$, then $|A|=|B|$.
	\begin{enumerate}
		\item (Knaster-Tarski) Now assume that $F$ has \textit{monotonicity} property: $F(X)\subseteq F(Y)$ if $X\subseteq Y$.
			Prove that $F$ has a fixed point $S$ (that is $F(S)=S$), where:
			$$S=\bigcup_{X\in U} X, \text{~where~} U= \{Y\in 2^A : Y\subseteq f(Y)\}.$$
		\item (Banach) Let $f: A\to B$ and $g:B\to A$ be injections.
			We introduce new symbol: $f[X]=\{b\in B : b=f(x) \text{ for some } x\in X\}$. Prove that
			function $$F:\mathcal P(A)\to \mathcal P(A),~F(X)=A\setminus g[B\setminus f[X]]$$
			has the monotonicity property.
    \item Prove that $A\setminus S\subseteq \text{Im}\,g$, where
      $F$ and $S$ are taken from above.
		\item Prove that function
			$$h(x) =
				\begin{cases}
					f(x), x\in S\\
					g^{-1}(x), x \notin S
				\end{cases}
			 $$
			 is a bijection.
	\end{enumerate}
\end{prob}


% \section{Pre-image of a function}
Let $f:A\to B$ and $C\subseteq A$. We used $f[C]$ for a set:
$$f[C] = \{ f(c) \in B : c\in C\},$$
but now we will abuse a bit our notation to stick to
the common nomenclature. Apparently, many mathematicians write:
$$f(C) = \{ f(c) \in B : c\in C\}.$$
This is not correct - as $f$ should take elements
$a\in A$ and returns elements $b\in B$, but here $f$ ,,takes"
a subset $C\subseteq A$ and returns a set $f(C)\subseteq B$. We will
follow this notation, but you should always check what meaning
the object feed to function has (whether it is an element or a subset).

\begin{prob}
  Let $f:A\to B$ and $X,Y\subseteq B$. Then:
  \begin{enumerate}
    \item $f(X\cup Y)=f(X)\cup f(Y)$
    \item $f(X\cap Y)\subseteq f(X)\cap f(Y)$
  \end{enumerate}
  You can also generalise this result to an arbitrary collection of
  sets.
\end{prob}

\noindent To even more abuse the notation, we will also give an additional meaning to $f^{-1}$. As we know, many functions $f$ \textit{don't} have inverses. But we will write for $D\subseteq B$:
$$f^{-1}(D) = \{a \in A : f(a)\in D\}\subseteq A.$$

We then say that $f^{-1}(D)$ is the \textbf{pre-image} of $D$.
To get accustomed with this notation, prove that:

\begin{prob}
  Let $f:A\to B$. Then $f(A)\subseteq B$ and $A=f^{-1}(B)$.
\end{prob}

You should also prove:
\begin{prob}
  Let $f:A\to B$ and $X,Y\subseteq B$. Then:
  \begin{enumerate}
    \item $f^{-1}(X\cup Y)=f^{-1}(X)\cup f^{-1}(Y)$
    \item $f^{-1}(X\cap Y)=f^{-1}(X)\cap f^{-1}(Y)$
  \end{enumerate}
  You can also generalise this result to an arbitrary collection of
  sets.
\end{prob}
Therefore, we see that although $f$ does \textit{not} preserve the
set structure, $f^{-1}$ does. This observation is crucial, we will later use it in topology and measure theory.

% \section{Numbers}

At the beginning we assumed that you know how to work with different kinds of numbers. This aim of this section is to give a few properties of them.

\subsection{Real numbers}
\begin{definition}
  A \textbf{field} is a tuple $(\F, +, \cdot)$ such that:
  \begin{itemize}
    \item $\F$ is a set
    \item $+$ and $\cdot$ are functions from $\F\times \F$ to $\F$. We write
      $a+b$ for $+(a,b)$ and $a\cdot b$ for $\cdot (a,b)$.
  \end{itemize}
  They have these properties:
  \begin{enumerate}
    \item $a+(b+c)=(a+b)+c$ for all $a,b,c$ (addition is associative)
    \item $a\cdot (b\cdot c)=(a\cdot b)\cdot c$ for all $a,b,c$ (multiplication is associative)
    \item $a+b=b+a$ for all $a, b$ (addition is commutative)
    \item $a\cdot b=b\cdot a$ for all $a,b$ (multiplication is commutative)
    \item there is an element $0\in\F$ such that $a+0=a$ for all $a$ (addition has a neutral element). We call it \textbf{zero}
    \item there is an element $1\in\F$ such that $a\cdot 1=a$ for all $a$ (so 1 is neutral element of multiplication). We call if \textbf{one}
    \item for every $a$ there is $a'$ such that $a+a'=0$ (existence of
      an inverse element for addition)
    \item $a\cdot (b+c) = a\cdot b + a\cdot c$ for all $a,b, c$
      (multiplication distributesover addition)
    \item for every $a\neq 0$ there is $\tilde a$ such that $a\cdot \tilde a=1$
      (multiplication has an inverse element for all non-zero numbers)
    \item $1\neq 0$ (so $\F$ has at least two elements)
  \end{enumerate}

  Usually we will reference a field just as $\F$.
\end{definition}

\begin{exercise}
  Check that
  \begin{enumerate}
    \item real numbers understood informally, have the field properties
    \item rational numbers form a field
  \end{enumerate}
\end{exercise}

From the above field axioms, you can derive many facts that may be obvious
to you:
\begin{exercise}
  Prove that there is 0 and 1 are unique. Hint: assume that
  0 and 0' have property such that $a=a+0=a+0'$ and try $a=0$ and $a=0'$.
\end{exercise}

\begin{exercise}
  Prove that if $a+a'=0$ and $a+a''=0$, then $a'=a''$.
  Therefore we can introduce special symbol for \textit{the} additiv
  inverse: $a + (-a)=0$ and define subtraction as $a-b := a + (-b)$.
\end{exercise}

\begin{exercise}
  Prove that $-a=(-1)\cdot a$.
\end{exercise}

% Consider more problems on this topic

As you see, many of the algebraic properties we are used to can be recovered from the axioms, but sometimes it can be complicated. Both real numbers and
rational numbers have also an order on them - for example $2>1$. It leads
to the definition of \textit{total order}.

\begin{definition}
  We call a pair $(\F, \le)$ a \textit{totally ordered set} if for every $a,b\in \F$ we have:
  \begin{enumerate}
    \item $a\le b$ or $b\le a$ (we call this property totality)
    \item $a\le b$ and $b\le a$ imply $a=b$ (it's called antisymmetry)
    \item $a\le b$ and $b\le c$ imply $a\le c$ (transitivity)
  \end{enumerate}
  Having relation $\le$ we can define others: $b\ge a$ means that $a\le b$ and
  $a<b$ means that $a\le b$ and $a\neq b$.
\end{definition}

\begin{definition}
  \textbf{Ordered field} is a field $\F$ with a total order such that:
  \begin{itemize}
    \item $a\le b$ implies $a+c\le b+c$
    \item $0\le a$ and $0\le b$ imply that $0\le a\cdot b$
  \end{itemize}
  We say that $\F_+ := \{x\in \F : x \le 0\}$ is a set of \textbf{non-negative} numbers.
\end{definition}

\begin{exercise}
  Check that non-negative numbers are closed under summation and multiplication (the sum and product of two non-negative numbers is non-negative).
\end{exercise}

\begin{definition}
  Consider a subset of a totally-ordered set $A\subseteq \F$. We say that $x$ is an
  \textbf{upper bound} of $A$ iff $x\ge a$ for every $a\in A$.
  In such case we say that $A$ \textbf{is bounded from above}.
\end{definition}

\begin{exercise}
  Let $\F$ be either $\R$ or $\Q$. Prove that a set $A\subseteq \mathbb \F$ can have no upper bounds
  or infinitely many of them.
\end{exercise}

\begin{definition}
  Let $A\subseteq \F$ be a set bounded from above. It's \textbf{supremum} or \textbf{lower upper bound} is a number $\sup A\in F$ such that:
  \begin{itemize}
    \item $\sup A$ is an upper bound of $\F$
    \item if $y$ is an upper bound of $A$, then $y\ge \sup A$
  \end{itemize}
  Supremum is often referenced as $\text{lub}$ or $\text{l.u.b}$.
\end{definition}

\begin{exercise}
  Prove that supremum is unique, so if $x$ and $x'$ are supremums
  of $A$, then $x=x'$.
\end{exercise}

\begin{exercise}
  Prove that $x=\sup A$ if and only if
  $x\ge a$ for every $a\in A$ and for every $\eps > 0$ there is
  $a\in A$ such that $x < a + \eps$.
\end{exercise}

\begin{definition}
  \textbf{Completeness axiom} - each non-empty, bounded from above subset of $\R$ has a supremum.
\end{definition}

This axiom allows us to prove many interesting things:

\begin{exercise}
  Prove that natural numbers are \textit{not} bounded from above.
  Hint: if $n\in \mathbb N$, then $n+1\in \mathbb N$
\end{exercise}

\begin{exercise}
  Prove the \textbf{Archimedean axiom}\footnote{In fact we do not
  need to call it axiom, as we are able to prove it.}
  that for every $r\in R$, there is $n\in \mathbb N$ such that $n>r$.
\end{exercise}

\begin{exercise}
  Prove that for any $r>0$ there is $n\in \mathbb N$ such that
  $1/n < r$.
\end{exercise}

\begin{exercise} Find infinite sum and intersection for the families of subsets of $\mathbb{R}$:
  \begin{enumerate}
    \item $A_i=(0,1/i)$ for $i=1,2,\dots$
    \item $B_i=[0,1/i)$ for $i=1,2,\dots$
  \end{enumerate}
\end{exercise}

\begin{exercise}
  Prove that rational numbers do \textit{not} have the completeness
  property:
  \begin{enumerate}
    \item Let $p, q\in \mathbb Z\setminus \{0\}$.
      Prove that $p^2\neq 2q^2$.
    \item Prove that root of two, defined as
      $x > 0, x^2=2$ is not rational.
    \item Find a subset of $\mathbb Q$ that is bounded above, but
      has no rational supremum.
  \end{enumerate}
\end{exercise}

\begin{exercise}
  You should prove that in each nonempty interval there is at least
  one rational number:
  \begin{enumerate}
    \item Assume that $0<a<b$. Define
      $$A=\left\{\frac m N : m\in \mathbb N\right\},~
      \frac 1{b-a} < N \in \mathbb N$$
      and prove that $A\cap (a,b)$ is non-empty.
    \item Use the above result to prove that in \textit{each}
      interval there is at least one rational number.
    \item Prove that in each interval there are infinitely but countably many, rational numbers.
    \item Prove that in each interval there is an irrational number.
    \item How many irrational numbers are in each interval?
  \end{enumerate}
\end{exercise}

\subsection{Absolute value}
Another concept that will be further useful is the \textbf{absolute value} of a real number:
if $x\in \mathbb R$ we write $|x|\in \mathbb R$ for:
$$|x| = \begin{cases}x &\text{ for } x \ge 0\\ -x &\text{ otherwise} \end{cases}.$$

\begin{exercise}
  Prove that for every $x,y\in \mathbb R$:
  \begin{enumerate}
    \item $|x|=|-x|$
    \item if $|x|=|y|$ then $x=y$ or $x=-y$.
    \item $|x+y| \le |x| + |y|$ (this is called \textbf{triangle inequality})
    \item $|x-y|\le |x| + |y|$
    \item $\left||x| - |y|\right|\le |x-y|$ (this is sometimes calles \textbf{reverse triangle inequality})
  \end{enumerate}
\end{exercise}

\subsection{Mathematical induction}
We characterise natural numbers as a subset of $\R$ with some properties.

\begin{definition}
  Let $A\subseteq \R$ be a set such that:
  \begin{itemize}
    \item $0\in A$
    \item if $n\in A$, then $n+1\in A$
  \end{itemize}
  The intersection of all sets with this property is called \textbf{the set of natural numbers}. Alternatively, natural numbers is the smallest set (in sense of subset order)
  with the properties listed above.
\end{definition}

These properties allow us to define a powerful proof technique:

\begin{definition}
  \textbf{Principle of mathematical induction} -
  let $f$ be a logical statement defined for all natural numbers (that is $f(n)$ is either true or false). Then if $f(0)$ is true and an implication:
  $$f(n)\Rightarrow f(n+1)$$
  is true for every $n$, then $f(n)$ is true for every $n$.
\end{definition}

Have you ever seen falling dominoes? To be sure that all domino falls, we need to:
\begin{enumerate}
	\item punch the first domino
	\item every domino must punch the next domino (if this particular domino falls, the next one also falls)
\end{enumerate}
And that's all, we can be sure that all the dominoes will eventually fall.

\begin{example}
  We'll prove that $2| n(n+1)$ for every $n\in \N$ ($a|b$ means: $a$ divides $b$).
  \begin{enumerate}
    \item the statement is true for $0$: $2 | 0\cdot 1$ as $0=2\cdot 0$
    \item I need to prove that $(2|n(n+1))\Rightarrow (2|(n+1)(n+2))$ for every $n$. Assume that $n$ is such a number that $2|n(n+1)$.
    Then $(n+1)(n+2)=n(n+1) + 2\cdot (n+1)$ is divisible by 2, what we needed to prove.
  \end{enumerate}
  Therefore from the principle of mathematical induction we know that for all number of the form $n(n+1)$ are divisible by 2\footnote{There is also an alternative proof: it's
  a product of two consecutive numbers - one of them is divisible by 2 and so is the product.}.
\end{example}

\begin{exercise}
	Prove that $2^n>n$ for every natural number $n$.
\end{exercise}

\noindent You can also modify slightly the induction principle - sometimes you should start with number different than 0 or use different induction step
(start 0 and step 2 can lead to theorems valid for even numbers, step 0 and steps 1 and -1 can lead to theorems valid for all integers...)
\begin{exercise}
    \begin{enumerate}
	   \item Prove\footnote{Another method is to notice that $n^3-n=(n-1)\cdot n\cdot (n+1)$. Why 2 does divide it? Why 3?} that 6 divides
		     $n^3-n$ for all natural $n$.
	    \item Prove\footnote{How $n^3-n$ and $(-n)^3-(-n)$ are related? Does this simplify the proof?} that 6 divides $n^3-n$ for all integers $n$.
		      You can use a slight modification mathematical induction principle proving the implication
		      ,,if the theorem works for $n$, it works also for $n-1$".
    \end{enumerate}
\end{exercise}

\begin{exercise}
	(Bernoulli's inequality) Prove that for real $x > -1$ and natural $n\ge 1$, the following inequality holds:
	$$(1+x)^n\ge 1+nx.$$
\end{exercise}

\begin{exercise}
	In Mathsland there are $n\ge 2$ cities. Between each pair of them there is a \textit{one-way} road.
	\begin{enumerate}
		\item Prove that there is a city from which you can drive to all the other cities. Hint: assume that the hypothesis works for some $n$ and any
			country with $n$ cities. Now consider an arbitrary $n+1$-city country. Hide one city and use your assumption.
		\item Prove that there is a city\footnote{Nice trick: what does happen if you reverse each way? Can you use the former result?}
			to which you can drive from all the others.
	\end{enumerate}
\end{exercise}

\begin{exercise}
	Let $S\subseteq R$. We say that $S$ is \textbf{well-ordered} iff any non-empty subset $X\subset S$ has the smallest element.
	\begin{enumerate}
		\item Prove that reals and integers with the default ordering are not well-ordered.
		\item Assume that $X\subseteq \mathbb N$ doesn't have the smallest element. Define $A=\{n\in \mathbb N : \{0,1,\dots,n\}\cap X=\emptyset\}$
			and use mathematical induction to prove that $X$ is empty.
		\item Why are natural numbers well-ordered?
	\end{enumerate}
\end{exercise}

% \section{Real numbers}
At the beginning we assumed that you had some intuition what real numbers
are and how to work with them - to provide examples and make set theory less
abstract. But we have not treated them rigorously, as we did not have proper
glossary - it's high time we filled this gap and defined them properly.
It's high time we defined them properly, as we .
A \textbf{field} is a tuple $(F, +, \cdot, 1, 0).$ We have
many symbols there, let's explain what they mean:
\begin{itemize}
  \item $F$ is a set
  \item $+$ and $\cdot$ are functions from $F^2$ to $F$. We write
    $a+b$ for $+(a,b)$ and $a\cdot b$ for $\cdot (a,b)$.
  \item $1, 0\in F$ are just distinguished elements of $F$
\end{itemize}
We know what objects are in the definition, so we can talk about
properties they must have to form a field:

\begin{enumerate}
  \item $1\neq 0$ (so $F$ has at least two elements)
  \item $a+(b+c)=(a+b)+c$ for all $a,b,c$ (addition is associative)
  \item $a\cdot (b\cdot c)=(a\cdot b)\cdot c$ for all $a,b,c$ (multiplication is associative)
  \item $a+b=b+a$ for all $a, b$ (addition is commutative)
  \item $a\cdot b=b\cdot a$ for all $a,b$ (multiplication is commutative)
  \item $a+0=a$ for all $a$ (so 0 is neutral element of addition)
  \item $a\cdot 1=a$ for all $a$ (so 1 is neutral element of multiplication)
  \item for every $a$ there is $a'$ such that $a+a'=0$ (existence of
    an inverse element for addition)
  \item $a\cdot (b+c) = a\cdot b + a\cdot c$ for all $a,b, c$
    (multiplication distributesover addition)
  \item for every $a\neq 0$ there is $\tilde a$ such that $a\cdot \tilde a=1$
    (multiplication has an inverse element for all non-zero numbers)
\end{enumerate}

\begin{prob}
  Check that
  \begin{enumerate}
    \item real numbers understood informally, have the properties
      listed above
    \item rational numbers form a field
  \end{enumerate}
\end{prob}

From the above field axioms, you can derive many facts that may be obvious
to you:

\begin{prob}
  Prove that there is only one 0 and only one 1. Hint: assume that
  0 and 0' have property such that $a=a+0=a+0'$ and try $a=0$ and $a=0'$.
\end{prob}

\begin{prob}
  Prove that if $a+a'=0$ and $a+a''=0$, then $a'=a''$.
  Therefore we can introduce special symbol for \textit{the} additiv
  inverse: $a + (-a)=0$ and define subtraction as $a-b := a + (-b)$.
\end{prob}

\begin{prob}
  Prove that $-a=(-1)\cdot a$.
\end{prob}

% Consider more problems on this topic

As you see, many of the algebraic properties we are used to can be recovered from the axioms, but sometimes it can be complicated. Both real numbers and
rational numbers have also an order on them - for example $2>1$. It leads
to the definition of \textit{total order}. We call a pair $(F, \le)$ a \textit{totally ordered set} if for every $a,b\in F$ we have:
\begin{enumerate}
  \item $a\le b$ or $b\le a$ (we call this property totality)
  \item $a\le b$ and $b\le a$ imply $a=b$ (it's called antisymmetry)
  \item $a\le b$ and $b\le c$ imply $a\le c$ (transitivity)
\end{enumerate}
Having relation $\le$ we can define other: $b\ge a$ means that $a\le b$ and
$a<b$ means that $a\le b$ and $a\neq b$.

We say that tuple $(F, +, \cdot, 1, 0, \le)$ is \textbf{ordered field} if:
\begin{itemize}
  \item $(F, +, \cdot, 1, 0)$ is a field
  \item $(F, \le)$ is totally ordered
  \item $a\le b$ implies $a+c\le b+c$
  \item $0\le a$ and $0\le b$ imply that $0\le a\cdot b$
\end{itemize}

You can check that reals and rationals are ordered fields. These axioms give
us much more abilities, for example one is able to prove that $1>0$.
But we still have no difference in properties that distuingish rationals
from reals. This is called the completeness axiom and we will need a few
more definitions.

Consider $A\subseteq \mathbb R$. We say that $x$ is an
\textbf{upper bound} of $A$ iff $x\ge a$ for every $a\in A$.

\begin{prob}
  Prove that a set $A\subseteq \mathbb R$ can have no upper bounds
  or infinitely many of them.
\end{prob}

\noindent If an upper bound of $A$ exists, we say that $A$ is
\textbf{bounded from above}. Among them we will distinguish the
\textbf{supremum} (or \textbf{the least upper bound - l.u.b}):
$x=\sup A$ iff $x$ is an upper bound of $A$ and for any upper bound
$y$ of $A$ we have $x\le y$.

\begin{prob}
  Prove that supremum is unique, so if $x$ and $x'$ are supremums
  of $A$, then $x=x'$.
\end{prob}

\begin{prob}
  Prove that $x=\sup A$ if and only if
  $x\ge a$ for every $a\in A$ and for every $\eps > 0$ there is
  $a\in A$ such that $x < a + \eps$.
\end{prob}

\noindent Now we can state the \textbf{completeness axiom}:
each non-empty and bounded from above subset of real numbers has
a supremum.
This axiom allows us to prove many interesting things:

\begin{prob}
  Prove that natural numbers are \textit{not} bounded from above.
  Hint: if $n\in \mathbb N$, then $n+1\in \mathbb N$
\end{prob}

\begin{prob}
  Prove the \textbf{Archimedean axiom}\footnote{In fact we do not
  need to call it axiom, as we are able to prove it.}
  that for every $r\in R$, there is $n\in \mathbb N$ such that $n>r$.
\end{prob}

\begin{prob}
  Prove that for any $r>0$ there is $n\in \mathbb N$ such that
  $1/n < r$.
\end{prob}

\begin{prob} Find infinite sum and intersection for the families of subsets of $\mathbb{R}$:
  \begin{enumerate}
    \item $A_i=(0,1/i)$ for $i=1,2,\dots$
    \item $B_i=[0,1/i)$ for $i=1,2,\dots$
  \end{enumerate}
\end{prob}

\begin{prob}
  Prove that rational numbers do \textit{not} have the completeness
  property:
  \begin{enumerate}
    \item Let $p, q\in \mathbb Z\setminus \{0\}$.
      Prove that $p^2\neq 2q^2$.
    \item Prove that root of two, defined as
      $x > 0, x^2=2$ is not rational.
    \item Find a subset of $\mathbb Q$ that is bounded above, but
      has no rational supremum.
  \end{enumerate}
\end{prob}

\begin{prob}
  You should prove that in each nonempty interval there is at least
  one rational number:
  \begin{enumerate}
    \item Assume that $0<a<b$. Define
      $$A=\left\{\frac m N : m\in \mathbb N\right\},~
      \frac 1{b-a} < N \in \mathbb N$$
      and prove that $A\cap (a,b)$ is non-empty.
    \item Use the above result to prove that in \textit{each}
      interval there is at least one rational number.
    \item Prove that in each interval there are infinitely but countably many, rational numbers.
    \item Prove that in each interval there is an irrational number.
    \item How many irrational numbers are in each interval?
  \end{enumerate}
\end{prob}

\subsection{Absolute value}

Another concept that will be further useful is the \textbf{absolute value} of a real number:
if $x\in \mathbb R$ we write $|x|\in \mathbb R$ for:
$$|x| = \begin{cases}x &\text{ for } x \ge 0\\ -x &\text{ otherwise} \end{cases}.$$

\begin{prob}
  Prove that for every $x,y\in \mathbb R$:
  \begin{enumerate}
    \item $|x|=|-x|$
    \item if $|x|=|y|$ then $x=y$ or $x=-y$.
    \item $|x+y| \le |x| + |y|$ (this is called \textbf{triangle inequality})
    \item $|x-y|\le |x| + |y|$
    \item $\left||x| - |y|\right|\le |x-y|$ (this is sometimes calles \textbf{reverse triangle inequality})
  \end{enumerate}
\end{prob}

% \subsection{Categories}
As you have had training in sets and functions, we are able to introduce some category theory that will quickly become useful - most of the objects in mathematics form a
category and this language will be extremely convenient to find similarities between different branches of mathematics.
As we remember, it is not possible to create a set of all sets without having a contradiction. So let's use a word \textbf{collection} of sets or \textbf{class}
\footnote{Formal treatment of classes - collections that are in some sense bigger than sets - is introduced in von Neumann–Bernays–Gödel and Morse-Kelley set theories.}
 of sets - that is not a set and we don't know how to express it formally - but what has an intuitive sense.

\begin{definition}
  A \textbf{category} is:
  \begin{enumerate}
    \item a collection of objects such that
    \item for each pair objects $A$, $B$ in the collection there is a \emph{set} $\Hom(A,B)$ called the set of \textbf{morphisms} or \textbf{maps} of \textbf{arrows} such that
    \item for morphisms $f\in \Hom(A,B), g\in \Hom(B,C)$ there is a morphism $g\circ f\in \Hom(A,C)$. We also require:
      \begin{itemize}
        \item that composition of morphisms is associative: $h\circ(g\circ f)=(h\circ g)\circ f$, where $h\in \Hom(C, D)$
        \item we have morphisms $\Id_X: X\to X$ for every object $X$ such that $f\circ\Id_A = f=\Id_B\circ f$ for $f\in \Hom(A,B)$
      \end{itemize}
  \end{enumerate}
  If $f\in \Hom(A,B)$, we can also write $f:A\to B$ or $A\xrightarrow{f} B$. Some authors also write $\text{Mor}(A,B)$ for $\Hom(A,B)$ and $gf$ for $g\circ f$. If the collection of objects happens
  to be a set, we call it \textbf{small category}.
\end{definition}

\begin{example}
  We already know very well a category - the category of sets and functions. Let's check carefully that is actually is a category:
  \begin{enumerate}
    \item objects are just sets
    \item take $\Hom(A,B)$ as a set of all functions from $A$ to $B$ (why is it a set?)
    \item define composition of morphisms just as function composition
      \begin{itemize}
        \item composition of functions is associative (recall why)
        \item identity morhpism is just identity function of a set
      \end{itemize}
  \end{enumerate}
\end{example}

\begin{exercise}
  Consider a category with one singleton: $\{\{0\}\}$, where $\{0\}$ is the only object, and functions as arrows. How many arrows are in this category?
\end{exercise}

\begin{exercise}
  Consider a category with two singletons: $\{\{0\}, \{1\}\}$, and functions as arrows. How many arrows can be in this category? Hint: 4 numbers.
\end{exercise}

\subsection{Morphisms and commutative diagrams}

