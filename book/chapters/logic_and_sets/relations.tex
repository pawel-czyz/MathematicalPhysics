\section{Relations}
Having defined Cartesian product, we can consider subsets of it. It will lead to two new, important concepts - relations and functions.

\begin{definition}
  A \textbf{relation $R$ between sets} $X$ and $Y$ is a subset of $X\times Y$. If $(x,y)\in R$ we write $x\,R\,y$. A \textbf{relation on a set} $X$ is a subset of $X\times X$.
\end{definition}

\begin{example}
  Consider the order of natural numbers (that is $0<1, \,1<2,\,2<3$ and so on). It is in fact a relation on $\N$: $a<b$ means exactly $(a,b)\in\, <\,\subseteq \N\times \N$ and is defined as:
  $$< := \bigcup_{n\in \N}\bigcup_{i\in \Z^+} \{(n, n+i)\}, \text{ where } \Z^+=\{n\in \N : n\neq 0\}.$$
\end{example}

\begin{exercise}
  What is "the smallest" relation between $X$ and $Y$ (in such sense that is a subset of \emph{every} relation between $X$ and $Y$)? What is "the biggest" one (every relation is a subset of the biggest one)?
\end{exercise}

\begin{exercise}
 Let $X$ and $Y$ be any sets. Prove that there exists the \textbf{set} of all relations between $X$ and $Y$. [Hint\footnote{Use the power set.}]
\end{exercise}

\begin{exercise}
  Let $X$ and $Y$ be finite sets. How many relations can be defined between them?
\end{exercise}

Among all the relations on a set $X$, we have some with very nice behaviour.

\begin{definition}
  Let $\equiv$ be a relation on $X$. We say that it is an \textbf{equivalence relation} if all of the following hold:
  \begin{enumerate}
    \item if $x\equiv y$ and $y\equiv z$, then also $x\equiv z$ (transitivity)
    \item if $x\equiv y$, then $y\equiv x$ (symmetry)
    \item $x\equiv x$ for every $x$ (reflexivity)
  \end{enumerate}
\end{definition}

\begin{example}
  Consider any set $X$. Then a set $$\Id_X := \{(x,x)\in X\times X : x\in X\}$$
  is an equivalence relation on $X$.
\end{example}

\begin{exercise}
  Prove that $n\equiv m$ iff $n$ and $m$ have the same parity is an equivalence relation on $\Z$.
\end{exercise}

As you may have noticed, using the equivalence relation with partition the set into some subsets.

\begin{definition}
  Let $X\neq \emptyset$ be a set. We say that a family of subsets $\mathcal A\subseteq \mathcal P(X)$ \textbf{partitions} X iff:
  \begin{enumerate}
    \item $\emptyset \neq X$
    \item $\bigcup \mathcal A=X$ (every element is somewhere)
    \item for $A,A'\in \mathcal A$ we have either $A=A'$ or $A\cap A'=\emptyset$ (partitioning sets are pairwise disjoint)
  \end{enumerate}
  Elements of $\mathcal A$ are called \textbf{equivalence classes}. If $a\in A\in\mathcal A$, we write $[a]:=A$.
\end{definition}

Why do we call it equivalence classes? Is it somehow related to equivalence relations?

\begin{exercise}
  Here you will prove the fundamental relationship between partitions and equivalence relations.
  \begin{enumerate}
    \item Prove that if we have a parition on $X$, then the relation given by: $x\equiv y$ iff $x$ and $y$ belong to the same equivalence class, is an equivalence relation on $X$.
    \item Let $\equiv$ be an equivalence relation on $X$. Prove that $\{[x] : x\in X\}$ is a partition on $X$, where $[x]=\{y\in X : y\equiv x\}$
  \end{enumerate}
  The partition of $X$ corresponding to relation $\equiv$ is written as $X/\equiv.$
\end{exercise}

\begin{exercise}
  Consider an equivalence relation $\equiv$.
  \begin{enumerate}
    \item Prove that $[a]=[b]$ iff $a\equiv b$.
    \item Prove that $[a]\cap [b]=\emptyset$ iff $a\not\equiv b$.
  \end{enumerate}
  This means that equivalence classes can be either identical or disjoint (what is not surprising as equivalence classes form a partition).
\end{exercise}

\begin{exercise}
  Let $X$ be a set with $n$ elements and $q$ be the number of possible equivalence classes on $X$. Prove that $$n\le q \le 2^{n^2}-1.$$ [Hint\footnote{For $n\ge 2$ construct $n$ equivalence relations with two classes.}]
\end{exercise}

Usually our sets will be equipped with some additional structure - for example integers can be added together. Sometimes we can move this structure to the equivalence classes. Let's start by finding a nice equivalence class on them.

\begin{example}
  \textbf{Modulo arithmetics}
  Let $p$ and $q$ be integers. $p\,|\,q$ means that $p$ divides $q$ (there exists a $m\in \Z$ such that $q=pm$). We fix a non-zero number $p\in \Z$ and define \textbf{equivalence modulo $p$}:
  $$m\equiv_p n \Leftrightarrow p\,|\,m-n.$$

  It's easy to check that this is an equivalence relation. We would like to define a sum on the set of equivalence classes. Let's try to do this intuitively:
  $$[m] + [n] := [m+n].$$

  Although it looks right, we need to check whether this definition is independent on the chosen representatives! So let's $m\equiv_p m'$ and $n\equiv_p n'$. We would like to show
  that $m+n\equiv_p m'+n'$. In other words, we want $p$ to divide $(m+n)-(m'+n')$, what is true as $(m+n)-(m'+n')=(m-m')+(n-n')$, that is a sum of numbers divisible by $p$.
\end{example}

Analogously one can define multplication and subtraction to get the modulo arithmetics known from elementary number theory.

\begin{exercise}
  \textbf{Construction of rationals}
  \begin{enumerate}
    \item Let $\Z^*=\Z\setminus\{0\}$. Consider $X=\Z\times \Z^*$. Prove that relation $\equiv$ on $X$ given as: $(m,n)\equiv (p,q)\Leftrightarrow mq=pn$ is an equivalence relation.
    \item To simplify notation, we will write $[m,n]$ for $[(m,n)]\in X/\equiv$. Prove that the following operations do not depend on class representatives:
      \begin{enumerate}
        \item $[m,n] + [p,q] := [mq+np, nq]$
        \item $[m,n]\cdot [p,q] := [mp, nq]$
      \end{enumerate}
    \item Prove that:
      \begin{enumerate}
        \item $[m,n]=[am,an]$
        \item $[0,1]+[m,n]=[m,n]$
        \item $[1,1]\cdot [m,n]=[m,n]$
        \item $[m,n] + [-m,n]=[0,1]$
        \item if $[a,b]\neq [0,1]$, then $[a,b]\cdot [b,a]=[1,1]$
      \end{enumerate}
    \item Consider any rational numbers $m/n$ and $p/q$. What equivalence classes do they correspond to? What is their sum and product? Do you see now how we can construct rationals using integers only?
  \end{enumerate}
\end{exercise}

The last example and exercise showed us how to move algebraic structures from one set to another (usually corresponding to equivalence classes of some relation). In fact one can define integers using natural numbers only\footnote{This is even simpler - our equivalence classes are 1-element. Consider $\N\times \{0,1\}$ with $(n,0)$ corresponding to $n$
and $(n,1)$ corresponding to $-n$. Figure how to define addition, subtraction and multiplication. Later we will also discover how to construct reals from rationals.} or reals from rationals\footnote{This actually involves equivalence classes, put on sequences of rationals. We will investigate this construction later.}.
