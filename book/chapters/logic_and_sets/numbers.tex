\section{Numbers}

At the beginning we assumed that you know how to work with different kinds of numbers. This aim of this section is to give a few properties of them.

\subsection{Real numbers}
\begin{definition}
  A \textbf{field} is a tuple $(\F, +, \cdot)$ such that:
  \begin{itemize}
    \item $\F$ is a set
    \item $+$ and $\cdot$ are functions from $\F\times \F$ to $\F$. We write
      $a+b$ for $+(a,b)$ and $a\cdot b$ for $\cdot (a,b)$.
  \end{itemize}
  They have these properties:
  \begin{enumerate}
    \item $a+(b+c)=(a+b)+c$ for all $a,b,c$ (addition is associative)
    \item $a\cdot (b\cdot c)=(a\cdot b)\cdot c$ for all $a,b,c$ (multiplication is associative)
    \item $a+b=b+a$ for all $a, b$ (addition is commutative)
    \item $a\cdot b=b\cdot a$ for all $a,b$ (multiplication is commutative)
    \item there is an element $0\in\F$ such that $a+0=a$ for all $a$ (addition has a neutral element). We call it \textbf{zero}
    \item there is an element $1\in\F$ such that $a\cdot 1=a$ for all $a$ (so 1 is neutral element of multiplication). We call if \textbf{one}
    \item for every $a$ there is $a'$ such that $a+a'=0$ (existence of
      an inverse element for addition)
    \item $a\cdot (b+c) = a\cdot b + a\cdot c$ for all $a,b, c$
      (multiplication distributesover addition)
    \item for every $a\neq 0$ there is $\tilde a$ such that $a\cdot \tilde a=1$
      (multiplication has an inverse element for all non-zero numbers)
    \item $1\neq 0$ (so $\F$ has at least two elements)
  \end{enumerate}

  Usually we will reference a field just as $\F$.
\end{definition}

\begin{exercise}
  Check that
  \begin{enumerate}
    \item real numbers understood informally, have the field properties
    \item rational numbers form a field
  \end{enumerate}
\end{exercise}

From the above field axioms, you can derive many facts that may be obvious
to you:
\begin{exercise}
  Prove that there is 0 and 1 are unique. Hint: assume that
  0 and 0' have property such that $a=a+0=a+0'$ and try $a=0$ and $a=0'$.
\end{exercise}

\begin{exercise}
  Prove that if $a+a'=0$ and $a+a''=0$, then $a'=a''$.
  Therefore we can introduce special symbol for \textit{the} additiv
  inverse: $a + (-a)=0$ and define subtraction as $a-b := a + (-b)$.
\end{exercise}

\begin{exercise}
  Prove that $-a=(-1)\cdot a$.
\end{exercise}

% Consider more problems on this topic

As you see, many of the algebraic properties we are used to can be recovered from the axioms, but sometimes it can be complicated. Both real numbers and
rational numbers have also an order on them - for example $2>1$. It leads
to the definition of \textit{total order}.

\begin{definition}
  We call a pair $(\F, \le)$ a \textit{totally ordered set} if for every $a,b\in \F$ we have:
  \begin{enumerate}
    \item $a\le b$ or $b\le a$ (we call this property totality)
    \item $a\le b$ and $b\le a$ imply $a=b$ (it's called antisymmetry)
    \item $a\le b$ and $b\le c$ imply $a\le c$ (transitivity)
  \end{enumerate}
  Having relation $\le$ we can define others: $b\ge a$ means that $a\le b$ and
  $a<b$ means that $a\le b$ and $a\neq b$.
\end{definition}

\begin{definition}
  \textbf{Ordered field} is a field $\F$ with a total order such that:
  \begin{itemize}
    \item $a\le b$ implies $a+c\le b+c$
    \item $0\le a$ and $0\le b$ imply that $0\le a\cdot b$
  \end{itemize}
  We say that $\F_+ := \{x\in \F : x \le 0\}$ is a set of \textbf{non-negative} numbers.
\end{definition}

\begin{exercise}
  Check that non-negative numbers are closed under summation and multiplication (the sum and product of two non-negative numbers is non-negative).
\end{exercise}

\begin{definition}
  Consider a subset of a totally-ordered set $A\subseteq \F$. We say that $x$ is an
  \textbf{upper bound} of $A$ iff $x\ge a$ for every $a\in A$.
  In such case we say that $A$ \textbf{is bounded from above}.
\end{definition}

\begin{exercise}
  Let $\F$ be either $\R$ or $\Q$. Prove that a set $A\subseteq \mathbb \F$ can have no upper bounds
  or infinitely many of them.
\end{exercise}

\begin{definition}
  Let $A\subseteq \F$ be a set bounded from above. It's \textbf{supremum} or \textbf{lower upper bound} is a number $\sup A\in F$ such that:
  \begin{itemize}
    \item $\sup A$ is an upper bound of $\F$
    \item if $y$ is an upper bound of $A$, then $y\ge \sup A$
  \end{itemize}
  Supremum is often referenced as $\text{lub}$ or $\text{l.u.b}$.
\end{definition}

\begin{exercise}
  Prove that supremum is unique, so if $x$ and $x'$ are supremums
  of $A$, then $x=x'$.
\end{exercise}

\begin{exercise}
  Prove that $x=\sup A$ if and only if
  $x\ge a$ for every $a\in A$ and for every $\eps > 0$ there is
  $a\in A$ such that $x < a + \eps$.
\end{exercise}

\begin{definition}
  \textbf{Completeness axiom} - each non-empty, bounded from above subset of $\R$ has a supremum.
\end{definition}

This axiom allows us to prove many interesting things:

\begin{exercise}
  Prove that natural numbers are \textit{not} bounded from above.
  Hint: if $n\in \mathbb N$, then $n+1\in \mathbb N$.
\end{exercise}

\begin{exercise}
  Prove the \textbf{Archimedean axiom}\footnote{In fact we do not
  need to call it axiom, as we are able to prove it.}
  that for every $r\in R$, there is $n\in \mathbb N$ such that $n>r$.
\end{exercise}

\begin{exercise}
  Prove that for any $r>0$ there is $n\in \mathbb N$ such that
  $1/n < r$.
\end{exercise}

\begin{exercise} Find infinite sum and intersection for the families of subsets of $\mathbb{R}$:
  \begin{enumerate}
    \item $A_i=(0,1/i)$ for $i=1,2,\dots$
    \item $B_i=[0,1/i)$ for $i=1,2,\dots$
  \end{enumerate}
\end{exercise}

\begin{exercise}
  Prove that rational numbers do \textit{not} have the completeness
  property:
  \begin{enumerate}
    \item Let $p, q\in \mathbb Z\setminus \{0\}$.
      Prove that $p^2\neq 2q^2$.
    \item Prove that root of two, defined as
      $x > 0, x^2=2$ is not rational.
    \item Find a subset of $\mathbb Q$ that is bounded above, but
      has no rational supremum.
  \end{enumerate}
\end{exercise}

\begin{exercise}
  You should prove that in each nonempty interval there is at least
  one rational number:
  \begin{enumerate}
    \item Assume that $0<a<b$. Define
      $$A=\left\{\frac m N : m\in \mathbb N\right\},~
      \frac 1{b-a} < N \in \mathbb N$$
      and prove that $A\cap (a,b)$ is non-empty.
    \item Use the above result to prove that in \textit{each}
      interval there is at least one rational number.
    \item Prove that in each interval there are infinitely but countably many, rational numbers.
    \item Prove that in each interval there is an irrational number.
    \item How many irrational numbers are in each interval?
  \end{enumerate}
\end{exercise}

\subsection{Absolute value}
Another concept that will be further useful is the \textbf{absolute value} of a real number:
if $x\in \mathbb R$ we write $|x|\in \mathbb R$ for:
$$|x| = \begin{cases}x &\text{ for } x \ge 0\\ -x &\text{ otherwise} \end{cases}.$$

\begin{exercise}
  Prove that for every $x,y\in \mathbb R$:
  \begin{enumerate}
    \item $|x|=|-x|$
    \item if $|x|=|y|$ then $x=y$ or $x=-y$.
    \item $|x+y| \le |x| + |y|$ (this is called \textbf{triangle inequality})
    \item $|x-y|\le |x| + |y|$
    \item $\left||x| - |y|\right|\le |x-y|$ (this is sometimes calles \textbf{reverse triangle inequality})
  \end{enumerate}
\end{exercise}

\subsection{Mathematical induction}
We characterise natural numbers as a subset of $\R$ with some properties.

\begin{definition}
  Let $A\subseteq \R$ be a set such that:
  \begin{itemize}
    \item $0\in A$
    \item if $n\in A$, then $n+1\in A$
  \end{itemize}
  The intersection of all sets with this property is called \textbf{the set of natural numbers}. Alternatively, natural numbers is the smallest set (in sense of subset order)
  with the properties listed above.
\end{definition}

These properties allow us to define a powerful proof technique:

\begin{definition}
  \textbf{Principle of mathematical induction} -
  let $f$ be a logical statement defined for all natural numbers (that is $f(n)$ is either true or false). Then if $f(0)$ is true and an implication:
  $$f(n)\Rightarrow f(n+1)$$
  is true for every $n$, then $f(n)$ is true for every $n$.
\end{definition}

Have you ever seen falling dominoes? To be sure that all domino falls, we need to:
\begin{enumerate}
	\item punch the first domino
	\item every domino must punch the next domino (if this particular domino falls, the next one also falls)
\end{enumerate}
And that's all, we can be sure that all the dominoes will eventually fall.

\begin{example}
  We'll prove that $2| n(n+1)$ for every $n\in \N$ ($a|b$ means: $a$ divides $b$).
  \begin{enumerate}
    \item the statement is true for $0$: $2 | 0\cdot 1$ as $0=2\cdot 0$
    \item I need to prove that $(2|n(n+1))\Rightarrow (2|(n+1)(n+2))$ for every $n$. Assume that $n$ is such a number that $2|n(n+1)$.
    Then $(n+1)(n+2)=n(n+1) + 2\cdot (n+1)$ is divisible by 2, what we needed to prove.
  \end{enumerate}
  Therefore from the principle of mathematical induction we know that for all number of the form $n(n+1)$ are divisible by 2\footnote{There is also an alternative proof: it's
  a product of two consecutive numbers - one of them is divisible by 2 and so is the product.}.
\end{example}

\begin{exercise}
	Prove that $2^n>n$ for every natural number $n$.
\end{exercise}

\noindent You can also modify slightly the induction principle - sometimes you should start with number different than 0 or use different induction step
(start 0 and step 2 can lead to theorems valid for even numbers, step 0 and steps 1 and -1 can lead to theorems valid for all integers...)
\begin{exercise}
    \begin{enumerate}
	   \item Prove\footnote{Another method is to notice that $n^3-n=(n-1)\cdot n\cdot (n+1)$. Why 2 does divide it? Why 3?} that 6 divides
		     $n^3-n$ for all natural $n$.
	    \item Prove\footnote{How $n^3-n$ and $(-n)^3-(-n)$ are related? Does this simplify the proof?} that 6 divides $n^3-n$ for all integers $n$.
		      You can use a slight modification mathematical induction principle proving the implication
		      ,,if the theorem works for $n$, it works also for $n-1$".
    \end{enumerate}
\end{exercise}

\begin{exercise}
	(Bernoulli's inequality) Prove that for real $x > -1$ and natural $n\ge 1$, the following inequality holds:
	$$(1+x)^n\ge 1+nx.$$
\end{exercise}

\begin{exercise}
	In Mathsland there are $n\ge 2$ cities. Between each pair of them there is a \textit{one-way} road.
	\begin{enumerate}
		\item Prove that there is a city from which you can drive to all the other cities. Hint: assume that the hypothesis works for some $n$ and any
			country with $n$ cities. Now consider an arbitrary $n+1$-city country. Hide one city and use your assumption.
		\item Prove that there is a city\footnote{Nice trick: what does happen if you reverse each way? Can you use the former result?}
			to which you can drive from all the others.
	\end{enumerate}
\end{exercise}

\begin{exercise}
	Let $S\subseteq R$. We say that $S$ is \textbf{well-ordered} iff any non-empty subset $X\subset S$ has the smallest element.
	\begin{enumerate}
		\item Prove that reals and integers with the default ordering are not well-ordered.
		\item Assume that $X\subseteq \mathbb N$ doesn't have the smallest element. Define $A=\{n\in \mathbb N : \{0,1,\dots,n\}\cap X=\emptyset\}$
			and use mathematical induction to prove that $X$ is empty.
		\item Why are natural numbers well-ordered?
	\end{enumerate}
\end{exercise}
