\section{Functions}
\label{sec:intro_to_functions}

\begin{definition}
  Consider two sets $A$ and $B$. We say that a relation $f$ (that is a subset $f\subseteq A\times B$) is a \textbf{function}
  iff the following two conditions hold:
  \begin{itemize}
    \item for every element $a\in A$ there is an element $b\in B$ such that $(a,b)\in f$
    \item if $(a,b)\in f$ and $(a,c)\in f$, then $b=c$
  \end{itemize}
  Therefore for each $a\in A$ there is exactly one $b\in B$ such that $(a,b)\in f$. Such $b$ will be called \textbf{value of $f$ at point $a$} and given a symbol $f(a).$
  We will frite $f: A\to B$ for $f$ and call $A$ the \textbf{domain of $f$} and $B$ the \textbf{codomain of $f$}.

  Being very concise we can also write $f$ as
  $$f: A\ni a \mapsto f(a)\in B.$$
  Note that we use two different arrows.
\end{definition}

\begin{example}
  $f: \N\to \R$ given by $f(n)=n^2$. We can also write: $$f: \N\ni n\mapsto n^2\in \R.$$
\end{example}

\begin{example}
  $f: \R\to \R$ given by $f(x)=x^{10}+x^2-1$.
\end{example}

\begin{example}
  $f: X\to \mathcal P(X)$ given by $f(x)=\{x\}$.
\end{example}

\begin{exercise}
  Let $X$ and $Y$ be two sets. Prove that there exists a set of all functions from $X$ to $Y$. Hint: you can form a set of all relations between $X$ and $Y$. How are functions related
  to relations?
\end{exercise}

\begin{exercise}
	How many\footnote{Thanks to Antek Hanke} are there functions from the empty set to $\{1,2,3,4\}$? Hint: what is a function in set-theoretical terms?
\end{exercise}

\begin{exercise}
  Here, we will prove a simple inequality using a set-theoretic reasoning. Let $X$ and $Y$ be finite sets, with numbers of elements, respectively, $x=|X|$ and $y=|Y|$.
  \begin{enumerate}
    \item Prove that the number of relations between $X$ and $Y$ is $2^{xy}$.
    \item Prove that the number of functions from $X$ to $Y$ is $y^x$. Hint: for first element in $X$ you have $y$ possibilities to choose.
    \item Prove that for every non-zero natural numbers $x$ and $y$ the following holds:
      $$y^x<2^{xy}.$$
  \end{enumerate}
\end{exercise}

\begin{exercise}
  Let $X$ and $Y$ be any two sets. Prove that you can create a set of all functions from $X$ to $Y$. Sometimes it is called $Y^X$. Do you know why?
\end{exercise}

\begin{exercise}
  Consider a function $f: X\to X'$ and assume that there is an equivalence relation $R'$ on $X'$. We will try to define a natural (in some sense) equivalence relation on $X$.
  \begin{enumerate}
    \item Define a relation $R$ on $X$ as $xRy\Leftrightarrow f(x) R' f(y)$. Prove that it is an equivalence relation.
    \item Consider $r: X\to X/R$ and $r': X'\to X'/R'$ given by $r(x)=[x]_R$ and $r'(x')=[x']_{R'}$ and inverse function.
  \end{enumerate}
\end{exercise}

\begin{prob}
	Let $f: A\to B$ and $C\subseteq D\subseteq A$. We define: $f[C] = \{b\in B : b=f(c) \text{ for some }c\in C \}$ and analogously $f[D]$. Prove that
	$f(C)\subseteq f(D).$
\end{prob}

\begin{definition}
  Consider a set $X$. We say that it's \textbf{identity function} is $f:X\to X$ given by $f(x)=x$ for all $x\in X$.
\end{definition}

\subsection{Injectivity, surjectivity and bijectivity}

\noindent As we have already seen, there may be some elements in codomain that are not values of $f$. Such a set is important enough to be given a name:

\begin{definition}
  Let $f:A\to B$ be a function. \textbf{The image of $f$} is a set:
  $$\Image f=\{b\in B : \text{there is } a\in A \text{ such that } b=f(a)\}.$$

  We say that the function $f: A\to B$ is \textbf{surjective} (or \textbf{onto}) iff $\Image f=B$.
\end{definition}

\begin{prob}
	As we remember, $\mathbb{R}$ stands for real numbers. Are the following functions surjective?
	\begin{enumerate}
		\item $f: \mathbb{R} \to \mathbb{R}, ~f(x)=x^3$
		\item $g: \mathbb{R} \to \mathbb{R}, ~g(x)=x^2$
		\item $h: \mathbb{R} \to \{5\}$
	\end{enumerate}
\end{prob}

\begin{definition}
  Let $f:A\to B$ be a function. If $f$ gives distinct values to distinct arguments (that is, if $f(a)=f(b)$, then $a=b$), we say that the function is \textbf{injective}
  (or \textbf{one-to-one}).
\end{definition}

\begin{exercise}
  Are the following functions injective?
	\begin{enumerate}
		\item $f: \mathbb{R} \to \mathbb R, ~f(x)=x^2$
		\item $h: \{0,1,2,3\} \to \mathbb R, ~h(x)=x$
	\end{enumerate}
\end{exercise}

\begin{exercise}
	Let $f$ be a function from $A$ to $B$. Prove that there exists a function $g: \Image B\to A$ such that $g\circ f=\Id_A$ iff $f$ is injective.
\end{exercise}

\begin{exercise}
	Let $f: A\to B$ and $g: B\to C$ be functions such that $g\circ f$ is injective but $g$ is not. Why isn't $f$ surjective?
\end{exercise}

\begin{definition}
If a function $f$ is both surjective and injective, we say that is \textbf{bijective}\footnote{If you prefer nouns: surjective function is called a surjection, injective - injection
and bijective - bijection}.
\end{definition}

\begin{exercise}
	Construct a function that is:
	\begin{enumerate}
		\item surjective, but not injective
		\item injective, but not surjective
		\item neither injective nor surjective
		\item bijective
	\end{enumerate}
\end{exercise}

\noindent Notice that if a function $f: A\to B$ is bijective, then we can construct a function $g:B\to A$
such that $f(g(b))=b$ and $g(f(a))=a$.

\begin{prob}
	Prove that, if exists, $g$ is unique.
\end{prob}

\begin{definition}
  Consider a bijective function $f:X\to Y$. We say that it's \textbf{inverse function} $f^{-1}:Y\to X$ iff:
  $$f^{-1}(f(x))=x, f(f^{-1}(y))=y,$$
  for all $x\in X,\, y\in Y$.
\end{definition}

\noindent We call this function \textbf{the inverse function}
\footnote{It becomes confusing when working on real numbers: $f^{-1}(x)$ is
\textbf{not} $(f(x))^{-1}=1/f(x)$}: $g=f^{-1}.$

\begin{prob}
	Assume that $f^{-1}$ exists. Prove that $(f^{-1})^{-1}$ exists and is equal to $f$.
\end{prob}

\subsection{Function composition}
If we have two functions: $f:A\to B$ and $g: B\to C$, we can construct the \textbf{composition} using formula:
$g\circ f: A\to C,~(g\circ f)(a) = g(f(a)).$

\begin{exercise}
  Recall that for two relations $R\subseteq X\times Y$ and $T\subseteq Y\times Z$ we defined their composition as $$R\circ T=\{(x,z)\in X\times Z : \exists_{y\in Y} (x,y)\in R \wedge (y,z)\in T\}$$
\end{exercise}

\begin{exercise}
	Find functions $f,~g$ such that:
	\begin{enumerate}
		\item $g\circ f$ exists, but $f\circ g$ is not defined
		\item both $f\circ g$ and $g\circ f$ exist, but $f\circ g\neq g\circ f$
	\end{enumerate}
\end{exercise}

Although function composition is not commutative, it is associative:
\begin{exercise}
	Left $f:A\to B, g: B\to C, h: C\to D$. Prove that
	$$h\circ (g\circ f) = (h\circ g)\circ f.$$
\end{exercise}
Therefore we can ommit the brackets and write just $h\circ g\circ f.$ We will use function composition very
often.

\begin{exercise}
    \begin{enumerate}
	   \item Prove that composition of two surjections is surjective.
	   \item Prove that composition of two injections is injective.
	   \item Prove that composition of two bijections is bijective.
    \end{enumerate}
\end{exercise}

\begin{definition} We will rephrase the definition of the inverse function using the identity function\footnote{For a set $X$, it's identity function is
    $$\Id_X=\{(x,x)\in X\times X : x\in X\}.$$}:

    consider a function $f:X\to Y$. If there exists a function $f^{-1}:Y\to X$ such that:

    $$f^{-1}\circ f=\Id_X,  f\circ f^{-1}=\Id_Y,$$

    we say that $f^{-1}$ if \textbf{the inverse} to $f$.
\end{definition}

\begin{exercise}
  Let $f: A\to B$ be an injection. Prove that there is a function
  $g: \text{Im\,} f \to A$ such that $g\circ f = \text{Id}_A.$
  Such $g$ is called \textbf{left inverse of $f$}.
\end{exercise}

\subsection{Commutative diagrams}
\begin{quote}
  Use a picture. It's worth a thousand words.\\
  - Tess Flanders
\end{quote}

Consider functions $f:X\to Y$ and $g:Y\to Z$. We introduced the composition of them given us $g\circ f: X\to Z$. We can visualise it using a following \textbf{diagram} (Fig. \ref{fig:commutative_diagram_intro}):

\begin{figure}
  \centering
  \begin{tikzcd}
    X \arrow[rd, "g\circ f"] \arrow[rr, "f"] &       & Y \arrow[ld, "g"]\\
                                           &  Z   &
  \end{tikzcd}
  \caption{An example of a diagram.}
  \label{fig:commutative_diagram_intro}
\end{figure}

We say that this diagram \textbf{commutes} (or we say that this is a \textbf{commutative diagram}) as you can use follow any path and obtain the same result.

\begin{exercise}
  Prove that the diagram \ref{fig:commutative_diagram_composition} commutes iff $h=g\circ f$?

  \begin{figure}
    \centering
    \begin{tikzcd}
      X \arrow[rd, "h"] \arrow[rr, "f"] &       & Y \arrow[ld, "g"]\\
                                             &  Z   &
    \end{tikzcd}

    \caption{What can you say about $f,g,h$ if the diagram commutes?}
    \label{fig:commutative_diagram_composition}

  \end{figure}
\end{exercise}

\begin{exercise}
  What can you say if diagram \ref{fig:commutative_diagram_toy_natural_transformation} commutes?

  \begin{figure}
    \centering
    \begin{tikzcd}
      X \arrow[d, "h"] \arrow[r, "f"] &  Y \arrow[d, "g"]\\
      Z \arrow[r, "j"]                &  T
    \end{tikzcd}

    \caption{What can you say about the functions involved if the diagram commutes?}
    \label{fig:commutative_diagram_toy_natural_transformation}

  \end{figure}
\end{exercise}
